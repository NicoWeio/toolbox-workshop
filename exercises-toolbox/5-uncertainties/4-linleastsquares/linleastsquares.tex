\documentclass{scrartcl}

\usepackage[main=ngerman]{babel}

\usepackage{fontspec}

\usepackage{mathtools}
\usepackage[
  math-style=ISO,
  bold-style=ISO,
]{unicode-math}

\usepackage{csquotes}
\usepackage[locale=DE]{siunitx}

\usepackage{hyperref}

\begin{document}

\section*{Die Methode der kleinsten Quadrate für Linearkombinationen von Funktionen}

Zur Anpassung einer Linearkombinationen aus Funktionen
\begin{equation}
  \sum_i^p a_i \cdot f_i(x)
\end{equation}
an $N$ Datenpunkte $(x_j, y_j)$ kann die analytische Methode der kleinsten Quadrate verwendet werden.
Ziel ist es, die Parameter $a_i$ zu bestimmen, bei denen die Summe
\begin{equation}
  \sum_j^N\left( y_j - \sum_i^p a_i\cdot f_i(x_j)\right)^{\!\!\! 2}
\end{equation}
minimal ist, der Funktionsgraph also minimal von den Datenpunkten abweicht.

\subsection*{Bestimmung der Parameter}

\begin{enumerate}
  \item Zuerst wird die sogenannte Designmatrix $\symbf{A}$ aufgestellt.
    Sie enthält die Funktionswerte für jede Funktion $f_i$ ausgewertet an den gemessenen $x_j$:
    \begin{equation}
      \symbf{A} =
      \begin{pmatrix}
        f_1(x_1) & f_2(x_1) & \cdots & f_p(x_1) \\
        f_1(x_2) & f_2(x_2) & \cdots & f_p(x_2) \\
        \vdots   &          &        &  \vdots  \\
        f_1(x_N) & f_2(x_N) & \cdots & f_p(x_N)
      \end{pmatrix} \ .
    \end{equation}
  \item Als nächstes definieren wir den Spalten-Vektor
    \begin{equation}
      \vec{y} = (y_1, y_2, …, y_N)^{\top} \ ,
    \end{equation}
    der die $y$-Koordinaten unserer Messwerte enthält.
  \item Die Kovarianzmatrix der Messwerte sei $\symbf{W}$ und $\symbf{Z} = \symbf{W}^{-1}$.
  \item der Parametervektor $\vec{a}$ ergibt sich dann zu:
    \begin{equation}
      \vec{a} = \left(\symbf{A}^\top \symbf{Z} \symbf{A}\right)^{-1} \cdot \symbf{A}^\top \symbf{Z} \cdot \vec{y}
    \end{equation}
  \item Die Kovarianzmatrix ergibt sich zu:
    \begin{equation}
      \symbf{V} = \left(\symbf{A}^\top \symbf{Z} \symbf{A}\right)^{-1}
    \end{equation}
\end{enumerate}
Wer sich für die Herleitung interessiert: Blobel-Lohrmann: \enquote{Statistische und numerische Methoden der Datenanalyse} bzw. SMD-Vorlesung.
Alternativ auch bei Wikipedia: \url{https://en.wikipedia.org/wiki/Linear_least_squares_(mathematics)}

\subsection*{Aufgaben}

\begin{enumerate}
  \item Implementiert eine Funktion, die eine Liste von Funktionen und die $x$- und $y$-Werte übergeben bekommt und die lineare Methode der kleinsten Quadrate anwendet.
    Die Funktion soll den Parametervektor als Array von korrelierten \texttt{ufloat}s zurückgeben.
  \item Testet eure Funktion indem ihr einen Fit der Form
    \begin{equation}
      \Psi(x) = a_1 \cos(x) + a_2 \sin(x)
    \end{equation}
    an die Daten in der Datei \texttt{daten.txt} durchführt.
    Die Unsicherheiten der $y_i$ sollen als $\sigma = \num{0.1}$ angenommen werden.
\end{enumerate}

\subsection*{Tipps}

\begin{itemize}
  \item Matrix-Multiplikation ziwschen numpy arrays mit dem \texttt{@} Operator
  \item Zum invertieren einer Matrix kann \texttt{numpy.linalg.inv} genutzt werden.
\end{itemize}

\end{document}

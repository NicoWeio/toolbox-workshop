\subsection{Mathe-Umgebungen – \texttt{amsmath}}

\begin{frame}[fragile]{
  amsmath und mathtools:
  \hfill\doc{http://mirrors.ctan.org/macros/latex/required/amslatex/math/amsldoc.pdf}{amsmath}
  \doc{http://mirrors.ctan.org/macros/latex/contrib/mathtools/mathtools.pdf}{mathtools}
}
  \begin{block}{benötigte Pakete}
    \begin{lstverbatim}
    \usepackage{amsmath}
    \usepackage{amssymb}
    \usepackage{mathtools}
    \end{lstverbatim}
  \end{block}
  \begin{itemize}
      \item Stellen Matheumgebungen für alles was man so braucht zur Verfügung.
      \item Alle Gleichungen werden automatisch nummeriert.
      \item * nach dem Umgebungsnamen sorgt für unnumerierte Gleichung
  \end{itemize}
\end{frame}

\begin{frame}[fragile]{Die \texttt{equation}-Umgebung}
  Abgesetzte Gleichung, automatische Nummerierung. \\
  \texttt{equation*} erzeugt unnummerierte Gleichung.
  \begin{columns}[T, onlytextwidth]
    \begin{column}{0.6\textwidth}
      \begin{block}{\LaTeX-Code}
        \begin{lstverbatim}
        \begin{equation}
          \nabla \cdot \vec{E} =
          \frac{\rho} {\epsilon_0}
          \label{eqn:maxwell1}
        \end{equation}
        Schon Gauß hatte das Durchflutungsgesetz \eqref{eqn:maxwell1} aufgestellt.
        \end{lstverbatim}
      \end{block}
    \end{column}
    \begin{column}{0.35\textwidth}
      \begin{block}{Ergebnis}
        \begin{equation}
          \nabla \cdot \vec{E} =
          \frac{\rho}{\epsilon_0}
          \label{eqn:maxwell1}
        \end{equation}
        Schon Gauß hatte das Durchflutungsgesetz \eqref{eqn:maxwell1} aufgestellt.
      \end{block}
    \end{column}
  \end{columns}
\end{frame}

\begin{frame}[fragile]{Die \texttt{gather}-Umgebung}
    \begin{itemize}
        \item für mehrere Gleichungen
        \item \texttt{\textbackslash\textbackslash} erzeugt neue Zeile
        \item jede Zeile bekommt eine Gleichungsnummer
    \end{itemize}
  \begin{columns}[T]
    \begin{column}{0.53\textwidth}
      \begin{block}{\LaTeX-Code}
        \begin{lstverbatim}
        \begin{gather}
            (a + b)^2 = a^2 + 2ab + b^2 \\
            (a - b)^2 = a^2 - 2ab + b^2 \\
            (a + b) \cdot (a-b) = a^2 -  b^2
        \end{gather}
        \end{lstverbatim}
      \end{block}
    \end{column}
    \begin{column}{0.43\textwidth}
      \begin{block}{Ergebnis}
        \begin{gather}
            (a + b)^2 = a^2 + 2ab + b^2 \\
            (a - b)^2 = a^2 - 2ab + b^2 \\
            (a + b) \cdot (a-b) = a^2 -  b^2 
        \end{gather}
      \end{block}
    \end{column}
  \end{columns}
\end{frame}

\begin{frame}[fragile]{Die \texttt{align}-Umgebung}
    \begin{itemize}
        \item für mehrere Gleichungen, die aneinander ausgerichtet werden
        \item \texttt{\&} steuert Ausrichtung
        \item \texttt{\textbackslash\textbackslash} erzeugt neue Zeile
        \item jede Zeile bekommt eine Gleichungsnummer
    \end{itemize}
  \begin{columns}[T]
    \begin{column}{0.65\textwidth}
      \begin{block}{\LaTeX-Code}
        \begin{lstverbatim}
        \begin{align}
          a         &= 1 & b           &= 2 \\
          a \cdot b &= 5 & \frac{a}{b} &= 0.5
        \end{align}
        \end{lstverbatim}
      \end{block}
    \end{column}
    \begin{column}{0.3\textwidth}
      \begin{block}{Ergebnis}
        \begin{align}
          a         &= 1 & b           &= 2 \\
          a \cdot b &= 2 & \frac{a}{b} &= 0.5
        \end{align}
      \end{block}
    \end{column}
  \end{columns}
\end{frame}

\begin{frame}[fragile]{Die \texttt{split}-Umgebung}
    \begin{itemize}
        \item um überlange Gleichungen auf zwei Zeilen aufzuteilen.
        \item kommt in den anderen Umgebungen zum Einsatz
        \item \texttt{\&} steuert Ausrichtung
        \item \texttt{\textbackslash\textbackslash} erzeugt neue Zeile
        \item gemeinsame Gleichungsnummer
    \end{itemize}
  \begin{columns}[T]
    \begin{column}{0.55\textwidth}
      \begin{block}{\LaTeX-Code}
        \begin{lstverbatim}
        \begin{equation}
            \begin{split}
                (a+b)^3 & = a^3 + 3a^2b \\
                        & \quad +3ab^2 + b^3
            \end{split}
        \end{equation}
        \end{lstverbatim}
      \end{block}
    \end{column}
    \begin{column}{0.40\textwidth}
      \begin{block}{Ergebnis}
          \vspace{0.5em}
        \begin{equation}
            \begin{split}
                (a+b)^3 & = a^3 + 3a^2b \\
                        & \quad +3ab^2 + b^3
            \end{split}
        \end{equation}
      \end{block}
    \end{column}
  \end{columns}
\end{frame}

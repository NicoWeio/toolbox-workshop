\subsection{Mathe-Umgebungen – \texttt{amsmath}}

\begin{frame}[fragile]{
  Die \texttt{equation}-Umgebung
  \hfill\doc{http://mirrors.ctan.org/macros/latex/required/amslatex/math/amsldoc.pdf}{amsmath}
  \doc{http://mirrors.ctan.org/macros/latex/contrib/mathtools/mathtools.pdf}{mathtools}
}
  \begin{block}{benötigte Pakete}
    \begin{lstverbatim}
    \usepackage{amsmath}
    \usepackage{amssymb}
    \usepackage{mathtools}
    \end{lstverbatim}
  \end{block}
  Abgesetzte Gleichung, automatische Nummerierung. \\
  \texttt{equation*} erzeugt unnummerierte Gleichung.
  \begin{columns}[T]
    \begin{column}{0.6\textwidth}
      \begin{block}{\LaTeX-Code}
        \begin{lstverbatim}
        \begin{equation}
          \nabla \cdot \vec{E} =
          \frac{\rho} {\epsilon_0}
          \label{eqn:maxwell1}
        \end{equation}
        Schon Gauß hatte das Durchflutungsgesetz \eqref{eqn:maxwell1} aufgestellt.
        \end{lstverbatim}
      \end{block}
    \end{column}
    \begin{column}{0.35\textwidth}
      \begin{block}{Ergebnis}
        \begin{equation}
          \nabla \cdot \vec{E} =
          \frac{\rho}{\epsilon_0}
          \label{eqn:maxwell1}
        \end{equation}
        Schon Gauß hatte das Durchflutungsgesetz \eqref{eqn:maxwell1} aufgestellt.
      \end{block}
    \end{column}
  \end{columns}
\end{frame}

\begin{frame}[fragile]{
  Die \texttt{align}-Umgebung
  \hfill\doc{http://mirrors.ctan.org/macros/latex/required/amslatex/math/amsldoc.pdf}{amsmath}
}
  \begin{columns}[t]
    \begin{column}{0.35\textwidth}
      \begin{itemize}
        \item für mehrere Gleichungen
        \item \texttt{\&} steuert Ausrichtung
      \end{itemize}
    \end{column}
    \begin{column}{0.6\textwidth}
      \begin{itemize}
        \item \texttt{\textbackslash\textbackslash} erzeugt neue Zeile
        \item jede Zeile bekommt eine Gleichungsnummer
      \end{itemize}
    \end{column}
  \end{columns}
  \vfill
  \begin{columns}[T]
    \begin{column}{0.65\textwidth}
      \begin{block}{\LaTeX-Code}
        \begin{lstverbatim}
        \begin{align}
          a         &= 1 & b           &= 2 \\
          a \cdot b &= 5 & \frac{a}{b} &= \num{0,5}
        \end{align}
        \end{lstverbatim}
      \end{block}
    \end{column}
    \begin{column}{0.3\textwidth}
      \begin{block}{Ergebnis}
        \begin{align}
          a         &= 1 & b           &= 2 \\
          a \cdot b &= 2 & \frac{a}{b} &= \num{0,5}
        \end{align}
      \end{block}
    \end{column}
  \end{columns}
\end{frame}

\begin{frame}{Übung: Maxwell-Gleichungen}
  Erstellt mit Hilfe der \texttt{align}-Umgebung die Maxwellgleichungen:
  \begin{block}{Ergebnis}
    \begin{align}
      \nabla \cdot \vec{E} &= \frac{\rho} {\epsilon_0} &
      \nabla \cdot \vec{B} &= 0 \\
      \nabla \times \vec{E} &= - \partial_t \vec{B} &
      \nabla \times \vec{B} &= \mu_0 \vec{\jmath} + \mu_0 \epsilon_0 \partial_t \vec{E}
      \label{align}
    \end{align}
  \end{block}
\end{frame}

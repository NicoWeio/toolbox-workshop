\subsection{Mathe-Umgebungen – \texttt{amsmath}}

\begin{frame}{Math-Umgebungen}
  \begin{itemize}
    \item \texttt{amsmath} stellt Mathe-Umgebungen für alles was man so braucht zur Verfügung.
      \item Alle Gleichungen werden automatisch nummeriert.
      \item \texttt{*} nach dem Umgebungsnamen sorgt für unnumerierte Gleichung
      \item Unnumerierte Gleichungen sollten selten sein.
  \end{itemize}
\end{frame}

\begin{frame}[fragile]{Die \texttt{equation}-Umgebung}
  Abgesetzte Gleichung, automatische Nummerierung. \\
  \texttt{equation*} erzeugt unnummerierte Gleichung.
  \begin{columns}[onlytextwidth, t]
    \begin{column}{0.60\textwidth}
      \begin{block}{Code}
        \begin{lstlisting}
        \begin{equation}
          \nabla \cdot \vec{E}
            = \frac{\rho}{\varepsilon_0}
            \label{eqn:maxwell1}
        \end{equation}
        Schon Gauß hatte das Durchflutungsgesetz \eqref{eqn:maxwell1} aufgestellt.
        \end{lstlisting}
      \end{block}
    \end{column}
    \begin{column}{0.36\textwidth}
      \begin{block}{Ergebnis}
        \begin{equation}
          \nabla \cdot \vec{E} =
            = \frac{\rho}{\varepsilon_0}
            \label{eqn:maxwell1}
        \end{equation}
        Schon Gauß hatte das Durchflutungsgesetz \eqref{eqn:maxwell1} aufgestellt.
      \end{block}
    \end{column}
  \end{columns}
\end{frame}

\begin{frame}[fragile]{Die \texttt{gather}-Umgebung}
  \begin{itemize}
    \item Für mehrere Gleichungen
    \item \verb+\\+ erzeugt neue Zeile
      \begin{itemize}
        \item Kein \verb+\\+ nach letzter Zeile!
      \end{itemize}
    \item Jede Zeile bekommt eine Gleichungsnummer
  \end{itemize}
  \begin{columns}[onlytextwidth, t]
    \begin{column}{0.53\textwidth}
      \begin{block}{Code}
        \begin{lstlisting}
        \begin{gather}
          (a + b)^2 = a^2 + 2ab + b^2 \\
          (a - b)^2 = a^2 - 2ab + b^2 \\
          (a + b) \cdot (a - b) = a^2 - b^2
        \end{gather}
        \end{lstlisting}
      \end{block}
    \end{column}
    \begin{column}{0.43\textwidth}
      \begin{block}{Ergebnis}
        \begin{gather}
          (a + b)^2 = a^2 + 2ab + b^2 \\
          (a - b)^2 = a^2 - 2ab + b^2 \\
          (a + b) \cdot (a - b) = a^2 - b^2
        \end{gather}
      \end{block}
    \end{column}
  \end{columns}
\end{frame}

\begin{frame}[fragile]{Die \texttt{align}-Umgebung}
  \begin{itemize}
    \item Für mehrere Gleichungen, die aneinander ausgerichtet werden
    \item \texttt{\&} steuert Ausrichtung
    \item \verb+\\+ erzeugt neue Zeile
    \item Jede Zeile bekommt eine Gleichungsnummer
  \end{itemize}
  \begin{columns}[onlytextwidth, t]
    \begin{column}{0.65\textwidth}
      \begin{block}{Code}
        \begin{lstlisting}
        \begin{align}
          a         &= 1 & b           &= 2 \\
          a \cdot b &= 5 & \frac{a}{b} &= 0.5
        \end{align}
        \end{lstlisting}
      \end{block}
    \end{column}
    \begin{column}{0.31\textwidth}
      \begin{block}{Ergebnis}
        \begin{align}
          a         &= 1 & b           &= 2 \\
          a \cdot b &= 2 & \frac{a}{b} &= 0.5
        \end{align}
      \end{block}
    \end{column}
  \end{columns}
\end{frame}

\begin{frame}[fragile]{Die \texttt{split}-Umgebung}
  \begin{itemize}
    \item Um überlange Gleichungen auf zwei Zeilen aufzuteilen.
    \item Kommt in den anderen Umgebungen zum Einsatz
    \item \texttt{\&} steuert Ausrichtung
    \item \verb+\\+ erzeugt neue Zeile
    \item Gemeinsame Gleichungsnummer
  \end{itemize}
  \begin{columns}[onlytextwidth, t]
    \begin{column}{0.55\textwidth}
      \begin{block}{Code}
        \begin{lstlisting}
        \begin{equation}
          \begin{split}
            (a+b)^3 = {} & a^3 + 3a^2b \\
                         & + 3ab^2 + b^3
          \end{split}
        \end{equation}
        \end{lstlisting}
      \end{block}
    \end{column}
    \begin{column}{0.41\textwidth}
      \begin{block}{Ergebnis}
        \vspace{0.5em}
        \begin{equation}
          \begin{split}
            (a+b)^3 = {} & a^3 + 3a^2b \\
                         & + 3ab^2 + b^3
          \end{split}
        \end{equation}
      \end{block}
    \end{column}
  \end{columns}
\end{frame}

\section{Einführung}

\begin{frame}{Was ist \LaTeX?}
    \begin{itemize}
        \item \emph{Programmiersprache} zum Setzen von Text
        \item Kein WYSIWYG, es werden Befehle und Inhalt in normale Text-Dateien geschrieben.
        \item Kompiler überträgt \LaTeX-Code in ein Ausgabedokument (meist PDF)
        \item OpenSource mit zahlreichen Erweiterungsmöglichkeit (Pakete)
    \end{itemize}
\end{frame}

\begin{frame}{Warum \LaTeX \ benutzen?}
    \begin{itemize}
        \item hervorragender Text- und Formelsatz
        \item automatisierte Erstellung von Inhalts- und Literaturverzeichnis
        \item Tex-Dateien sind reine Text-Dateien \\
              $\Rightarrow$ kleine Datein, gut für Versionskontrolle geeignet
        \item sehr gute Vorlagen für wissenschaftliche Arbeiten 
        \item aber auch: Briefe, Notensatz, Präsentationen 
        \item ausgezeichnete Dokumentionen
        \item erweiterbar durch zahlreiche und mächtige Pakete
        \item auf allen geläufigen Betriebssystemen verfügbar
        \item Ausgabe direkt als PDF mit Hyperlinks
    \end{itemize}
\end{frame}

\begin{frame}
  \centering
  \only<1>{
    \includegraphics[height=0.95\textheight]{figures/engines-few.pdf}
    \hspace{1.43em}
  }
  \only<2>{
    \includegraphics[height=0.95\textheight]{figures/engines-many.pdf}
  }
\end{frame}

\begin{frame}[t]
  \centering
  \only<1>{
    \includegraphics[height=5.45em]{figures/formats-few.pdf}
    \hspace{0.85em}
  }
  \only<2>{
    \includegraphics[height=0.95\textheight]{figures/formats-many.pdf}
  }
\end{frame}

\section{Makros}

\begin{frame}[fragile]{Eigene \LaTeX-Kommandos}
  Nach 20 Mal \verb+\mathrm{e}+ oder \verb+\mathrm{i}+ schreiben hat man keine Lust mehr
  \begin{block}{Kommandostruktur}
    \begin{lstverbatim}
    \newcommand\Kommandoname[#Argumente]{Code mit #1, #2, ...} 
    \newenvironment{Umgebungsname}[#Argumente]{\begin-Code}{\end-Code}
    \end{lstverbatim}
  \end{block}
  \begin{columns}[T]
    \column{0.75\textwidth}
    \begin{block}{Beispiele}
      \begin{lstverbatim}
      \newcommand\I{\mathrm{i}} 
      \newcommand\t[1]{\text{#1}}
      \newcommand\dd[2]{\frac{\mathrm{d}#1}{\mathrm{d}#2}}
      \renewcommand\v[1]{\vec{#1}} % Überschreiben bereits belegter Kommandos
      \newenvironment{eqn}{\begin{equation}}{\end{equation}}
      \end{lstverbatim}
    \end{block}
    \column{0.20\textwidth}
    \begin{block}{Benutzung}
      \begin{lstverbatim}
      \I
      \t{foo}
      \dd{x}{y}
      \v{u}
      \begin{eqn} 
        ...
      \end{eqn}
      \end{lstverbatim}
    \end{block}
  \end{columns}
  \vspace{0.5cm}
  \mbox{\rightarrow} Okay für simple Kommandos, stößt aber schnell an gewisse Grenzen
\end{frame}

% \begin{frame}[fragile]{\texttt{xparse}}
%   \begin{block}{benötigte Pakete}
%     \begin{lstverbatim}
%     \usepackage{xparse}
%     \end{lstverbatim}
%   \end{block}
%   \begin{block}{Kommandostruktur}
%     \begin{lstverbatim}
%     \NewDocumentCommand\Kommandoname{m o O{...} ...}{Code mit #1, #2,...}
%     \NewDocumentEnvironment{Umgebungsname}{m o O{...} ...}
%     {
%       \begin-Code
%     }
%     {
%       \end-Code
%     }
%     \end{lstverbatim}
%   \end{block}
%   Platzhalter legen Anzahl, Typ und Reihenfolge der Argumente fest 
%   \begin{description}
%     \item[\texttt{m}] Pflichtargument
%     \item[\texttt{o}] optionales Argument, kein Defaultwert
%     \item[\texttt{O\{foo\}}] optionales Argument mit Default \texttt{foo}
%   \end{description}
% \end{frame}

% \begin{frame}[fragile]{Nützliche eigene Umgebungen}
%   \begin{block}{Code}
%     \begin{lstverbatim}
%     \NewDocumentEnvironment{eqns}{O{rCl}}
%     {
%       \begin{IEEEeqnarray}{#1}
%     }
%     {
%       \end{IEEEeqnarray}
%       \ignorespacesafterend % verhindert Einrückung nach end-Block
%     }
%     \NewDocumentEnvironment{eqn}{}  
%     {
%       \begin{eqns}[c]
%     }
%     {
%       \end{eqns}
%       \ignorespacesafterend
%     }
%     \end{lstverbatim}
%   \end{block}
% \end{frame}

% \begin{frame}[fragile]{Mehr Beispiele}
%   \begin{block}{Code}
%     \begin{lstverbatim}
%     \RenewDocumentCommand\v{m}{\mathbf{#1}}
%     \NewDocumentCommand\setN{}{\mathbb{N}}
%     \NewDocumentCommand\grad{}{\operatorname{grad}} % rm-font und passende Abstände im Mathemodus
%     \AtBeginDocument % wegen unicode-math
%     {       
%       \RenewDocumentCommand\div{}{\operatorname{div}}
%     }
%     \NewDocumentCommmand\til{m}{\ensuremath{\tilde{#1}}} % erzwingt Mathemodus; sparend einsetzen
%     \end{lstverbatim}
%   \end{block}
%   \begin{columns}[T]
%     \column{0.70\textwidth}
%     \begin{block}{Test}
%       \begin{lstverbatim}
%       $ \setN \quad \grad \v{F} \quad \div \v{A} $ \\
%       \til{x}
%       \end{lstverbatim}
%     \end{block}
%     \column{0.25\textwidth}
%     \begin{block}{Ergebnis}
%       $ \setN \quad \grad \v{F} \quad \div \v{A} $ \\
%       \til{x}
%     \end{block}
%   \end{columns}
% \end{frame}

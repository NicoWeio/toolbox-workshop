\subsection{Makros}

\begin{frame}[fragile]{Eigene \LaTeX-Kommandos}
  Nach 20 Mal \lstinline+\mathrm{e}+ oder \lstinline+\mathrm{i}+ schreiben hat man keine Lust mehr.
  \begin{tblock}{Kommandostruktur}
    \begin{lstlisting}
      \newcommand\Kommandoname[#Argumente]{Code mit #1, #2, ...}
      \newenvironment{Umgebungsname}[#Argumente]{\begin-Code}{\end-Code}
    \end{lstlisting}
  \end{tblock}
  \verb+##+ gibt ein wirkliches \verb+#+.
  \begin{CodeExample}{0.75}
    \begin{lstlisting}
      \newcommand\I{\mathrm{i}}
      \newcommand\t[1]{\text{#1}}
      \newcommand\dd[2]{\frac{\mathrm{d}#1}{\mathrm{d}#2}}
      \renewcommand\v[1]{\vec{#1}} % Überschreiben bereits belegter Kommandos
      \newenvironment{eqn}{\begin{equation}}{\end{equation}}
    \end{lstlisting}
  \CodeResult
    \begin{lstlisting}
      \I
      \t{foo}
      \dd{x}{y}
      \v{u}
      \begin{eqn}
        % .
      \end{eqn}
    \end{lstlisting}
  \end{CodeExample}
\end{frame}

% \begin{frame}[fragile]{\texttt{xparse}}
%   \begin{Packages}
%     \begin{lstlisting}
%       \usepackage{xparse}
%     \end{lstlisting}
%   \end{Packages}
%   \begin{tblock}{Kommandostruktur}
%     \begin{lstlisting}
%       \NewDocumentCommand\Kommandoname{m o O{...} ...}{Code mit #1, #2,...}
%       \NewDocumentEnvironment{Umgebungsname}{m o O{...} ...}
%       {
%         \begin-Code
%       }
%       {
%         \end-Code
%       }
%     \end{lstlisting}
%   \end{tblock}
%   Platzhalter legen Anzahl, Typ und Reihenfolge der Argumente fest
%   \begin{description}
%     \item[\texttt{m}] Pflichtargument
%     \item[\texttt{o}] optionales Argument, kein Defaultwert
%     \item[\texttt{O\{foo\}}] optionales Argument mit Default \texttt{foo}
%   \end{description}
% \end{frame}

% \begin{frame}[fragile]{Nützliche eigene Umgebungen}
%   \begin{tblock}{Code}
%     \begin{lstlisting}
%       \NewDocumentEnvironment{eqns}{O{rCl}}
%       {
%         \begin{IEEEeqnarray}{#1}
%       }
%       {
%         \end{IEEEeqnarray}
%         \ignorespacesafterend % verhindert Einrückung nach end-Block
%       }
%       \NewDocumentEnvironment{eqn}{}
%       {
%         \begin{eqns}[c]
%       }
%       {
%         \end{eqns}
%         \ignorespacesafterend
%       }
%       \end{lstlisting}
%   \end{tblock}
% \end{frame}

% \begin{frame}[fragile]{Mehr Beispiele}
%   \begin{tblock}{Code}
%     \begin{lstlisting}
%       \RenewDocumentCommand\v{m}{\mathbf{#1}}
%       \NewDocumentCommand\setN{}{\mathbb{N}}
%       \NewDocumentCommand\grad{}{\operatorname{grad}} % rm-font und passende Abstände im Mathemodus
%       \AtBeginDocument % wegen unicode-math
%       {
%         \RenewDocumentCommand\div{}{\operatorname{div}}
%       }
%       \NewDocumentCommmand\til{m}{\ensuremath{\tilde{#1}}} % erzwingt Mathemodus; sparend einsetzen
%     \end{lstlisting}
%   \end{tblock}
%   \begin{CodeExample}{0.70}[Test]
%     \begin{lstlisting}
%       $ \setN \quad \grad \v{F} \quad \div \v{A} $ \\
%       \til{x}
%     \end{lstlisting}
%   \CodeResult
%     $ \setN \quad \grad \v{F} \quad \div \v{A} $ \\
%     \til{x}
%   \end{CodeExample}
% \end{frame}

\subsection{Makros}

\begin{frame}[fragile]{Eigene \LaTeX-Kommandos}
  Nach 20 Mal \lstinline+\symup{e}+ oder \lstinline+\symup{i}+ schreiben hat man keine Lust mehr.
  \vspace{1em}
  \begin{block}{Kommandostruktur}
    \begin{lstlisting}
      \newcommand*\Kommandoname[#Argumente]{Code mit #1, #2, ...}
      \newcommand*\Command[#Argumente][Default]{Code}
      \newenvironment*{Umgebungsname}[#Argumente]{\begin-Code}{\end-Code}
    \end{lstlisting}
  \end{block}
  \begin{itemize}
    \item \verb+##+ gibt ein wirkliches \verb+#+
    \item Ohne \texttt{*}, um mehrere Absätze als Argument übergeben zu können
    \item \lstinline+\end+-Code kann Argumente nicht nutzen
    \item Man sollte neue Befehle nur in der Präambel definieren
  \end{itemize}
\end{frame}

\begin{frame}[fragile]{Beispiele}
  \begin{CodeExample}{0.79}[Code][Verwendung]
    \begin{lstlisting}
      \newcommand*\I{\symup{i}}
      \newcommand*\t[1]{\text{#1}}
      \newcommand*\dd[2]{\frac{\symup{d}#1}{\symup{d}#2}}

      % Überschreiben bereits belegter Kommandos
      \renewcommand*\v[1]{\symup{#1}}

      \newenvironment*{eqn}{\begin{equation}}{\end{equation}\ignorespacesafterend}
    \end{lstlisting}
  \CodeResult
    \begin{lstlisting}
      \I
      \t{foo}
      \dd{x}{y}
      \v{u}
      \begin{eqn}
        % .
      \end{eqn}
    \end{lstlisting}
  \end{CodeExample}
\end{frame}

% \begin{frame}[fragile]{\texttt{xparse}}
%   \begin{Packages}
%     \begin{lstlisting}
%       \usepackage{xparse}
%     \end{lstlisting}
%   \end{Packages}
%   \begin{block}{Kommandostruktur}
%     \begin{lstlisting}
%       \NewDocumentCommand\Kommandoname{m o O{...} ...}{Code mit #1, #2,...}
%       \NewDocumentEnvironment{Umgebungsname}{m o O{...} ...}
%       {
%         \begin-Code
%       }
%       {
%         \end-Code
%       }
%     \end{lstlisting}
%   \end{block}
%   Platzhalter legen Anzahl, Typ und Reihenfolge der Argumente fest
%   \begin{description}
%     \item[\texttt{m}] Pflichtargument
%     \item[\texttt{o}] optionales Argument, kein Defaultwert
%     \item[\texttt{O\{foo\}}] optionales Argument mit Default \texttt{foo}
%   \end{description}
% \end{frame}

% \begin{frame}[fragile]{Nützliche eigene Umgebungen}
%   \begin{block}{Code}
%     \begin{lstlisting}
%       \NewDocumentEnvironment{eqns}{O{rCl}}
%       {
%         \begin{IEEEeqnarray}{#1}
%       }
%       {
%         \end{IEEEeqnarray}
%         \ignorespacesafterend % verhindert Einrückung nach end-Block
%       }
%       \NewDocumentEnvironment{eqn}{}
%       {
%         \begin{eqns}[c]
%       }
%       {
%         \end{eqns}
%         \ignorespacesafterend
%       }
%       \end{lstlisting}
%   \end{block}
% \end{frame}

% \begin{frame}[fragile]{Mehr Beispiele}
%   \begin{block}{Code}
%     \begin{lstlisting}
%       \RenewDocumentCommand\v{m}{\symbf{#1}}
%       \NewDocumentCommand\setN{}{\symbb{N}}
%       \NewDocumentCommand\grad{}{\operatorname{grad}} % rm-font und passende Abstände im Mathemodus
%       \AtBeginDocument % wegen unicode-math
%       {
%         \RenewDocumentCommand\div{}{\operatorname{div}}
%       }
%       \NewDocumentCommmand\til{m}{\ensuremath{\tilde{#1}}} % erzwingt Mathemodus; sparend einsetzen
%     \end{lstlisting}
%   \end{block}
%   \begin{CodeExample}{0.70}[Test]
%     \begin{lstlisting}
%       $ \setN \quad \grad \v{F} \quad \div \v{A} $ \\
%       \til{x}
%     \end{lstlisting}
%   \CodeResult
%     $ \setN \quad \grad \v{F} \quad \div \v{A} $ \\
%     \til{x}
%   \end{CodeExample}
% \end{frame}

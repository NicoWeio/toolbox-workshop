\subsection{Mathe-Umgebungen}

\begin{frame}{Mathe-Umgebungen}
  \begin{itemize}
    \item \texttt{amsmath} stellt Mathe-Umgebungen für alles was man so braucht zur Verfügung.
      \item Alle Gleichungen werden automatisch nummeriert.
      \item \texttt{*} nach dem Umgebungsnamen sorgt für unnumerierte Gleichung
      \item Unnumerierte Gleichungen sollten selten sein.
  \end{itemize}
\end{frame}

\begin{frame}[fragile]{Die \texttt{equation}-Umgebung}
  Abgesetzte Gleichung, automatische Nummerierung. \\
  \texttt{equation*} erzeugt unnummerierte Gleichung.
  \begin{CodeExample}{0.60}
    \begin{lstlisting}
      Es gilt
      \begin{equation}
        \nabla \cdot \vec{E}
          = \frac{\rho}{\varepsilon_0} .
          \label{eqn:maxwell1}
      \end{equation}
      Schon Gauß hatte das Durchflutungsgesetz \eqref{eqn:maxwell1} aufgestellt.
    \end{lstlisting}
  \CodeResult
    \defaultdisplayskip
    Es gilt
    \begin{equation}
      \nabla \cdot \vec{E}
        = \frac{\rho}{\varepsilon_0} .
        \label{eqn:maxwell1}
    \end{equation}
    Schon Gauß hatte das Durchflutungsgesetz \eqref{eqn:maxwell1} aufgestellt.
  \end{CodeExample}

  \vspace{1em}
  \begin{itemize}
    \item Satzzeichen gehören in die \texttt{equation}-Umgebung!
    \item Gleichung ist grammatikalisch ein Substantiv
    \item Gleichungen sollten immer Teil eines vollständigen Satzes sein
  \end{itemize}
\end{frame}

\begin{frame}[fragile]{Die \texttt{gather}-Umgebung}
  \begin{itemize}
    \item Für mehrere Gleichungen
    \item \verb+\\+ erzeugt neue Zeile
      \begin{itemize}
        \item Kein \verb+\\+ nach letzter Zeile!
      \end{itemize}
    \item Jede Zeile bekommt eine Gleichungsnummer
  \end{itemize}
  \begin{CodeExample}{0.53}
    \begin{lstlisting}
      \begin{gather}
        (a + b)^2 = a^2 + 2ab + b^2 \\
        (a - b)^2 = a^2 - 2ab + b^2 \\
        (a + b) \cdot (a - b) = a^2 - b^2
      \end{gather}
    \end{lstlisting}
  \CodeResult
    \begin{gather}
      (a + b)^2 = a^2 + 2ab + b^2 \\
      (a - b)^2 = a^2 - 2ab + b^2 \\
      (a + b) \cdot (a - b) = a^2 - b^2
    \end{gather}
  \end{CodeExample}

  \vspace{1em}
  \begin{itemize}
    \item Abhängig vom Fall ist die \texttt{gather}-Umgebung grammatikalisch ein Substantiv oder eine Aufzählung
  \end{itemize}
\end{frame}

\begin{frame}[fragile]{Die \texttt{align}-Umgebung}
  \begin{itemize}
    \item Für mehrere Gleichungen, die aneinander ausgerichtet werden
    \item \lstinline+&+ steuert Ausrichtung
    \item \verb+\\+ erzeugt neue Zeile
    \item Jede Zeile bekommt eine Gleichungsnummer
  \end{itemize}
  \begin{CodeExample}{0.6}
    \begin{lstlisting}
      \begin{align}
        a         &= 1 & b           &= 2 \\
        a \cdot b &= 5 & \frac{a}{b} &= 0.5
      \end{align}
    \end{lstlisting}
  \CodeResult
    \begin{align}
      a         &= 1 & b           &= 2 \\
      a \cdot b &= 2 & \frac{a}{b} &= 0.5
    \end{align}
  \end{CodeExample}
\end{frame}

\begin{frame}[fragile]{Die \texttt{split}-Umgebung}
  \begin{itemize}
    \item Um überlange Gleichungen auf zwei Zeilen aufzuteilen.
    \item Kommt in den anderen Umgebungen zum Einsatz
    \item \lstinline+&+ steuert Ausrichtung
    \item \verb+\\+ erzeugt neue Zeile
    \item Gemeinsame Gleichungsnummer
  \end{itemize}
  \begin{CodeExample}{0.55}
    \begin{lstlisting}
      \begin{equation}
        \begin{split}
          (a+b)^3 = {} & a^3 + 3a^2b \\
                       & + 3ab^2 + b^3
        \end{split}
      \end{equation}
    \end{lstlisting}
  \CodeResult
    \vspace{0.5em}
    \begin{equation}
      \begin{split}
        (a+b)^3 = {} & a^3 + 3a^2b \\
                     & + 3ab^2 + b^3
      \end{split}
    \end{equation}
  \end{CodeExample}
\end{frame}

\section{Grundlagen}

\begin{frame}[fragile]{Das Dokument}
  Diese drei Zeilen braucht jedes \LaTeX-Dokument:
  \begin{columns}[onlytextwidth, t]
    \begin{column}{0.50\textwidth}
      \begin{block}{Code}
        \begin{lstlisting}
          \documentclass[optionen]{klasse}
            % .
            % Präambel
            % .
            % .
          \begin{document}
            % Inhalt des Dokuments
          \end{document}
        \end{lstlisting}
      \end{block}
    \end{column}
    \begin{column}{0.48\textwidth}
      \begin{block}{\lstinline+\\documentclass+}
        Vorlage wählen, mit Optionen anpassen.
      \end{block}
      \begin{block}{Präambel}
        Globale Optionen und zusätzliche Pakete.
      \end{block}
      \begin{block}{\texttt{document}-Umgebung}
        Inhalt des Dokuments.
      \end{block}
    \end{column}
  \end{columns}
\end{frame}
\begin{frame}[fragile]{Hallo Welt}
  \begin{CodeExample}{0.5}
    \begin{lstlisting}
      \documentclass{scrartcl}
      \begin{document}
        Hallo Welt!
      \end{document}
    \end{lstlisting}
  \CodeResult
    \begin{minipage}[c][4\baselineskip][c]{\textwidth}
      \strut
      Hallo Welt!
    \end{minipage}
  \end{CodeExample}
\end{frame}

\begin{frame}[fragile]{Syntax: Befehle}
  \LaTeX-Befehle beginnen stets mit einem \verb+\+ (Backslash).

  Obligatorische Argumente stehen in \lstinline+{ }+, optionale Argumente stehen in \lstinline+[ ]+.
    \begin{block}{Syntax}
      \begin{lstlisting}
        \befehl[optional]{obligatorisch}
        \befehl*[optional]{obligatorisch}
      \end{lstlisting}
    \end{block}

  \verb+*+ ruft häufig eine Alternativform des Befehls auf.
  \begin{CodeExplanation}{0.53}[Code][Erklärung]
    \begin{lstlisting}
      \documentclass[paper=a4]{scrartcl}

      \tableofcontents
      \frac{1}{2}
      % Kommentar
    \end{lstlisting}
  \Explanation
    \strut
    Dokumentenklasse \texttt{scrartcl},\\
    Papierformat DIN\,A4 \\
    Keine Argumente \\
    Zwei oder mehr Pflichtargumente\\
    \verb+%+-Zeichen für Kommentare
  \end{CodeExplanation}
\end{frame}

\begin{frame}[fragile]{Syntax: Umgebungen}
  \begin{itemize}
    \item Einstellungen für Bereich des Dokuments
    \item Extrem vielseitig
    \item Können ggfs.\ auch Optionen übergeben bekommen
    \item Oft auch Alternativform mit \lstinline+*+
  \end{itemize}
  \begin{CodeExplanation}{0.65}[Syntax][Beispiel]
    \begin{lstlisting}
      \begin{Umgebung}[optional]{obligatorisch}
        % .
      \end{Umgebung}
    \end{lstlisting}
    \Explanation
    \begin{lstlisting}
      \begin{flushright}
        % .
      \end{flushright}
    \end{lstlisting}
  \end{CodeExplanation}
\end{frame}

\begin{frame}[fragile]{Syntax: Umgebungen}
  \begin{itemize}
    \item Können weitere Umgebungen enthalten
    \item Diese müssen aber in der Umgebung wieder geschlossen werden
  \end{itemize}
  \begin{columns}[onlytextwidth, t]
    \begin{column}{0.48\textwidth}
      \begin{block}{Geht:}
        \begin{lstlisting}
          \begin{document}
            \begin{flushright}
              % .
            \end{flushright}
          \end{document}
        \end{lstlisting}
      \end{block}
    \end{column}
    \begin{column}{0.48\textwidth}
      \begin{alertblock}{Geht nicht:}
        \begin{lstlisting}
          \begin{itemize}
            \begin{enumerate}
              % .
          \end{itemize}
            \end{enumerate}
        \end{lstlisting}
      \end{alertblock}
    \end{column}
  \end{columns}
\end{frame}

\begin{frame}[fragile]{Standardpakete}
  Die hier aufgezählten Pakete sollten immer geladen werden, da sie wesentliche Funktionen bieten und wichtige Einstellungen vornehmen.
  \vspace{-1em}
  \begin{CodeExplanation}{0.52}[Paket][Funktion]
    \begin{lstlisting}
      \usepackage[aux]{rerunfilecheck}


      \usepackage{polyglossia}
      \setmainlanguage{german}
      \usepackage{fontspec}

      % mehr Pakete hier

      \usepackage[unicode]{hyperref}

      \usepackage{bookmark}
    \end{lstlisting}
  \Explanation
    \strut
    Warnung, falls nochmal kompiliert werden muss. \\[\baselineskip]
  Deutsche Spracheinstellungen. \\[\baselineskip]
    Für Fonteinstellungen \\[3\baselineskip]
    Für Hyperlinks (z.B. Inhaltsverzeichnis → Kapitel). \\
    Erweiterte Bookmarks im PDF.
  \end{CodeExplanation}
  Die Reihenfolge ist manchmal wichtig, z.B. damit Pakete die Spracheinstellung kennen.
\end{frame}

\begin{frame}[fragile]{
  KOMA-Script-Klassen
  \hfill\doc{http://mirrors.ctan.org/macros/latex/contrib/koma-script/doc/scrguide.pdf}{\textsf{KOMA-Skript}}
}
  \begin{itemize}
    \item \texttt{scrartcl}, \texttt{scrreprt} und \texttt{scrbook}
    \item Sehr gute Vorlagen
    \item Schnell global mit Klassenoptionen anpassbar
  \end{itemize}
  \begin{block}{Fürs Praktikum empfohlenene Klasse}
    \begin{lstlisting}
      \documentclass[…]{scrartcl}
    \end{lstlisting}
  \end{block}
\end{frame}

\begin{frame}[fragile]{
  Fonteinstellungen
  \hfill\doc{http://mirrors.ctan.org/macros/latex/contrib/fontspec/fontspec.pdf}{fontspec}
}
  Standardeinstellung sind die Latin-Modern-Fonts.
  \vspace{1em}
  \begin{CodeExplanation}{0.48}[Latin Modern][Alternativ: Tex Gyre]
    \begin{lstlisting}
      \usepackage{fontspec}
    \end{lstlisting}
  \Explanation
    \begin{lstlisting}
      \usepackage{fontspec}
      \setmainfont{Tex Gyre Termes}
      \setsansfont{Tex Gyre Heros}
      \setmonofont{Tex Gyre Cursor}
    \end{lstlisting}
  \end{CodeExplanation}
  \begin{itemize}
    \item Jede System-Schriftart kann genutzt
    \item \alert{Das ist i.A. nicht sinnvoll: \fontspec{Comic Sans MS} Hallo Welt in Comic Sans!}
    \item Schriften müssen zueinander passen
    \item Schriften müssen alle benötigten Sonderzeichen enthalten
    \item Bei Änderung auch Mathefont anpassen → später
  \end{itemize}
\end{frame}

\begin{frame}[fragile]{Gerüst}
  \begin{lstlisting}
    \documentclass{scrartcl}

    \usepackage[aux]{rerunfilecheck}
    \usepackage{polyglossia}
    \setmainlanguage{german}

    \usepackage{fontspec}
    % mehr Pakete hier

    \usepackage[unicode]{hyperref}
    \usepackage{bookmark}
    % Einstellungen hier, z.B. Fonts

    \begin{document}
      % Text hier
    \end{document}
  \end{lstlisting}
\end{frame}

\begin{frame}[fragile]{Das Ausgabedokument erstellen}
  Es gibt verschiedene \LaTeX-Kompiler, die verschiedene Ausgabeformate erzeugen können.
  Der modernste Kompiler, der PDF-Dateien erstellt, ist \alert{\texttt{lualatex}}.

  \begin{block}{\LaTeX-Dokument kompilieren}
    Terminal öffnen:
    \begin{lstlisting}
      lualatex MeinDokument.tex
    \end{lstlisting}
  \end{block}

  \begin{alertblock}{Vorsicht!}
    \begin{itemize}
      \item Es muss fast immer mindestens zweimal kompiliert werden.
      \item Es werden diverse Hilfs- und Logdateien erzeugt.
      \item Die Input-Dokumente müssen UTF-8 codiert sein.
    \end{itemize}
  \end{alertblock}
\end{frame}

\begin{frame}{\texttt{texdoc}}
  \LaTeX\ und (fast) alle Pakete sind hervorragend dokumentiert. Die Dokumentation wird automatisch mitinstalliert.
  \begin{block}{Dokumentation zu einem Paket}
    \texttt{texdoc \textit{paket}}
  \end{block}

  Dabei ist \texttt{\textit{paket}} ein Suchstring.
  \begin{block}{Nach Dokumentation suchen}
    \texttt{texdoc -l \textit{name}}
  \end{block}

  Es ist wichtig zu lernen, Dokumentationen zu lesen. Probiert es an den oben genannten Paketen aus.

  \vspace{10pt}
  Alternativ kann man das Paket bei Google suchen, dann findet man auch die Dokumentation auf CTAN.
\end{frame}

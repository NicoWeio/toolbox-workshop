\section{Grundlagen}

\begin{frame}[fragile]{Das Dokument}
  Diese drei Zeilen braucht jedes \LaTeX-Dokument:
  \begin{columns}[T]
    \begin{column}{0.47\textwidth}
      \begin{block}{\LaTeX-Code}
        \begin{lstverbatim}
        \documentclass[optionen]{klasse}
          % Präambel
          % .
          % .
        \begin{document}
          % Inhalt des Dokuments
          % .
          % .
        \end{document}
        \end{lstverbatim}
      \end{block}
    \end{column}
    \begin{column}{0.47\textwidth}
      \begin{block}{\texttt{\textbackslash documentclass}}
        Hier wird die Dokumentenklasse gewählt und mit Optionen angepasst.
      \end{block}
      \begin{block}{Präambel}
        Hier werden globale Optionen gesetzt und zusätzliche Pakete eingebunden.
      \end{block}
      \begin{block}{\texttt{document}-Umgebung}
        Hier wird das eigentliche Dokument erstellt.
      \end{block}
    \end{column}
  \end{columns}
\end{frame}

\begin{frame}[fragile]{Syntax: Befehle}
  \LaTeX-Befehle beginnen stets mit einem \textbackslash\ (Backslash).

  Obligatorische Argumente stehen in \{ \}, optionale Argumente stehen in [ ].

  * ruft häufg eine Alternativform des Befehls auf.
  \begin{columns}[T]
    \begin{column}{0.53\textwidth}
      \begin{block}{\LaTeX-Code}
        \begin{lstverbatim}
        \befehl*[optional]{obligatorisch}
        \documentclass[paper=a4]{scrartcl}


        \frac{1}{2}


        % Kommentar
        \end{lstverbatim}
      \end{block}
    \end{column}
    \begin{column}{0.41\textwidth}
      \begin{block}{Erklärung}
        Beispiel \\
        Setzt die Dokumentenklasse auf \texttt{scrartcl} und das Papierformat auf DIN\,A4. \\
        Es gibt auch Befehle mit zwei oder mehr Pflichtargumenten, z.B. der Bruch. \\
        Text oder Befehle nach einem \%-Zeichen werden nicht berücksichtigt.
      \end{block}
    \end{column}
  \end{columns}
\end{frame}

\begin{frame}[fragile]{Syntax: Umgebungen}
  Das zweite wichtige \LaTeX-Element sind die Umgebungen.
  Umgebungen können extrem vielseitig sein, von einer Änderung der Textausrichtung über mathematische Formel bis hin zu Designelementen.
  \begin{columns}[T]
    \begin{column}{0.47\textwidth}
      \begin{block}{Syntax}
        \begin{lstverbatim}
        \begin{Umgebung*}[optional]
            {obligatorisch}
          % .
          % .
        \end{Umgebung*}
        \end{lstverbatim}
      \end{block}
    \end{column}
    \begin{column}{0.47\textwidth}
      \begin{block}{Eigenschaften}
        \begin{itemize}
          \item Umgebungen können verschachtelt werden
          \item Umgebungen können \alert{nicht} verschränkt werden
        \end{itemize}
      \end{block}
    \end{column}
  \end{columns}
\end{frame}

\begin{frame}[fragile]{Standardpakete}
  Die hier aufgezählten Pakete sollten immer geladen werden, da sie wesentliche Funktionen bieten und wichtige Einstellungen vornehmen.
  \begin{columns}[T]
    \begin{column}{0.47\textwidth}
      \begin{block}{Paket}
        \begin{lstverbatim}
        \usepackage{fixltx2e}
        \usepackage[aux]{rerunfilecheck}

        \usepackage[main=ngerman]{babel}

        \usepackage{fontspec}
        % mehr Pakete hier

        \usepackage[unicode,pdfusetitle]{hyperref}
        \usepackage{bookmark}
        \end{lstverbatim}
      \end{block}
    \end{column}
    \begin{column}{0.47\textwidth}
      \begin{block}{Funktion}
        \LaTeXe\ korrigieren. \\
        Warnung, falls nochmal kompiliert werden muss. \\
        Deutsche Spracheinstellungen für das Dokument. \\
        Für Fonteinstellungen \\[2\baselineskip]
        Für Hyperlinks im Dokument (z.B. Inhaltsverzeichnis $\rightarrow$ Kapitel). \\
        Erweiterte Bookmarks im PDF.
      \end{block}
    \end{column}
  \end{columns}

  \vspace{5pt}
  Die Reihenfolge ist manchmal wichtig!
\end{frame}

\begin{frame}[fragile]{
  KOMA-Script-Klassen
  \hfill\doc{http://mirrors.ctan.org/macros/latex/contrib/koma-script/doc/scrguide.pdf}{KOMA-Skript}
}
  Stellt die wichtigen Klassen \texttt{scrartcl}, \texttt{scrreprt} und \texttt{scrbook} zur Verfügung.
  Sehr gute Vorlagen schon mit den Standardeinstellungen, schnell global mit Klassenoptionen anpassbar.
  \begin{block}{fürs Praktikum empfohlenene Einstellungen}
    \begin{lstverbatim}
    \documentclass[
      captions=tableheading,  % Spacing für Tabellenüberschriften
      titlepage=firstiscover, % Titleseite ist Deckblatt
    ]{scrartcl}
    \end{lstverbatim}
  \end{block}
\end{frame}

\begin{frame}{Fonteinstellungen – minimal}
  Ein Minimalset von Standardfonts ist ohne Einstellungen verfügbar.
\end{frame}

\begin{frame}[fragile]{Fonteinstellungen – Latin Modern komplett}
  \begin{lstverbatim}
  \setmainfont[
    SmallCapsFont = {Latin Modern Roman Caps},
    SlantedFont = {Latin Modern Roman Slanted},
    ItalicFeatures  = {
      SmallCapsFont = {LMRomanCaps10-Oblique}
    },
    ]{Latin Modern Roman}

  \setsansfont{Latin Modern Sans}

  \setmonofont[
    SmallCapsFont = {Latin Modern Mono Caps},
    SlantedFont = {Latin Modern Mono Slanted},
    ItalicFeatures  = {
      SmallCapsFont = {LMMonoCaps10-Oblique}
    },
    ]{Latin Modern Mono}
  \end{lstverbatim}
\end{frame}

\begin{frame}[fragile]{Gerüst}
  \begin{lstverbatim}
  \documentclass[
    captions=tableheading,
    titlepage=firstiscover,
  ]{scrartcl}

  \usepackage{fixltx2e}
  \usepackage[aux]{rerunfilecheck}
  \usepackage[main=ngerman]{babel}
  \usepackage{fontspec}
  % mehr Pakete hier

  \usepackage[unicode,pdfusetitle]{hyperref}
  \usepackage{bookmark}
  % Einstellungen hier, z.B. Fonts

  \begin{document}
    % Text hier
  \end{document}
  \end{lstverbatim}
\end{frame}

\begin{frame}[fragile]{Das Ausgabedokument erstellen}
  Es gibt mehrere verschiedene \LaTeX-Kompiler, die verschiedene Ausgabeformate erzeugen können.
  Der modernste Kompiler, der PDF-Dateien erstellt, ist \alert{lualatex}.

  \begin{block}{\LaTeX-Dokument kompilieren}
    Konsole öffnen:
    \begin{lstverbatim}
    lualatex MeinDokument.tex
    \end{lstverbatim}
  \end{block}
  \begin{alertblock}{Vorsicht!}
    \begin{itemize}
      \item Es muss fast immer mindestens zweimal kompiliert werden.
      \item Es werden diverse Hilfs- und Logdateien erzeugt.
      \item Die Input-Dokumente müssen UTF-8 codiert sein.
    \end{itemize}
  \end{alertblock}
\end{frame}

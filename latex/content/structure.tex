\section{Struktur}

\begin{frame}[fragile]{Titelseite und Metadaten}
  \LaTeX\ erstellt automatisch eine Titelei aus den Metadaten. \\
  Mit der Klassenoption \lstinline{titlepage=firstiscover} wird diese  als eigene Seite gesetzt.
  
  \begin{block}{Neue Klassenoption}
    \begin{lstlisting}
      \documentclass[…, titlepage=firstiscover, …]{scrartcl}
    \end{lstlisting}
  \end{block}

  \begin{block}{Empfehlung fürs Praktikum:}
    \begin{lstlisting}
      \title{101 Titel des Versuchs}
      % Mehrere Autoren mit \and:
      \author{Max Mustermann \and Maria Musterfrau}
      \date{Durchführung: 26.09.2014, Abgabe: 29.09.2014}
    \end{lstlisting}
  \end{block}

  \begin{block}{Titelseite generieren}
    \begin{lstlisting}
      \maketitle
    \end{lstlisting}
  \end{block}
\end{frame}

\begin{frame}[fragile]{Gliederung}
  \LaTeX\ bietet Befehle zum erstellen von Gliederungsebenen.
  Diese werden automatisch nummeriert und in entsprechend größerer und fetter Schrift gesetzt.

  \begin{block}{Gliederungsebenen für \texttt{scrartcl}}
    \begin{lstlisting}
      \section{Überschrift}
      \subsection{Überschrift}
      \subsubsection{Überschrift}
      \paragraph{Überschrift}    % wird nicht nummeriert
      \subparagraph{Überschrift} % wird nicht nummeriert
    \end{lstlisting}
  \end{block}
  \begin{block}{Höhere Gliederungsebenen für \texttt{scrreprt} und \texttt{scrbook}}
    \begin{lstlisting}
      \part{Überschrift}
      \chapter{Überschrift}
      \section{Überschrift}
    \end{lstlisting}
  \end{block}
\end{frame}

\begin{frame}[fragile]{Inhaltsverzeichnis}
  \begin{block}{Inhaltsverzeichnis generieren}
    \begin{lstlisting}
      \tableofcontents
      \newpage
    \end{lstlisting}
  \end{block}
\end{frame}

\begin{frame}[fragile]{Referenzen}
  \begin{block}{Code}
    \begin{lstlisting}
      \section{Messung mit Apparatur 2}
      \label{sec:apparatur2}
      % .
      \section{Auswertung}
      Wie in \ref{sec:apparatur2} beschrieben, ...
    \end{lstlisting}
  \end{block}
  \begin{itemize}
    \item Auch für Gleichungen, Grafiken, Tabellen → später
    \item Für Übersichtlichkeit sollten Labels den Typ der Referenz nennen:
      \begin{description}
        \item[Sections]    \texttt{sec:}
        \item[Gleichungen] \texttt{eqn:}
        \item[Abbildungen] \texttt{fig:}
        \item[Tabellen]    \texttt{tab:}
      \end{description}
    \item Bei Gleichungen: \lstinline+\eqref+ statt \lstinline+\ref+ → setzt Klammern: \eqref{eqn:maxwell1}
    \item \lstinline+\label+ immer nach dem, worauf verwiesen wird
  \end{itemize}
\end{frame}

\section{Text erstellen}

\begin{frame}[fragile]{Text schreiben}
  \begin{block}{Beispiel}
    \begin{lstlisting}
      % Präambel
      \begin{document}
        Hallo, Welt!

        Dies ist ein dummer Beispieltext.
        Er soll zeigen, dass \LaTeX sich nicht um
        Zeilenumbrüche im Code    oder    zuviele
        Leerzeichen kümmert.

        Ein Absatz wird mit einer leeren Code-Zeile
        markiert.
      \end{document}
    \end{lstlisting}
  \end{block}
\end{frame}

\begin{frame}[fragile]{Konventionen für Text}
  \begin{itemize}
    \item Höchstens ein Satz pro Code-Zeile
    \item Absätze werden durch eine Leerzeile markiert
    \item Im Fließtext sollten keine Umbrüche mit \lstinline+\\+ erzwungen werden
  \end{itemize}
  \begin{alertblock}{Sonderzeichen}
    Viele Sonderzeichen sind \LaTeX-Steuerzeichen.
    Damit diese im Text genutzt werden können, muss meist ein \verb+\+ vorangestellt oder ein Befehl genutzt werden.
  \end{alertblock}
  \begin{CodeExample}{0.7}
    \begin{lstlisting}
      \# \$ \% \& \_ \{ \}
      \textbackslash \textasciicircum \textasciitilde
    \end{lstlisting}
  \CodeResult
    \strut
    \# \$ \% \& \_ \{ \} \\
    \textbackslash\ \textasciicircum\ \textasciitilde
  \end{CodeExample}
\end{frame}

\begin{frame}[fragile]{Textauszeichnung}
  Änderungen der Schrifteigenschaften sind mit diesen Befehlen möglich:
  \begin{CodeExample}{0.60}
    \begin{lstlisting}
      \textit{kursiv} \emph{kursiv}
      \textbf{fett}
      \textbf{\textit{fett-kursiv}}
      \textrm{Serifen-Schrift}
      \texttt{Mono-Schrift}
      \textsf{Sans-Serif-Schrift}
      \textsc{Kapitälchen}
    \end{lstlisting}
  \CodeResult
    \strut
    \textit{kursiv} \emph{kursiv} \\
    \textbf{fett} \\
    \textbf{\textit{fett-kursiv}} \\
    \textrm{Serifen-Schrift} \\
    \texttt{Mono-Schrift} \\
    \textsf{Sans-Serif-Schrift} \\
    \textsc{Kapitälchen}
  \end{CodeExample}

  \vspace{1em}
  Diese Befehle sollten sehr selten benutzt werden, semantischer Markup ist besser.
\end{frame}

\begin{frame}[fragile]{Schriftgrößen}
  Gelten immer für den aktuellen Block, z.\,B. in einer Umgebung oder zwischen \lstinline+{ }+
  \begin{CodeExample}{0.48}
    \begin{lstlisting}
      {\tiny tiny}
      {\small small}
      {\normalsize normal}
      {\large large}
      {\huge huge}
    \end{lstlisting}
  \CodeResult
    \begin{minipage}[c][5\baselineskip][c]{\textwidth}
      {\tiny tiny}
      {\small small}
      {\normalsize normal}
      {\large large}
      {\huge huge}
    \end{minipage}
  \end{CodeExample}
  \vspace{1em}
  \begin{block}{Alle Größen}
    \begin{lstlisting}
      \tiny, \scriptsize, \footnotesize, \small, \normalsize, \large, \Large, \LARGE, \huge, \Huge
    \end{lstlisting}
  \end{block}
  Auch diese Befehle sollten nur über semantischen Markup benutzt werden.
\end{frame}

\begin{frame}[fragile]{Inhalt auslagern}
  \begin{block}{Code}
    \begin{lstlisting}
      \documentclass[
  bibliography=totoc,     % Literatur im Inhaltsverzeichnis
  captions=tableheading,  % Tabellenüberschriften
  titlepage=firstiscover, % Titelseite ist Deckblatt
]{scrartcl}

% LaTeX2e korrigieren.
\usepackage{fixltx2e}
% Warnung, falls nochmal kompiliert werden muss
\usepackage[aux]{rerunfilecheck}

% deutsche Spracheinstellungen
\usepackage{polyglossia}
\setmainlanguage{german}

% unverzichtbare Mathe-Befehle
\usepackage{amsmath}
% viele Mathe-Symbole
\usepackage{amssymb}
% Erweiterungen für amsmath
\usepackage{mathtools}

% Fonteinstellungen
\usepackage{fontspec}
\defaultfontfeatures{Ligatures=TeX}

\usepackage[
  math-style=ISO,    % \
  bold-style=ISO,    % |
  sans-style=italic, % | ISO-Standard folgen
  nabla=upright,     % |
  partial=upright,   % /
]{unicode-math}

\setmathfont{Latin Modern Math}
\setmathfont[range={\mathscr, \mathbfscr}]{XITS Math}
\setmathfont[range=\coloneq]{XITS Math}
\setmathfont[range=\propto]{XITS Math}
% make bar horizontal, use \hslash for slashed h
\let\hbar\relax
\DeclareMathSymbol{\hbar}{\mathord}{AMSb}{"7E}
\DeclareMathSymbol{ℏ}{\mathord}{AMSb}{"7E}

% richtige Anführungszeichen
\usepackage[autostyle]{csquotes}

% Zahlen und Einheiten
\usepackage[
  locale=DE,                   % deutsche Einstellungen
  separate-uncertainty=true,   % Immer Fehler mit \pm
  per-mode=symbol-or-fraction, % m/s im Text, sonst Brüche
]{siunitx}

% chemische Formeln
\usepackage[version=3]{mhchem}

% schöne Brüche im Text
\usepackage{xfrac}

% Floats innerhalb einer Section halten
\usepackage[section, below]{placeins}
% Captions schöner machen.
\usepackage[
  labelfont=bf,        % Tabelle x: Abbildung y: ist jetzt fett
  font=small,          % Schrift etwas kleiner als Dokument
  width=0.9\textwidth, % maximale Breite einer Caption schmaler
]{caption}
% subfigure, subtable, subref
\usepackage{subcaption}

% Grafiken können eingebunden werden
\usepackage{graphicx}
% größere Variation von Dateinamen möglich
\usepackage{grffile}

% Standardplatzierung für Floats einstellen
\usepackage{float}
\floatplacement{figure}{htbp}
\floatplacement{table}{htbp}

% schöne Tabellen
\usepackage{booktabs}

% Seite drehen für breite Tabellen
\usepackage{pdflscape}

% Literaturverzeichnis
\usepackage{biblatex}
% Quellendatenbank
\addbibresource{lit.bib}
\addbibresource{programme.bib}

% Hyperlinks im Dokument
\usepackage[
  unicode,
  pdfusetitle,    % Titel, Autoren und Datum als PDF-Attribute
  pdfcreator={},  % PDF-Attribute säubern
  pdfproducer={}, % "
]{hyperref}
% erweiterte Bookmarks im PDF
\usepackage{bookmark}

% Trennung von Wörtern mit Strichen
\usepackage[shortcuts]{extdash}

\author{
  AUTOR A%
  \texorpdfstring{
    \\
    \href{mailto:authorA@udo.edu}{authorA@udo.edu}
  }{}%
  \texorpdfstring{\and}{, }
  AUTOR B
  \texorpdfstring{
    \\
    \href{mailto:authorB@udo.edu}{authorB@udo.edu}
  }{}
}
\publishers{TU Dortmund – Fakultät Physik}

      \begin{document}
        \input{Teil1.tex}
        \input{Teil2.tex}
        % .
      \end{document}
    \end{lstlisting}
  \end{block}
  \begin{itemize}
    \item Verschachtelung möglich
    \item Zur Aufteilung größerer Dokumente (z.B. diese Präsentation)
    \item Für häufig wiederverwendeten Code (Header, Erläuterungen zu Fehlerrechnung, …)
    \item Für per Skript erzeugte Tabelleninhalte
  \end{itemize}
\end{frame}

\begin{frame}[fragile]{
  Anführungszeichen
  \hfill
  \doc{http://mirrors.ctan.org/macros/latex/contrib/csquotes/csquotes.pdf}{csquotes}
}
  Die richtigen Anführungszeichen, wo die Satzzeichen hingehören und vieles mehr hängt von der Sprache ab.
  So macht man es richtig:
  \begin{Packages}
    \begin{lstlisting}
      % babel mit anderen Sprachen laden
      \usepackage[main=ngerman, english, french]{babel}
      \usepackage[autostyle]{csquotes}    % babel
    \end{lstlisting}
  \end{Packages}
  \begin{CodeExample}{0.60}
    \begin{lstlisting}
      foo \enquote{bar} baz
      \enquote{foo \enquote{bar} baz}
      \foreignlanguage{english}{\enquote{foo}}
      \foreignlanguage{french}{\enquote{foo}}
      \textcquote{root}{foo}
    \end{lstlisting}
  \CodeResult
    \strut
    foo \enquote{bar} baz \\
    \enquote{foo \enquote{bar} baz} \\
    \foreignlanguage{english}{\enquote{foo}}\\
    \foreignlanguage{french}{\enquote{foo}}\\
    \textcquote{root}{foo}
  \end{CodeExample}
\end{frame}

\section{Text erstellen}

\begin{frame}[fragile]{Titelseite und Metadaten}
  \LaTeX\ erstellt automatisch eine Titelseite aus den Metadaten.

  \begin{block}{Empfehlung fürs Praktikum:}
    \begin{lstverbatim}
    \subject{Fakultät Physik, TU Dortmund}
    \title{Vxxx: Titel des Versuchs}
    \subtitle{Physikalisches Anfängerpraktikum}
    \author{Max Mustermann}
    % Mehrere Autoren mit \and :
    \author{Max Mustermann \and Maria Musterfrau}
    \date{Datum der Versuchsdurchführung}
    % Zusätzlich möglich:
    \titlehead{Kopf}
    \publishers{Verlag}
    \end{lstverbatim}
  \end{block}

  \begin{block}{Titelseite generieren}
    \begin{lstverbatim}
    \maketitle
    \end{lstverbatim}
  \end{block}
\end{frame}

\begin{frame}[fragile]{Inhaltsverzeichnis und Gliederung}
  \LaTeX\ bietet Befehle zum erstellen von Gliederungsebenen.
  Diese werden automatisch nummeriert und in entsprechend größerer und fetter Schrift gesetzt.

  Mit \verb+\tableofcontents+ wird das Inhaltsverzeichnis erstellt.

  \begin{block}{Gliederungsebenen für \texttt{scrartcl}}
    \begin{lstverbatim}
    \section{Überschrift}
    \subsection{Überschrift}
    \subsubsection{Überschrift}
    \paragraph{Überschrift} % wird nicht nummeriert
    \subparagraph{Überschrift} % wird nicht nummeriert
    \end{lstverbatim}
  \end{block}
  \begin{block}{höhere Gliederungsebenen für \texttt{scrrepr} und \texttt{scrbook}}
    \begin{lstverbatim}
    \part{Überschrift}
    \chapter{Überschrift}
    \section{Überschrift}
    \end{lstverbatim}
  \end{block}
\end{frame}

\begin{frame}[fragile]{Konventionen für Text}
  \begin{itemize}
    \item höchstens ein Satz pro Code-Zeile
    \item Absätze werden durch eine Leerzeile markiert
    \item Im Fließtext sollten keine Umbrüche mit \textbackslash\textbackslash\  erzwungen werden.
  \end{itemize}
  \begin{alertblock}{Sonderzeichen}
    Viele Sonderzeichen sind \LaTeX-Steuerzeichen.
    Damit diese im Text genutzt werden können, muss meist ein \textbackslash\ vorangestellt oder ein Befehl genutzt werden:
    \begin{center}
      \begin{lstlisting}
      \% \& \_ \textbackslash \$ \{ \}
      \end{lstlisting}
      \% \& \_ \textbackslash\ \$ \{ \}
    \end{center}
  \end{alertblock}
\end{frame}

\begin{frame}{Übung: Aufbau des Protokolls}
  \begin{block}{Aufgabe}
    Schreibt und kompiliert den groben Rahmen für das Protokoll.
    Es sollte enthalten:
    \begin{enumerate}
      \item Titelseite mit den wichtigen Informationen
      \item Inhaltsverzeichnis
      \item Section Theorie
        \begin{enumerate}
          \item subsection Theorie A
          \item subsection Theorie B
        \end{enumerate}
      \item Section Aufbau und Durchführung
        \begin{enumerate}
          \item subsection Aufbau
          \item subsection Versuchsdurchführung
        \end{enumerate}
      \item Section Auswertung
        \begin{enumerate}
          \item subsection Teil 1 der Auswertung
          \item subsection Teil 2 der Auswertung
        \end{enumerate}
      \item Section Diskussion
    \end{enumerate}
  \end{block}
\end{frame}

\begin{frame}[fragile]{Textauszeichnung}
  Änderungen der Schrifteigenschaften sind mit diesen Befehlen möglich:
  \begin{columns}[t]
    \begin{column}{0.6\textwidth}
      \begin{block}{\LaTeX-Code}
        \begin{lstverbatim}
        \textit{kursiv} \emph{kursiv}
        \textbf{fett}
        \textbf{\textit{fett-kursiv}}
        \textrm{Serifen-Schrift}
        \texttt{Mono-Schrift}
        \textsf{Sans-Serif-Schrift}
        \end{lstverbatim}
      \end{block}
    \end{column}
    \begin{column}{0.35\textwidth}
      \begin{block}{Ergebnis}
        \textit{kursiv} \emph{kursiv} \\
        \textbf{fett} \\
        \textbf{\textit{fett-kursiv}} \\
        \textrm{Serifen-Schrift} \\
        \texttt{Mono-Schrift} \\
        \textsf{Sans-Serif-Schrift}
      \end{block}
    \end{column}
  \end{columns}
  \vspace{1em}
  In der Matheumgebung statt \verb+\textxx+ einfach \verb+\mathxx+, z.B. \verb+\mathrm+.
\end{frame}

\begin{frame}[fragile]{Textgrößen}
  Gelten immer für den aktuellen Block, z.B. in einer Umgebung oder zwischen \{\dots\}
  \begin{columns}[t]
    \begin{column}{0.47\textwidth}
      \begin{block}{\LaTeX-Code}
        \begin{lstverbatim}
        {\tiny tiny}
        {\small small}
        {\normalsize normal}
        {\large large}
        {\huge huge}
        \end{lstverbatim}
      \end{block}
    \end{column}
    \begin{column}{0.47\textwidth}
      \begin{block}{Ergebnis}
        {\tiny tiny}
        {\small small}
        {\normalsize normal}
        {\large large}
        {\huge huge}
      \end{block}
    \end{column}
  \end{columns}
  \vspace{1em}
  \begin{block}{alle Größen}
    \begin{lstverbatim}
    \tiny, \scriptsize, \footnotesize, \small, \normalsize, \large, \Large, \LARGE, \huge, \Huge
    \end{lstverbatim}
  \end{block}
\end{frame}

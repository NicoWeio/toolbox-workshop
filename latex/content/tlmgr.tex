\section{\texttt{tlmgr}}

\begin{frame}[fragile]{
  \texttt{tlmgr} - \TeX Live (Paket) Manager
}
  \begin{itemize}
    \item \TeX-Live kommt mit einem eigenen Verwaltungstool
    \item Neue Pakete installieren, updaten, suchen, ...
  \end{itemize}

  \begin{block}{Ein neues Paket installieren}
    Wenn man die Dokumentation auch haben möchte, \texttt{--with-doc} nutzen:
    \begin{center}
      \texttt{tlmgr install [--with-doc] <name>}
    \end{center}
  \end{block}

  \begin{block}{Welches Paket muss ich installieren?}

    \begin{center}
      \texttt{tlmgr search --global --file booktabs.sty}
    \end{center}
    Findet heraus, welches Paket eine bestimmte Datei zur Verfügung stellt.

    Hilfreich bei Fehlermeldungen wie \texttt{booktabs.sty not found}.
  \end{block}

  \begin{block}{Updates installieren}
    \begin{center}
      \texttt{tlmgr update --self --all --forcibly-removed}
    \end{center}
  \end{block}

  Besonders wichtig, wenn man nicht das ganze TeX-Live installiert hat,
  um Platz zu sparen.
\end{frame}

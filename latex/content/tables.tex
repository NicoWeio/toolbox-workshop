\section{Tabellen}

\begin{frame}[fragile]{Tabellen}
  \begin{columns}[t]
    \begin{column}{0.33\textwidth}
      \begin{Packages}
        \begin{lstlisting}
          \usepackage{booktabs}
        \end{lstlisting}
      \end{Packages}
    \end{column}
    \begin{column}{0.63\textwidth}
    \begin{block}{Neue Klassenoption}
      \begin{lstlisting}
        \documentclass[…, captions=tableheading, …]{scrartcl}
      \end{lstlisting}
    \end{block}
    \end{column}
  \end{columns}
  \vspace{-2pt}
  \begin{columns}[onlytextwidth, t]
    \begin{column}{0.60\textwidth}
      \fontsize{8}{6}
      \begin{block}{Code}
        \begin{lstlisting}
          \begin{table}
            \centering
            \caption{Eine Tabelle mit Messdaten.}
            \label{tab:some_data}
            \begin{tabular}{c c c c c}
              \toprule
              $f$ & $l_\text{start}$ & $l_1$ & $l_{\text{kor},1}$ & $B_1$ \\
              \midrule
              100 & 1.14 & 3.51 & 0.00 &   4.30 \\
              300 & 1.27 & 2.42 & 0.13 &  41.14 \\
              500 & 1.21 & 1.70 & 0.25 & 168.73 \\
              \bottomrule
            \end{tabular}
          \end{table}
        \end{lstlisting}
      \end{block}
    \end{column}
    \begin{column}{0.36\textwidth}
      \begin{itemize}
        \item Äußere \texttt{table}-Umgebung behandelt Tabelle wie ein float
        \item Innere \texttt{tabular}-Umgebung für eigentlichen Tabelleninhalt
        \item \texttt{l},\texttt{c} oder \texttt{r} geben Ausrichtung der einzelnen Spalten an
        \item \lstinline+\caption+, \lstinline+\label+ oberhalb von \texttt{tabular}
      \end{itemize}
    \end{column}
  \end{columns}
\end{frame}

\begin{frame}{Ergebnis}
  \begin{table}
    \centering
    \caption{Eine Tabelle mit Messdaten.}
    \begin{tabular}{l c c c c}
      \toprule
      $f$ & $l_\text{start}$ & $l_1$ & $l_{\text{kor},1}$ & $B_1$ \\
      \midrule
      100 & 1.14 & 3.51 & 0.00 &   4.30 \\
      300 & 1.27 & 2.42 & 0.13 &  41.14 \\
      500 & 1.21 & 1.70 & 0.25 & 168.73 \\
      \toprule
    \end{tabular}
  \end{table}
  \begin{itemize}
    \item Keine vertikalen Linien!
    \item Keine horizontalen Linien zwischen Daten!
  \end{itemize}
\end{frame}

\begin{frame}[fragile]{Schönere Tabellen mit \texttt{siunitx}}
  \fontsize{8}{6}
  \begin{block}{Code}
    \begin{lstlisting}
      \begin{table}
        \centering
        \caption{Eine schöne Tabelle mit Messdaten.}
        \label{tab:some_data}
        \sisetup{table-format=1.2}
        \begin{tabular}{S[table-format=3.0] S S S S[table-format=3.2]}
          \toprule
          {$f$} & {$l_\text{start}$} & {$l_1$} & {$l_{\text{kor},1}$} & {$B_1$} \\
          \midrule
          100 & 1.14 & 3.51 & 0.00 &   4.30 \\
          200 & 1.30 & 2.99 & 0.06 &  25.98 \\
          300 & 1.27 & 2.42 & 0.13 &  41.14 \\
          400 & 1.28 & 1.47 & 0.20 &  53.76 \\
          500 & 1.21 & 1.70 & 0.25 & 168.73 \\
          \bottomrule
        \end{tabular}
      \end{table}
    \end{lstlisting}
  \end{block}
\end{frame}

\begin{frame}[fragile]{Ergebnis}
  \begin{table}
    \centering
    \caption{Eine gewöhnliche Tabelle mit Messdaten.}
    \sisetup{table-format=1.2}
    \begin{tabular}{S[table-format=3.0] S S S S[table-format=3.2]}
      \toprule
      {$f$} & {$l_\text{start}$} & {$l_1$} & {$l_{\text{kor},1}$} & {$B_1$} \\
      \midrule
      100 & 1.14 & 3.51 & 0.00 &   4.30 \\
      200 & 1.30 & 2.99 & 0.06 &  25.98 \\
      300 & 1.27 & 2.42 & 0.13 &  41.14 \\
      400 & 1.28 & 1.47 & 0.20 &  53.76 \\
      500 & 1.21 & 1.70 & 0.25 & 168.73 \\
      \bottomrule
    \end{tabular}
  \end{table}
  \begin{itemize}
    \item \texttt{S}-Spalte eröffnet mehr Ausrichtungsmöglichkeiten mit \lstinline+\sisetup+ und \lstinline+[...]+
    \item \texttt{s}-Spalte für Einheiten
    \item Standard: Ausrichtung an Dezimalkomma
    \item Spaltennamen durch \lstinline+{ }+ schützen
  \end{itemize}
\end{frame}

\begin{frame}[fragile]{Gruppieren von mehreren Spalten}
  \begin{block}{Kommandostruktur}
    \begin{lstlisting}
      \multicolumn{#Spalten}{Ausrichtung}{Inhalt}
    \end{lstlisting}
  \end{block}
  \fontsize{8}{6}
  \begin{block}{Beispiel}
    \begin{lstlisting}
      \begin{table}
        \centering
        \caption{Messdaten für dubiose Elemente}
        \sisetup{table-format=2.1}
        \begin{tabular}{S[table-format=3.1] S S S S}
          \toprule
          & \multicolumn{2}{c}{Technetium} & \multicolumn{2}{c}{Molybdän} \\
          {$\lambda \mathbin{/} \si{\nano\meter}$}
          & {$\phi_1$} & {$\phi_2$} & {$\phi_1$} & {$\phi_2$} \\
          \midrule
          663.0 & 12.1 & 14.4 & 13.1 & 16.9 \\
          670.0 & 10.9 & 12.9 & 11.8 & 15.7 \\
          678.0 &  9.1 & 11.4 & 10.3 & 14.6 \\
          684.0 &  8.2 & 10.2 &  9.5 & 13.5 \\
          \bottomrule
        \end{tabular}
      \end{table}
    \end{lstlisting}
  \end{block}
\end{frame}

\begin{frame}{Resultat}
  \begin{table}
    \centering
    \caption{Messdaten für dubiose Elemente}
    \sisetup{table-format=2.1}
    \begin{tabular}{S[table-format=3.1] S S S S}
      \toprule
      & \multicolumn{2}{c}{Technetium} & \multicolumn{2}{c}{Molybdän} \\
      {$\lambda \mathbin{/} \si{\nano\meter}$}
      & {$\phi_1$} & {$\phi_2$} & {$\phi_1$} & {$\phi_2$} \\
      \midrule
      663.0 & 12.1 & 14.4 & 13.1 & 16.9 \\
      670.0 & 10.9 & 12.9 & 11.8 & 15.7 \\
      678.0 &  9.1 & 11.4 & 10.3 & 14.6 \\
      684.0 &  8.2 & 10.2 &  9.5 & 13.5 \\
      \bottomrule
    \end{tabular}
  \end{table}
\end{frame}

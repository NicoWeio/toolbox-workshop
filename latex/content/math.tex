\section{Formelsatz}

\subsection{im Fließtext}

\begin{frame}[fragile]{\$\dots \$-Umgebung}
  Für mathematische Symbole, Variablen und kleine Formeln im Fließtext.

  \begin{columns}[T]
    \begin{column}{0.6\textwidth}
      \begin{block}{\LaTeX-Code}
        \begin{lstverbatim}
        $x$
        $x^i$
        $x^12$ bzw. $x^{12}$ % Vorsicht
        $x_\text{max}$
        $U(t) = U_0 \cdot \cos(\omega t)$
        \end{lstverbatim}
      \end{block}
    \end{column}
    \begin{column}{0.35\textwidth}
      \begin{block}{Ergebnis}
        $x$  \\
        $x^i$ \\
        $x^12$ bzw. $x^{12}$ \\
        $x_\text{max}$ \\
        $U(t) = U_0 \cdot \cos(\omega t)$
      \end{block}
    \end{column}
  \end{columns}
\end{frame}

\subsection{Mathe-Umgebungen}

\begin{frame}[fragile]{
  Die \texttt{equation}-Umgebung
  \hfill\doc{http://mirrors.ctan.org/macros/latex/required/amslatex/math/amsldoc.pdf}{amsmath}
  \doc{http://mirrors.ctan.org/macros/latex/contrib/mathtools/mathtools.pdf}{mathtools}
}
  \begin{block}{benötigte Pakete}
    \begin{lstverbatim}
    \usepackage{amsmath}
    \usepackage{amssymb}
    \usepackage{mathtools}
    \end{lstverbatim}
  \end{block}
  Abgesetzte Gleichung, automatische Nummerierung. \\
  \texttt{equation*} erzeugt unnummerierte Gleichung.
  \begin{columns}[T]
    \begin{column}{0.6\textwidth}
      \begin{block}{\LaTeX-Code}
        \begin{lstverbatim}
        \begin{equation}
          \nabla \cdot \vec{E} =
          \frac{\rho} {\epsilon_0}
          \label{eqn:maxwell1}
        \end{equation}
        Schon Gauß hatte das Durchflutungsgesetz \eqref{eqn:maxwell1} aufgestellt.
        \end{lstverbatim}
      \end{block}
    \end{column}
    \begin{column}{0.35\textwidth}
      \begin{block}{Ergebnis}
        \begin{equation}
          \nabla \cdot \vec{E} =
          \frac{\rho}{\epsilon_0}
          \label{eqn:maxwell1}
        \end{equation}
        Schon Gauß hatte das Durchflutungsgesetz \eqref{eqn:maxwell1} aufgestellt.
      \end{block}
    \end{column}
  \end{columns}
\end{frame}

\begin{frame}[fragile]{
  Die \texttt{align}-Umgebung
  \hfill\doc{http://mirrors.ctan.org/macros/latex/required/amslatex/math/amsldoc.pdf}{amsmath}
}
  \begin{columns}[t]
    \begin{column}{0.35\textwidth}
      \begin{itemize}
        \item für mehrere Gleichungen
        \item \texttt{\&} steuert Ausrichtung
      \end{itemize}
    \end{column}
    \begin{column}{0.6\textwidth}
      \begin{itemize}
        \item \texttt{\textbackslash\textbackslash} erzeugt neue Zeile
        \item jede Zeile bekommt eine Gleichungsnummer
      \end{itemize}
    \end{column}
  \end{columns}
  \vfill
  \begin{columns}[T]
    \begin{column}{0.65\textwidth}
      \begin{block}{\LaTeX-Code}
        \begin{lstverbatim}
        \begin{align}
          a         &= 1 & b           &= 2 \\
          a \cdot b &= 5 & \frac{a}{b} &= \num{0,5}
        \end{align}
        \end{lstverbatim}
      \end{block}
    \end{column}
    \begin{column}{0.3\textwidth}
      \begin{block}{Ergebnis}
        \begin{align}
          a         &= 1 & b           &= 2 \\
          a \cdot b &= 2 & \frac{a}{b} &= \num{0,5}
        \end{align}
      \end{block}
    \end{column}
  \end{columns}
\end{frame}

\begin{frame}[fragile]{
  Symbol-Sammlung
  \hfill\doc{http://mirrors.ctan.org/info/symbols/comprehensive/symbols-a4.pdf}{symbols-a4}
}
  Praktischer Link: \\
  \url{http://detexify.kirelabs.org/classify.html} \\
  (Symbol malen und LaTeX-Code angezeigt bekommen)
  \begin{columns}[T]
    \begin{column}{0.7\textwidth}
      \begin{block}{\LaTeX-Code}
        \begin{lstverbatim}
        \begin{align*}
          \leq \geq \gg \ll \approx \propto \\
          \cdot \times \bar{x} \vec{x} \vec{\jmath} \\
          \pm \mp \infty \partial \nabla \\
          \int\limits_0^1 \sum\limits_{i=1}^N \\
          \iint \iiint \oint
        \end{align*}
        \end{lstverbatim}
      \end{block}
    \end{column}
    \begin{column}{0.25\textwidth}
      \begin{block}{Ergebnis}
        \begin{align*}
          \leq \geq \gg \ll \approx \propto \\
          \cdot \times \bar{x} \vec{x} \vec{\jmath} \\
          \pm \mp \infty \partial \nabla \\
          \int\limits_0^1 \sum\limits_{i=1}^N \\
          \iint \iiint \oint
        \end{align*}
      \end{block}
    \end{column}
  \end{columns}
\end{frame}

\begin{frame}[fragile]{Mehr Mathematik}
  \begin{columns}[T]
    \column{0.7\textwidth}
    \begin{block}{\LaTeX-Code}
      \begin{lstverbatim}
      \begin{align*}
        \alpha, \beta, \gamma, \delta \\ 
        \Alpha, \Beta, \Gamma, \Delta \\
        \frac12 \quad \frac{3x^2 + 2x}{4x - 7} \\
        \sin(x) \quad \cos(x) \\
        \lim_{x \to \infty} \exp{-x} = 0 \\
        \sqrt[n]{a^2 + b^2} = c \\
        \langle x \rangle \; \text{vs.} \; < x >  \\
        \left(\frac{x^2}{y^3}\right)
      \end{align*}
      \end{lstverbatim}
    \end{block}
    \column{0.25\textwidth}
    \begin{block}{Resultat}
      \begin{align*}
        \alpha, \beta, \gamma, \delta \\ 
        \Alpha, \Beta, \Gamma, \Delta \\
        \frac12 \quad \frac{3x^2 + 2x}{4x - 7} \\
        \sin(x) \quad \cos(x) \\
        \lim_{x \to \infty} \exp{(-x)} = 0 \\
        \sqrt[n]{a^2 + b^2} = c \\
        \langle x \rangle \; \text{vs.} \; < x >  \\
        \left(\frac{x^2}{y^3}\right)
      \end{align*}
    \end{block}
  \end{columns}
  
\end{frame}

\begin{frame}[fragile]{Konventionen: Variablen, Zahlen, Einheiten, Indizes}
  \begin{itemize}
    \item Variablen werden kursiv gesetzt.
      Dies geschieht im Mathematikmodus automatisch.
    \item Einheiten werden aufrecht gesetzt und haben ein kleines Leerzeichen (\verb+\,+) Abstand zu ihrer Zahl.
      Am besten benutzt man hierfür immer \texttt{siunitx}.
    \item Die eulersche Zahl $\mathrm{e}$, das imaginäre $\mathrm{i}$ und das infinitesimale $\mathrm{d}$ werden ebenfalls aufrecht gesetzt.
      Im Mathematikmodus erreicht man dies mit
      \begin{lstverbatim}
      \mathrm{e}, \mathrm{d}, \mathrm{i}.
      \end{lstverbatim}
    \item Bestehen Indizes aus Text, wie min oder max, so wird dies ebenfalls aufrecht gesetzt.
      \begin{lstverbatim}
      x_\text{min}
      \end{lstverbatim}
    \item ein $\mathrm{d}x$ sollte durch ein kleines Leerzeichen (\verb+\,+) vom Integranden abgetrennt werden.
  \end{itemize}
\end{frame}

\begin{frame}{Übung: Maxwell-Gleichungen}
  Erstellt mit Hilfe der \texttt{align}-Umgebung die Maxwellgleichungen:
  \begin{block}{Ergebnis}
    \begin{align}
      \nabla \cdot \vec{E} &= \frac{\rho} {\epsilon_0} &
      \nabla \cdot \vec{B} &= 0 \\
      \nabla \times \vec{E} &= - \partial_t \vec{B} &
      \nabla \times \vec{B} &= \mu_0 \vec{\jmath} + \mu_0 \epsilon_0 \partial_t \vec{E}
      \label{align}
    \end{align}
  \end{block}
\end{frame}

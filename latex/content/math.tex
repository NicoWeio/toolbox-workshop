\section{Formelsatz}
\subsection{im Fließtext}
\begin{frame}[fragile]
   \frametitle{\$ \dots \$-Umgebung}
   Für mathematische Symbole, Variablen und kleine Formeln im Fließtext.

    \begin{columns}[T]
        \begin{column}{0.6\textwidth}
            \begin{block}{\LaTeX-Code}
                \begin{lstverbatim}
                $x$
                $x^i$
                $x^12$ bzw. $x^{12}$ % Vorsicht
                $x_\mathrm{max}$
                $U(t) = U_0 \cdot \cos(\omega t)$
                \end{lstverbatim}
            \end{block}
        \end{column}
        \begin{column}{0.35\textwidth}
            \begin{block}{Ergebnis}
                $x$  \\
                $x^i$ \\
                $x^12$ bzw. $x^{12}$ \\
                $x_\mathrm{max}$ \\
                $U(t) = U_0 \cdot \cos(\omega t)$
            \end{block}
        \end{column}
    \end{columns}
\end{frame}
\subsection{Mathe-Umgebungen}
\begin{frame}[fragile]
    \frametitle{Die \texttt{equation}-Umgebung
        \hfill\doc{http://ftp.fau.de/ctan/macros/latex/required/amslatex/math/amsldoc.pdf}{amsmath}
        \doc{http://ftp.uni-erlangen.de/mirrors/CTAN/macros/latex/contrib/mathtools/mathtools.pdf}{mathtools}
    }
   \begin{block}{benötigte Pakete}
    \begin{lstverbatim}
        \usepackage{amsmath}
        \usepackage{amssymb}
        \usepackage{mathtools}
    \end{lstverbatim}
   \end{block}
    Abgesetzte Gleichung, automatische Nummerierung. \\
    \texttt{equation*} erzeugt unnummerierte Gleichung.
    \begin{columns}[T]
        \begin{column}{0.6\textwidth}
            \begin{block}{\LaTeX-Code}
                \begin{lstverbatim}
                \begin{equation}
                    \nabla \cdot \vec{E} =
                    \frac{\rho} {\epsilon_0}
                    \label{eq:maxwell1}
                \end{equation}
                \end{lstverbatim}
            \end{block}
        \end{column}
        \begin{column}{0.35\textwidth}
            \begin{block}{Ergebnis}
                \begin{equation}
                    \nabla \cdot \vec{E} =
                    \frac{\rho}{\epsilon_0}
                    \label{eq:maxwell1}
                \end{equation}
            \end{block}
        \end{column}
    \end{columns}
\end{frame}
\begin{frame}[fragile]
    \frametitle{Die \texttt{align}-Umgebung \hfill\doc{http://ftp.fau.de/ctan/macros/latex/required/amslatex/math/amsldoc.pdf}{amsmath}}
    \begin{columns}[t]
        \begin{column}{0.35\textwidth}
            \begin{itemize}
                \item für mehrere Gleichungen
                \item \texttt{\&} steuert Ausrichtung
            \end{itemize}
        \end{column}
        \begin{column}{0.6\textwidth}
            \begin{itemize}
                \item \texttt{\textbackslash\textbackslash} erzeugt neue Zeile
                \item jede Zeile bekommt eine Gleichungsnummer
            \end{itemize}
        \end{column}
    \end{columns}
    \vfill
    \begin{columns}[T]
        \begin{column}{0.65\textwidth}
            \begin{block}{\LaTeX-Code}
                \begin{lstverbatim}
                \begin{align}
                    a &= 1       & b&=2 \\
                    a\cdot b &=5 & \frac{a}{b} &= \num{0,5}
                \end{align}
                \end{lstverbatim}
            \end{block}
        \end{column}
        \begin{column}{0.3\textwidth}
            \begin{block}{Ergebnis}
                \begin{align}
                    a &= 2       & b&=2 \\
                    a\cdot b &=2 & \frac{a}{b} &= \num{0,5}
                \end{align}
            \end{block}
        \end{column}
    \end{columns}
\end{frame}
\begin{frame}[fragile]
    \frametitle{Symbol-Sammlung}
    Praktischer Link: \\
    \href{http://detexify.kirelabs.org/classify.html}{http://detexify.kirelabs.org/classify.html} \\
    (Symbol malen und LaTeX-Code angezeigt bekommen)
    \begin{columns}[T]
        \begin{column}{0.7\textwidth}
            \begin{block}{\LaTeX-Code}
                \begin{lstverbatim}
                \begin{align}
                    \leq \geq \gg \ll \approx \propto \\
                    \cdot \times \partial \bar{x} \vec{x} \\
                    \pm \mp \infty \partial \nabla \\
                    \int\limits_0^1 \sum\limits_{i=1}^N \\
                    \iint \iiint \oint
                \end{align}
                \end{lstverbatim}
            \end{block}
        \end{column}
        \begin{column}{0.25\textwidth}
            \begin{block}{Ergebnis}
                \begin{align*}
                    \leq \geq \gg \ll \approx \propto \\
                    \cdot \times \bar{x} \vec{x} \\
                    \pm \mp \infty \partial \nabla \\
                    \int\limits_0^1 \sum\limits_{i=1}^N \\
                    \iint \iiint \oint
                \end{align*}
            \end{block}
        \end{column}
    \end{columns}
\end{frame}
\begin{frame}[fragile]
    \frametitle{Konventionen: Variablen, Zahlen, Einheiten, Indizes}
    \begin{itemize}
        \item Variablen werden kursiv gesetzt. Dies geschieht im Mathematikmodus automatisch.
        \item Einheiten werden aufrecht gesetzt und haben ein kleines Leerzeichen (\verb+\,+) Abstand zu ihrer Zahl. Am besten benutzt man hierfür immer \texttt{siunitx}.
        \item Die eulersche Zahl e, das imaginäre i und das infinitesimale d werden ebenfalls aufrecht gesetzt. Im Mathematikmodus erreicht man dies mit
            \begin{lstverbatim}
            \mathrm{e}, \mathrm{d}, \mathrm{i}.
            \end{lstverbatim}
        \item Bestehen Indizes aus Text, wie min oder max, so wird dies ebenfalls aufrecht gesetzt.
            \begin{lstverbatim}
            x_\text{min}
            \end{lstverbatim}
        \item ein $\mathrm{d}x$ sollte durch ein kleines Leerzeichen (\verb+\,+) vom Integranden abgetrennt werden.
    \end{itemize}
\end{frame}
\begin{frame}
    \frametitle{Übung: Maxwell-Gleichungen}
    Erstellt mit Hilfe der \texttt{align}-Umgebung die Maxwellgleichungen:
    \begin{block}{Ergebnis}
        \begin{align}
            \nabla \cdot \vec{E} &= \frac{\rho} {\epsilon_0} &
            \nabla \cdot \vec{B} &= 0 \\
            \nabla \times \vec{E} &= - \partial_t \vec{B} &
            \nabla \times \vec{B} &= \mu_0 \vec{j} + \mu_0 \epsilon_0 \partial_t \vec{E}
            \label{align}
        \end{align}
    \end{block}
\end{frame}

\section{\TeX\ in \texttt{matplotlib} in \TeX}

\begin{frame}[fragile]{\TeX\ in \texttt{matplotlib} (1)}
  \lstinputpython[firstline=3]{script/mattex1.py}
\end{frame}

\AddToShipoutPictureFG*{\ShowFramePicture}
\begin{frame}{Ergebnis (1)}
  \centering
  \includegraphics{figures/mattex1.pdf}
\end{frame}

\begin{frame}[fragile]{\TeX\ in \texttt{matplotlib} (2)}
  \lstinputpython[firstline=3]{script/mattex2.py}
\end{frame}

\begin{frame}[fragile]{Bildgröße}
  \lstpythoninline+plt.figure(figsize=(4.76, 2.94))+
  \begin{itemize}
    \item Größe der Zeichenfläche setzen (in Zoll)
    \item Breite des textes kann mit \lstinline+\the\textwidth+ ins Dokument geschrieben werden
    \item $1\,\mathrm{in} = \num{72.27}\,\mathrm{pt}$
    \item Goldener Schnitt für Höhe
    \item Für \texttt{scrartcl} mit Standardeinstellungen: \texttt{5.78, 3.57}
  \end{itemize}

  \begin{lstpython}
    plt.tight_layout(pad=0)
    plt.savefig(..., bbox_inches='tight', pad_inches=0)
  \end{lstpython}
  \begin{itemize}
    \item Weiße Leerräume am Rand eliminieren
    \item Inhalt des Bilds ist genauso breit wie der Text
  \end{itemize}
\end{frame}

\AddToShipoutPictureFG*{\ShowFramePicture}
\begin{frame}{Ergebnis (2)}
  \centering
  \includegraphics{figures/mattex2.pdf}
\end{frame}

\begin{frame}[fragile]{\TeX\ in \texttt{matplotlib} (3)}
  \lstinputpython[firstline=3]{script/mattex3.py}
\end{frame}

\AddToShipoutPictureFG*{\ShowFramePicture}
\begin{frame}{Ergebnis (3)}
  \centering
  \includegraphics{figures/mattex3.pdf}
\end{frame}

\begin{frame}[fragile]{\TeX\ in \texttt{matplotlib} (4)}
  \lstinputpython[firstline=3]{script/mattex4.py}
\end{frame}

\begin{frame}[fragile]{\texttt{header-matplotlib.tex}}
  \lstinputlisting{header-matplotlib.tex}

  \vspace*{-0.5em}
  \begin{itemize}
    \item \TeX\ wird von \texttt{matplotlib} in \texttt{/tmp} ausgeführt
      \begin{itemize}
        \item Datei kann nicht gefunden werden
      \end{itemize}
    \item Lösung: \texttt{TEXINPUTS} setzen!
    \item \texttt{TEXINPUTS=\$(pwd): python script/mattex4.py}
    \item \texttt{Makefile}: \texttt{TEXINPUTS=\$\$(pwd): python script/mattex4.py}
  \end{itemize}
  \vspace*{-1em}
\end{frame}

\begin{frame}[fragile]{Windows}
  Hier funktioniert Windows leider anders als Linux/Max.
  Auf Windows muss man Python so starten:

  \texttt{TEXINPUTS="\$(cygpath -m "\$(pwd)")" python script/mattex4.py}

  Hier eine Makefile, die überall funktioniert:

  \begin{lstmake}
    ifeq (,$(shell sh -c 'cygpath --version 2> /dev/null'))
      # Unix
      pwd := $$(pwd)
      translate = $1
    else
      # Windows mit MSYS2/Cygwin
      pwd := $$(cygpath -m "$$(pwd)")
      translate = $(shell echo '$1' | sed 's/:/;/g')
    endif

    build/document.pdf: ...
        TEXINPUTS="$(call translate,build:)" ...

    build/figures/mattex4.pdf: script/mattex4.py
        TEXINPUTS="$(call translate,$(pwd):)" python script/mattex4.py
  \end{lstmake}
\end{frame}

\AddToShipoutPictureFG*{\ShowFramePicture}
\begin{frame}{Ergebnis (4)}
  \centering
  \includegraphics{figures/mattex4.pdf}
\end{frame}

\begin{frame}[fragile]{\TeX\ in \texttt{matplotlib} (5)}
  \lstinputpython[firstline=3]{script/mattex5.py}
\end{frame}

\begin{frame}[fragile]{\texttt{matplotlibrc}}
  \lstMatplotlibrcSettings
  \lstinputlisting{matplotlibrc.txt}

  \begin{itemize}
    \item Datei heißt \texttt{matplotlibrc} ohne Endung!
    \item Wird im aktuellen Verzeichnis gesucht
      \begin{itemize}
        \item nicht unbedingt gleich dem Ordner, wo das Skript liegt
      \end{itemize}
  \end{itemize}
\end{frame}

\AddToShipoutPictureFG*{\ShowFramePicture}
\begin{frame}{Ergebnis (5)}
  \centering
  \includegraphics{figures/mattex5.pdf}
\end{frame}

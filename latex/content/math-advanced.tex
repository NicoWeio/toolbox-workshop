\section{Fortgeschrittener Formelsatz}

\begin{frame}[fragile]{
  Matrizen
  \hfill\doc{http://mirrors.ctan.org/macros/latex/required/amslatex/math/amsldoc.pdf}{amsmath}
  \doc{http://mirrors.ctan.org/macros/latex/contrib/mathtools/mathtools.pdf}{mathtools}
}
  Matrizen (und damit Vektoren) werden fett geschrieben.
  \begin{CodeExample}{0.70}
    \begin{lstlisting}
      \symbf{M}^\top \symbf{M}^*
      \symbf{M}^\dagger \symbf{M}^{-1} M_{12}
    \end{lstlisting}
  \CodeResult
    $\symbf{M}^\top \symbf{M}^*$ \\
    $\symbf{M}^\dagger \symbf{M}^{-1} M_{12}$
  \end{CodeExample}

  \fontsize{8}{6}
  \begin{CodeExample}{0.70}
    \begin{lstlisting}
      \begin{pmatrix} \begin{bmatrix} \begin{Bmatrix*}[r]
        x & y \\        -1 \\           -1 \\
        z & abc          2               2
      \end{pmatrix}   \end{bmatrix}   \end{Bmatrix*}
    \end{lstlisting}
  \CodeResult
    \removedisplayskip
    \begin{equation*}
      \begin{pmatrix}
        x & y \\
        z & abc
      \end{pmatrix}
      \begin{bmatrix}
        -1 \\
         2
      \end{bmatrix}
      \begin{Bmatrix*}[r]
        -1 \\
         2
      \end{Bmatrix*}
    \end{equation*}
  \end{CodeExample}

  \vspace{1em}
  Mehr Möglichkeiten in der Doku.
\end{frame}

\begin{frame}[fragile]{Unnötigen Leerraum loswerden}
  \lstinline+mathtools+ stellt zwei wichtige Befehle, um Leerraum zu eliminieren:
  \begin{CodeExample}{0.7}
    \begin{lstlisting}
      \lim_{x\to\infty} f(x)

      \lim_{\mathclap{x\to\infty}} f(x)

      \sum_{i \in \{1, 2, 3, 4, 5\}} i

      \sum_{\mathclap{i \in \{1, 2, 3, 4, 5\}}} i
    \end{lstlisting}
  \CodeResult
    \begin{minipage}[t][2\baselineskip][t]{\textwidth}
      $\displaystyle \lim_{x\to\infty} f(x)$
    \end{minipage} \\\nointerlineskip
    \begin{minipage}[t][2\baselineskip][t]{\textwidth}
      $\displaystyle \lim_{\mathclap{x\to\infty}} f(x)$
    \end{minipage} \\\nointerlineskip
    \begin{minipage}[t][2\baselineskip][t]{\textwidth}
      $\displaystyle\sum_{i \in \{1, 2, 3, 4, 5\}} i$
    \end{minipage}\\\nointerlineskip
    \begin{minipage}[t][2\baselineskip][t]{\textwidth}
      $\qquad\displaystyle\sum_{\mathclap{i \in \{1, 2, 3, 4, 5\}}} i$
    \end{minipage}
  \end{CodeExample}

  \begin{CodeExample}{0.7}
    \begin{lstlisting}
      2^{2^2} \cramped{2^{2^2}}
    \end{lstlisting}
  \CodeResult
    $2^{2^2} \cramped{2^{2^2}}$
  \end{CodeExample}
  
\end{frame}

\begin{frame}[fragile]{Fallunterscheidungen}
  Befehle mit * aktivieren den Textmodus nach dem \lstinline+&+.
  \begin{CodeExample}{0.55}
    \begin{lstlisting}
      f(x) =
      \begin{cases}
         x ,         & x \ge 0 \\
        \int_0^1 x , & \text{sonst}
      \end{cases}

      \begin{drcases*}
         x ,         & $x < 0$ \\
        \int_0^1 x , & sonst
      \end{drcases*}
      \ne - \lvert x \rvert
    \end{lstlisting}
  \CodeResult
    \removedisplayskip
    \begin{equation*}
      f(x) =
      \begin{cases}
         x ,         & x \ge 0 \\
        \int_0^1 x , & \text{sonst}
      \end{cases}
    \end{equation*}

    \vspace{3em}
    \begin{equation*}
      \begin{drcases*}
         x ,         & $x < 0$ \\
        \int_0^1 x , & sonst
      \end{drcases*}
      \ne - \lvert x \rvert
    \end{equation*}
  \end{CodeExample}
\end{frame}

\begin{frame}[fragile]{\lstinline+\\intertext+}
  \lstinline+\intertext+ erhält die Ausrichtung der \lstinline+align+-Umgebung.
  \begin{CodeExample}{0.60}
    \begin{lstlisting}
      Es gilt
      \begin{align*}
        f  &= xyz , \\
        \intertext{wobei dies ein langer Erklärungstext ist, und dass}
        g' &= \symup{e}^x \\
        \shortintertext{von}
        g  &= \symup{e}^x
      \end{align*}
      gelöst wird.
    \end{lstlisting}
  \CodeResult
    Es gilt
    \begin{align*}
      f  &= xyz , \\
      \intertext{wobei dies ein langer Erklärungstext ist, und dass}
      g' &= \symup{e}^x \\
      \shortintertext{von}
      g  &= \symup{e}^x
    \end{align*}
    gelöst wird.
  \end{CodeExample}
\end{frame}

\begin{frame}[fragile]{\lstinline+\\underbrace+}
  \begin{CodeExample}{0.58}
    \begin{lstlisting}
      f(x)= \underbrace{g(x)}
        _{x + x^2 + x^3}
      + \overbrace{h(x)}
        ^{\mathclap{\text{mehr Erklärung}}}
      + 2
    \end{lstlisting}
  \CodeResult
    \begin{equation*}
      f(x)= \underbrace{g(x)}_{x + x^2 + x^3}
      + \overbrace{h(x)}^{
        \mathclap{\text{mehr Erklärung}}
      } + 2
    \end{equation*}
  \end{CodeExample}
\end{frame}

\begin{frame}[fragile]{\lstinline+\\phantom+}
  \begin{CodeExample}{0.60}
    \begin{lstlisting}
      \begin{align*}
        f_{123} &=          2  x + 3y + z \\
        g       &= \phantom{2} x + 3y + z \\
        h       &=             x + 3y + z
      \end{align*}
    \end{lstlisting}
  \CodeResult%
  \removedisplayskip
    \begin{align*}
      f_{123} &=          2  x + 3y + z \\
      g       &= \phantom{2} x + 3y + z \\
      h       &=             x + 3y + z
    \end{align*}
  \end{CodeExample}
  \begin{CodeExample}{0.70}
    \begin{lstlisting}
      \begin{align*}
        f_{123} &=           \frac{1}{2}  x + y \\
        g       &= \hphantom{\frac{1}{2}} x + y \\
        h       &=  \phantom{\frac{1}{2}} x + y
      \end{align*}
    \end{lstlisting}
  \CodeResult
    \mathtoolsset{shortvdotsadjustbelow=1.8em}
    \removedisplayskip
    \begin{align*}
      f_{123} &=           \frac{1}{2}  x + y \\
      g       &= \hphantom{\frac{1}{2}} x + y \\
      h       &=  \phantom{\frac{1}{2}} x + y
    \end{align*}
  \end{CodeExample}
  \lstinline+\hphantom+ wirkt nur horizontal und hat keine Höhe. \\
\end{frame}

\begin{frame}[fragile]{\lstinline+\\vphantom+}
  \begin{CodeExample}{0.70}
    \begin{lstlisting}
      \begin{align*}
        f = {} & \left( \frac{1}{2} \right. \\
               & \left. \vphantom{\frac{1}{2}} + x \right)^{\! 2}
      \end{align*}
    \end{lstlisting}
  \CodeResult
    \removedisplayskip
    \begin{align*}
      f = {} & \left( \frac{1}{2} \right. \\
             & \left. \vphantom{\frac{1}{2}} + x \right)^{\! 2}
    \end{align*}
  \end{CodeExample}
  \vspace{5pt}
  \lstinline+\vphantom+ wirkt nur vertikal und hat keine Breite.
\end{frame}

\section{Präsentationen mit \LaTeX: \lstinline+beamer+}

\begin{frame}[fragile]{\lstinline+beamer+}
  \begin{itemize}
    \item Dokumentenklasse für Präsentationen
    \item \lstinline+frame+-Umgebung erzeugt eine Folie
    \item Bei Nutzung mit \lstinline+fontspec+ und \lstinline+unicode-math+ muss
      \begin{lstlisting}
      \usefonttheme{professionalfonts}
      \end{lstlisting}
      vor diesen Paketen gesetzt werden.
    \item Aussehen wird durch \enquote{themes} gesteuert.
    \item Viele themes werden mit \TeX-Live mitgeliefert.
    \item Sehen leider alle fast gleich aus.
  \end{itemize}
\end{frame}
\begin{frame}[fragile]{Minimal-Beispiel}
  \begin{center}
    \begin{lstlisting}
      \documentclass{beamer}

      \usefonttheme{professionalfonts}
      \usepackage{fontspec}
      \usepackage[
        math-style=ISO,
        bold-style=ISO,
        nabla=upright,
        partial=upright,
        sans-style=italic,
      ]{unicode-math}
      \setmathfont{Latin Modern Math}

      \begin{document}
        \begin{frame}{title}
          Hallo Welt!
        \end{frame}
      \end{document}
    \end{lstlisting}
  \end{center}
\end{frame}

\begin{frame}[fragile]{Nervige Buttons abschalten}
  \begin{center}
    \begin{lstlisting}
      \documentclass{beamer}
      %  …
      %  packages here
      %  …

      \setbeamertemplate{navigation symbols}{}

      \begin{document}
        \begin{frame}{title}
          Hallo Welt!
        \end{frame}
      \end{document}
    \end{lstlisting}
  \end{center}
\end{frame}


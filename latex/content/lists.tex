\section{Aufzählungen}

\begin{frame}[fragile]{Aufzählungen: Itemize}
  \begin{itemize}
    \item \LaTeX\ bietet drei Umgebungen für Aufzählungen
    \item Standardeinstellungen gut, Änderungen mit Paket \texttt{enumitem}
    \item Verschachteln für Unterpunkte
    \item Unnummerierte Listen: \texttt{itemize}
  \end{itemize}
  \begin{CodeExample}{0.48}
    \begin{lstlisting}
      \begin{itemize}
        \item Punkt 1
        \item Punkt 2
          \begin{itemize}
            \item Unterpunkt 1
            \item Unterpunkt 2
          \end{itemize}
        \item[→] Punkt 3
      \end{itemize}
    \end{lstlisting}
  \CodeResult
  \setbeamerfont{item}{family=\rmfamily, size=\normalfont}
  \setbeamerfont{itemize/enumerate body}{family=\rmfamily}
  \setbeamerfont{itemize/enumerate subbody}{family=\rmfamily}
  \setbeamertemplate{itemize item}{\bullet}
  \setbeamertemplate{itemize subitem}{--}
    \begin{itemize}
      \item Punkt 1
      \item Punkt 2
        \begin{itemize}
          \item Unterpunkt 1
          \item Unterpunkt 2
        \end{itemize}
      \item[→] Punkt 3
    \end{itemize}
  \end{CodeExample}
\end{frame}

\begin{frame}[fragile]{Aufzählungen: Enumerate}
  Für nummerierte Listen wird \texttt{enumerate} genutzt.
  \begin{CodeExample}{0.48}
    \begin{lstlisting}
      \begin{enumerate}
        \item Punkt 1
        \item Punkt 2
          \begin{enumerate}
            \item Unterpunkt 1
            \item Unterpunkt 2
          \end{enumerate}
        \item Punkt 3
      \end{enumerate}
    \end{lstlisting}
  \CodeResult
    \setbeamerfont{item}{family=\rmfamily, size=\normalfont}
    \setbeamerfont{itemize/enumerate body}{family=\rmfamily}
    \setbeamerfont{itemize/enumerate subbody}{family=\rmfamily}
    \setbeamertemplate{enumerate item}{\theenumi .}
    \setbeamertemplate{enumerate subitem}{(\alph{enumii})}
    \begin{enumerate}
      \item Punkt 1
      \item Punkt 2
        \begin{enumerate}
          \item Unterpunkt 1
          \item Unterpunkt 2
        \end{enumerate}
      \item Punkt 3
    \end{enumerate}
  \end{CodeExample}
\end{frame}

\begin{frame}[fragile]{Aufzählungen: Description}
  Zur Beschreibung von Stichwörtern wird \texttt{description} benutzt, dabei wird das
Stichwort \lstinline+\item+ als optionales Argument übergeben.
  \begin{CodeExample}{0.48}
    \begin{lstlisting}
      \begin{description}
        \item[\LaTeX] gut
        \item[Word] böse
      \end{description}
    \end{lstlisting}
  \CodeResult
    \setbeamerfont{description item}{series=\bfseries}
    \begin{description}
      \item[\LaTeX] gut
      \item[Word] böse
    \end{description}
  \end{CodeExample}
\end{frame}

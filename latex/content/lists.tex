\section{Aufzählungen}

\begin{frame}[fragile]{Aufzählungen: Itemize}
  \LaTeX\ bietet verschiedene: Umgebungen für Aufzählungen.
  Für unnummerierte Listen wird \texttt{itemize} genutzt.
  Wie alle anderen Umgebungen können diese auch verschachtelt werden.
  \begin{columns}[onlytextwidth, t]
    \begin{column}{0.48\textwidth}
      \begin{block}{Code}
        \begin{lstlisting}
        \begin{itemize}
          \item Punkt 1
          \item Punkt 2
            \begin{itemize}
              \item Unterpunkt 1
              \item Unterpunkt 2
            \end{itemize}
          \item[→] Punkt 3
        \end{itemize}
        \end{lstlisting}
      \end{block}
    \end{column}
    \begin{column}{0.48\textwidth}
      \begin{block}{Ergebnis}
        \begin{itemize}
          \item Punkt 1
          \item Punkt 2
            \begin{itemize}
              \item Unterpunkt 1
              \item Unterpunkt 2
            \end{itemize}
          \item[→] Punkt 3
        \end{itemize}
      \end{block}
    \end{column}
  \end{columns}
\end{frame}

\begin{frame}[fragile]{Aufzählungen: Enumerate}
  Für nummerierte Listen wird \texttt{enumerate} genutzt.
  \begin{columns}[onlytextwidth, t]
    \begin{column}{0.48\textwidth}
      \begin{block}{Code}
        \begin{lstlisting}
        \begin{enumerate}
          \item Punkt 1
          \item Punkt 2
            \begin{enumerate}
              \item Unterpunkt 1
              \item Unterpunkt 2
            \end{enumerate}
          \item Punkt 3
        \end{enumerate}
        \end{lstlisting}
      \end{block}
    \end{column}
    \begin{column}{0.48\textwidth}
      \begin{block}{Ergebnis}
        \begin{enumerate}
          \item Punkt 1
          \item Punkt 2
            \begin{enumerate}
              \item Unterpunkt 1
              \item Unterpunkt 2
            \end{enumerate}
          \item Punkt 3
        \end{enumerate}
      \end{block}
    \end{column}
  \end{columns}
  Anpassung der Listen mit dem Paket \texttt{enumitem}.
\end{frame}

\begin{frame}[fragile]{Aufzählungen: Description}
  Zur Beschreibung von Stichwörtern wird \texttt{description} benutzt, dabei wird das
Stichwort \verb-\item- als optionales Argument übergeben.
  \begin{columns}[onlytextwidth, t]
    \begin{column}{0.48\textwidth}
      \begin{block}{Code}
        \begin{lstlisting}
        \begin{description}
          \item[\LaTeX] gut
          \item[Word] böse
        \end{description}
        \end{lstlisting}
      \end{block}
    \end{column}
    \begin{column}{0.48\textwidth}
      \begin{block}{Ergebnis}
        \begin{description}
          \item[\LaTeX] gut
          \item[Word] böse
        \end{description}
      \end{block}
    \end{column}
  \end{columns}
\end{frame}

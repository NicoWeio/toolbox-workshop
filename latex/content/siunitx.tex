\section{Zahlen und Einheiten}

\begin{frame}[fragile]{
  Das \texttt{siunitx}-Paket
  \hfill\doc{http://mirrors.ctan.org/macros/latex/contrib/siunitx/siunitx.pdf}{siunitx}
}
\begin{itemize}
    \item Einheiten werden aufrecht gesetzt
    \item zwischen Zahl und Einheit steht ein kleines Leerzeichen \verb+\,+
    \item \texttt{siunitx} stellt Befehle zur Verfügung, die das korrekte Setzen von Zahlen und Einheiten stark vereinfachen
    \item[$\Rightarrow$] Dieses Paket sollte immer und für jede Zahl mit oder ohne Einheit verwendet werden.
\end{itemize}
  \begin{block}{benötigte Pakete}
    \begin{lstverbatim}
    \usepackage[locale=DE,
                separate-uncertainty=true,  % Immer Fehler mit ±
                per-mode=symbol-or-fraction, % m/s im Text, sonst Bruch
            ]{siunitx}
    \end{lstverbatim}
  \end{block}
\end{frame}

\begin{frame}[fragile]{ Das \texttt{siunitx}-Paket }
  \begin{columns}[T]
    \begin{column}{0.6\textwidth}
      \begin{block}{\LaTeX-Code}
        \begin{lstverbatim}
        \num{1.23456} und \num{987654321}
        \num{1.2e2}
        \si{\newton} = \si{\kilo\gram\metre\per\second\squared}
        \SI{1.2}{\metre\per\second}
        \SI{4.3(12)}{\micro\second}
        \SI{4.3(12)e-6}{\second}
        \end{lstverbatim}
      \end{block}
    \end{column}
    \begin{column}{0.35\textwidth}
      \begin{block}{Ergebnis}
        \num{1.23456} und \num{987654321} \\
        \num{1.2e2} \\
        \smallbreak
        \si{\newton} = \si{\kilo\gram\metre\per\second\squared} \\
        \medbreak
        \SI{1.2}{\metre\per\second} \\
        \SI{4.3(12)}{\micro\second}
        \SI{4.3(12)e-6}{\second}
      \end{block}
    \end{column}
  \end{columns}
\end{frame}

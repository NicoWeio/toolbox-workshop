\section{Zahlen und Einheiten}

\begin{frame}[fragile]{Zahlen und Einheiten}
  \begin{itemize}
    \item Regeln zur Benutzung der SI-Einheiten: \\
      \url{https://www.bipm.org/utils/common/pdf/si-brochure/SI-Brochure-9-EN.pdf}
    \item Einheiten werden aufrecht gesetzt
    \item Zwischen Zahl und Einheit steht ein kleines Leerzeichen
    \item Ab 5 Stellen wird ein kleines Leerzeichen als 1000er Trennzeichen genutzt:
  \end{itemize}

  \begin{CodeExample}{0.7}[Zahl mit Einheit]
    \begin{lstlisting}
      $5\,\mathrm{kg}$
    \end{lstlisting}
    \CodeResult
    $5\,\mathrm{kg}$
  \end{CodeExample}
  \begin{CodeExample}{0.7}[Zahl mit mehr als vier Stellen]
    \begin{lstlisting}
      $10\,000$
    \end{lstlisting}
    \CodeResult
    $10\,000$
  \end{CodeExample}
  \begin{CodeExample}{0.7}[Zehnerpotenz mit Unsicherheit in Klammern]
    \begin{lstlisting}
      $(5{,}34 \pm 0{,}54) \cdot 10^{-3}\,\mathrm{GeV}$
    \end{lstlisting}
    \CodeResult
    $(5{,}34 \pm 0{,}54) \cdot 10^{-3}\,\mathrm{GeV}$
  \end{CodeExample}
  
  \onslide<2->{%
  \begin{center}
    \LARGE  Das muss einfacher gehen
  \end{center}
  }
\end{frame}

\begin{frame}[fragile]{
  Das \texttt{siunitx}-Paket
  \hfill
  \doc{http://mirrors.ctan.org/macros/latex/contrib/siunitx/siunitx.pdf}{siunitx}
}
\begin{itemize}
    \item \texttt{siunitx} stellt Befehle zur Verfügung, die das korrekte Setzen von Zahlen und Einheiten stark vereinfachen
    \item Funktioniert in Fließtext und Matheumgebung
    \item[$\color{vertexDarkRed}\Rightarrow$] Dieses Paket sollte \alert{immer} und für \alert{jede} Zahl mit oder ohne Einheit verwendet werden.
\end{itemize}
  \begin{Packages}
    \begin{lstlisting}
      \usepackage[
        locale=DE,
        separate-uncertainty=true,  % Immer Unsicherheit mit ±
        per-mode=symbol-or-fraction, % m/s im Text, sonst \frac
        % alternativ:
        % per-mode=reciprocal,      % m s^{-1}
        % output-decimal-marker=.,         % . statt , für Dezimalzahlen
      ]{siunitx}
      % Fix missing micro sign with TL2017
      \sisetup{math-micro=\text{µ},text-micro=µ}
    \end{lstlisting}
  \end{Packages}
\end{frame}

\begin{frame}[fragile]{\texttt{siunitx}: Zahlen mit \lstinline+\\num+}
  \begin{CodeExample}{0.7}[Zahlen mit automatischen 3er-Gruppen]
    \begin{lstlisting}
      \num{1.23456}
      \num{987654321}
    \end{lstlisting}
  \CodeResult
    \strut
    \num{1.23456} \\
    \num{987654321}
  \end{CodeExample}
  \begin{CodeExample}{0.7}[Einfaches Eingeben von 10er Potenzen]
    \begin{lstlisting}
      \num{6.022e23}
    \end{lstlisting}
  \CodeResult
    \strut
    \num{6.022e23}
  \end{CodeExample}
  \begin{CodeExample}{0.7}[Angabe von Unsicherheiten]
    \begin{lstlisting}
      \num{1.54 +- 0.1}
      \num{1.54(10)}
      \num{1.54 \pm 0.1}
      \num[separate-uncertainty=false]{1.54 +- 0.1}
      \num{3.5(1)e6}
    \end{lstlisting}
  \CodeResult
    \strut
    \num{1.54 +- 0.1} \\
    \num{1.54(10)} \\
    \num{1.54 \pm 0.1} \\
    \num[separate-uncertainty=false]{1.54 +- 0.1} \\
    \num{3.5(1)e6}
  \end{CodeExample}
\end{frame}

\begin{frame}[fragile]{\texttt{siunitx}: Einheiten mit \lstinline+\\unit+}
  \begin{CodeExample}{0.8}[Einheiten]
    \begin{lstlisting}
      \unit{\meter\per\second}
      \unit[per-mode=fraction]{\meter\per\second}
      \unit{\meter\per\second\squared}
      \unit[per-mode=reciprocal]{\gram\per\cubic\centi\meter}
      \unit{\kelvin\tothe{4}}
    \end{lstlisting}
  \CodeResult
    \strut
    \unit{\meter\per\second} \\
    \unit[per-mode=fraction]{\meter\per\second} \\
    \unit{\meter\per\second\squared} \\
    \unit[per-mode=reciprocal]{\gram\per\cubic\centi\meter} \\
    \unit{\kelvin\tothe{4}}
  \end{CodeExample}
  \begin{CodeExample}{0.80}[\texttt{per-mode=symbol-or-fraction}]
    \begin{lstlisting}
      \begin{equation}
        \unit{\kilo\gram\meter\per\second\squared}
      \end{equation}
      $\unit{\kilo\gram\meter\per\second\squared}$
    \end{lstlisting}
  \CodeResult
    \removedisplayskip
    \begin{minipage}[c][3\baselineskip][c]{\textwidth}
      \begin{equation}
        \unit{\kilo\gram\meter\per\second\squared}
      \end{equation}
    \end{minipage}
    $\unit{\kilo\gram\meter\per\second\squared}$
  \end{CodeExample}
  \begin{CodeExample}{0.80}[Meter mal Sekunde oder Millisekunde?]
    \begin{lstlisting}
      \unit{\milli\second}
      \unit{\meter\second}
      \unit[inter-unit-product=\cdot]{\meter\second}
    \end{lstlisting}
  \CodeResult
    \strut
    \unit{\milli\second} \\
    \unit{\meter\second} \\
    \unit[inter-unit-product=\cdot]{\meter\second}
  \end{CodeExample}
\end{frame}

\begin{frame}[fragile]{\texttt{siunitx}: Physikalische Größe, eine Zahl mit Einheit: \lstinline+\\qty+}
  \begin{CodeExample}{0.72}[\lstinline+\\qty+ {=} Kombination aus \lstinline+\\num+ und \lstinline+\\unit+]
    \begin{lstlisting}
      \qty{5}{\percent}
      \qty{10}{\degreeCelsius}
      \qty{2.5(1)e6}{\kilo\gram\square\meter\per\second\squared}
    \end{lstlisting}
  \CodeResult
    \strut
    \qty{5}{\percent} \\
    \qty{10}{\degreeCelsius} \\
    \qty{2.5(1)e6}{\kilo\gram\square\meter\per\second\squared}
  \end{CodeExample}
  \begin{description}
    \item[1. Argument] Kann alles, was \lstinline+\num+ kann
    \item[2. Argument] Kann alles, was \lstinline+\unit+ kann
  \end{description}
  \begin{CodeExample}{0.72}[Winkel]
    \begin{lstlisting}
      \ang{5;;}
      \ang{;5;}
      \ang{;;5}
      \ang{5;55;}
      \ang{5;55;59}
    \end{lstlisting}
  \CodeResult
    \strut\ang{5;;}\\
    \strut\ang{;5;}\\
    \strut\ang{;;5}\\
    \strut\ang{5;55;}\\
    \strut\ang{5;55;59}\\
  \end{CodeExample}
\end{frame}

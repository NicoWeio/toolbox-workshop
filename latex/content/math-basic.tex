\begin{frame}[fragile]{
  Benötigte Pakete
  \hfill\doc{http://mirrors.ctan.org/macros/latex/required/amslatex/math/amsldoc.pdf}{amsmath}
  \doc{http://mirrors.ctan.org/macros/latex/contrib/mathtools/mathtools.pdf}{mathtools}
}
  \begin{lstlisting}
    \usepackage{amsmath}   % unverzichtbare Mathe-Befehle
    \usepackage{amssymb}   % viele Mathe-Symbole
    \usepackage{mathtools} % Erweiterungen für amsmath

    \usepackage{fontspec} % nach amssymb

    \usepackage[
      math-style=ISO,    % \
      bold-style=ISO,    % |
      sans-style=italic, % | ISO-Standard folgen
      nabla=upright,     % |
      partial=upright,   % /
    ]{unicode-math}      % "Does exactly what it says on the tin."

    % \setmathfont{Latin Modern Math}     % standard
    % \setmathfont{Tex Gyre Pagella Math} % alternativ
  \end{lstlisting}
\end{frame}

\begin{frame}[fragile]{\lstinline+\$...\$+-Umgebung}
  Aktiviert den Mathematikmodus im Fließtext.

  \begin{CodeExample}{0.56}
    \begin{lstlisting}
      Dies ist eine Variable: $x$.
      Liste von Variablen $x$, $y$, $z$.
      Kleine Formel: $a^2 + b^2 = c^2$.
      Vorsicht Höhe: $x^{2^{2^{2^2}}}$
          $x_{2_{2_{2_2}}}$
      Mehr Text. Mehr Text. Mehr
    \end{lstlisting}
  \CodeResult
    Dies ist eine Variable: $x$. \\
    Liste von Variablen $x$, $y$, $z$. \\
    kleine Formel: $a^2 + b^2 = c^2$. \\
    Vorsicht Höhe: $x^{2^{2^{2^2}}}$ $x_{2_{2_{2_2}}}$ \\
    Mehr Text. Mehr Text. Mehr
  \end{CodeExample}

  \vspace{1em}
  \begin{itemize}
    \item Leerzeichen werden im Mathe-Modus ignoriert.
    \item \TeX\ hat Algorithmen für das richtige Spacing.
    \item Satzzeichen gehören nicht in die \lstinline+$...$+-Umgebung!
  \end{itemize}
\end{frame}

\begin{frame}[fragile]{Griechisch und mehr}
  \begin{CodeExample}{0.70}
    \begin{lstlisting}
      \epsilon \theta \kappa \pi \rho \sigma \phi
      \varepsilon \vartheta \varkappa \varpi \varrho \varsigma \varphi
      \Alpha \Beta \Gamma
      \hbar \imath \jmath \ell \wp
      \aleph \beth \gimel
      \partial \eth \nabla \square \increment \infty \diameter \ldots \cdots
    \end{lstlisting}
  \CodeResult
    \Umathordordspacing\textstyle=4mu
    $\epsilon \theta \kappa \pi \rho \sigma \phi$ \\
    $\varepsilon \vartheta \varkappa \varpi \varrho$ \\
    $\quad \varsigma \varphi$ \\
    $\Alpha \Beta \Gamma$ \\
    $\hbar \imath \jmath \ell \wp$ \\
    $\aleph \beth \gimel$ \\
    $\partial \eth \nabla \square \increment \infty$ \\
    $\quad \diameter \ldots \cdots$
  \end{CodeExample}
\end{frame}

\begin{frame}[fragile]{Operatoren und Relationen}
  \vspace{-1em}
  \begin{CodeExample}{0.74}
    \begin{lstlisting}
      + - / \pm \mp \cdot \times
      = \simeq \equiv \cong \approx \propto \sim
      \coloneq
      \to \iff \implies
      \mapsto \leadsto
    \end{lstlisting}
  \CodeResult
    \Umathbinbinspacing\textstyle=4mu
    \Umathrelrelspacing\textstyle=4mu
    $+ - / \pm \mp \cdot \times$ \\
    $= \simeq \equiv \cong \approx \propto \sim$ \\
    $\coloneq$ \\
    $\to \iff \implies$ \\
    $\mapsto \leadsto$
  \end{CodeExample}
  Die meisten Relationen lassen sich durch ein \texttt{n} negieren:
  \vspace{-1em}
  \begin{CodeExample}{0.74}
    \begin{lstlisting}
      \neq \nsime \nexists \nni
    \end{lstlisting}
  \CodeResult
    \Umathbinbinspacing\textstyle=4mu
    \Umathrelrelspacing\textstyle=4mu
  $\neq \nsime \nexists \nni$
  \end{CodeExample}
  Häufig möchte man etwas über eine Relation schreiben:
  \vspace{-1em}
  \begin{CodeExample}{0.74}
    \begin{lstlisting}
      \stackrel{!}{=} \stackrel{\text{def}}{=}
    \end{lstlisting}
  \CodeResult
    $\stackrel{!}{=} \stackrel{\text{def}}{=}$
  \end{CodeExample}
\end{frame}

\begin{frame}[fragile]{Indizes}
  \begin{CodeExample}{0.50}
    \begin{lstlisting}
      x^2 x_2 x²
      x^10 x^{10}
      x' x^' x'' x'^2
      {}^2 x
      x_{min} x_\text{min}

      x^2^2 
      x^{2^2} \cramped{x^{2^2}}
      x_\sqrt[3]{2}
      x_{\sqrt[3]{2}}
    \end{lstlisting}
  \CodeResult
    $x^2 \quad x_2 \quad x²$ \\
    $x^10 \quad x^{10}$ \\
    $x' \quad x^{'} \quad x'' \quad x'^2$ \\
    ${}^2 x$ \\
    $x_{min} \quad x_\text{min}$ \\[\baselineskip]
    \alert{Fehler}\\
    $\quad x^{2^2} \quad \cramped{x^{2^2}}$ \\
    \alert{Fehler} \\
    $\quad x_{\sqrt[3]{2}}$
  \end{CodeExample}
  \begin{itemize}
    \item Man muss häufig den Index in \lstinline+{ }+ schreiben
    \item Beim mehrfachen Hochstellen jeweils \lstinline+{ }+ nötig
    \item Nur wenige Befehle können ohne \lstinline+{ }+ im Index stehen
  \end{itemize}
\end{frame}

\begin{frame}[fragile]{Akzente}
  \begin{CodeExample}{0.70}
    \begin{lstlisting}
      \bar{x}
      \hat{x}
      \tilde{x}
      \vec{x}
      \mathring{x}
      \dot{x} \ddot{x} \dddot{x} \ddddot{x}
      \hat{x_\text{min}} \hat{x}_\text{min}
      \underline{xy} \overline{xy}
    \end{lstlisting}
  \CodeResult
    \Umathordordspacing\textstyle=4mu
    $\bar{x}$ \\
    $\hat{x}$ \\
    $\tilde{x}$ \\
    $\vec{x}$ \\
    $\mathring{x}$ \\
    $\dot{x} \ddot{x} \dddot{x} \ddddot{x}$ \\
    $\hat{x_\text{min}} \hat{x}_\text{min}$ \\
    $\underline{xy} \overline{xy}$
  \end{CodeExample}
\end{frame}

\begin{frame}[fragile]{Funktionen}
  \begin{CodeExample}{0.70}
    \begin{lstlisting}
      x \sin y
      x \sin(y)
      \cos \tan \exp \ln
      \lim_{x \to \infty} x^2
    \end{lstlisting}
  \CodeResult
    $x \sin y$ \\
    $x \sin(y)$ \\
    $\cos \tan \exp \ln$ \\
    $\displaystyle \lim_{x \to \infty} x^2$
  \end{CodeExample}
  \vspace{5pt}
  Man kann auch eigene Funktionen definieren:
  \vspace{-1em}
  \begin{CodeExample}{0.70}
    \begin{lstlisting}
      \operatorname{xyz}_i(a)
      \operatorname*{xyz}_i(a)

      % in Präambel
      \DeclareMathOperator{\xyz}{xyz} \xyz_i(a)
      \DeclareMathOperator*{\Xyz}{Xyz} \Xyz_i(a)
    \end{lstlisting}
  \CodeResult
    $\operatorname{xyz}_i(a)$ \\
    $\displaystyle \operatorname*{xyz}_i(a)$ \\[\baselineskip]
    $\operatorname{xyz}_i(a)$ \\
    $\displaystyle \operatorname*{Xyz}_i(a)$
  \end{CodeExample}
\end{frame}

\begin{frame}[fragile]{Große Operatoren}
  \begin{CodeExample}{0.70}
    \begin{lstlisting}
      \sum_{i=0}^\infty x_i

      \prod \bigotimes

      \int_0^1 \iiint \oint


      \sum\nolimits_0^1 \int\limits_0^1


      \sideset{_a^b}{_c^d}\sum_{i=0}^n

      \sum_{i=1+2+3+4+5+6} x_i

      \sum_{\mathclap{i=1+2+3+4+5+6}} x_i
    \end{lstlisting}
  \CodeResult
    $\displaystyle \sum_{i=0}^\infty x_i$ \\
    $\displaystyle \prod \bigotimes$ \\
    $\displaystyle \int_0^1 \iiint \oint$  \\
    $\displaystyle \sum\nolimits_0^1 \int\limits_0^1$ \\
    $\displaystyle \sideset{_a^b}{_c^d}\sum_{i=0}^n$ \\
    $\displaystyle \sum_{i=1+2+3+4+5+6} x_i$ \\
    \hspace*{0.75cm}$\displaystyle \sum_{\mathclap{i=1+2+3+4+5+6}} x_i$
  \end{CodeExample}
\end{frame}

\begin{frame}[fragile]{Fonts}
  \begin{CodeExample}{0.70}
    \begin{lstlisting}
      x \alpha \mathup{x \alpha}
      \mathbf{x\alpha}
      \mathbfsf{x \alpha}
      \mathbb{R N 1 0 x}
      \mathcal{I A O} \mathbfcal{I A}
      \mathfrak{A B c} \mathbffrak{A B}
    \end{lstlisting}
  \CodeResult
    \Umathordordspacing\textstyle=4mu
    $x \alpha \mathup{x \alpha}$ \\
    $\mathbf{x\alpha}$ \\
    $\mathbfsf{x \alpha}$ \\
    $\mathbb{R N 1 0 x}$ \\
    $\mathcal{I A O} \mathbfcal{I A O}$ \\
    $\mathfrak{A B c} \mathbffrak{A B c}$
  \end{CodeExample}
\end{frame}

\begin{frame}[fragile]{Spaces}
  Manchmal muss man manuell eingreifen, um das Spacing zu perfektionieren.
  \begin{CodeExample}{0.48}
    \begin{lstlisting}
      %
      \,
      \:
      \;
      \quad
      \qquad
      \! % negativer \,
    \end{lstlisting}
  \CodeResult
    $|\mspace{-5mu} |$ \\
    $|\mspace{-5mu} \,|$ \\
    $|\mspace{-5mu} \:|$ \\
    $|\mspace{-5mu} \;|$ \\
    $|\mspace{-5mu} \quad|$ \\
    $|\mspace{-5mu} \qquad|$ \\
    $|\mspace{-5mu} \!|$
  \end{CodeExample}
  \begin{CodeExample}{0.48}
    \begin{lstlisting}
      ^2       ^{\!\! 2}

    \end{lstlisting}
  \CodeResult
    $\displaystyle \left( \frac{2^2}{2} \right)^2
    \qquad \left( \frac{2^2}{2} \right)^{\!\! 2}$
  \end{CodeExample}
\end{frame}

\begin{frame}[fragile]{Klammern}
  \vspace{-1.5em}
  \begin{CodeExample}{0.74}
    \begin{lstlisting}
      () [] \{\} \langle\rangle \lvert\rvert \lVert\rVert
    \end{lstlisting}
  \CodeResult
    \Umathcloseopenspacing\textstyle=4mu
    $() [] \{\} \langle\rangle \lvert x\rvert \lVert x\rVert$
  \end{CodeExample}

  \begin{CodeExample}{0.7}[ Häufig braucht man größere Klammern.]
    \begin{lstlisting}
      \bigl(\bigr) \Bigl(\Bigr) \biggl(\biggr) \Biggl(\Biggr) \bigl<\bigr> \bigl|\bigr|
    \end{lstlisting}
  \CodeResult
    $\bigl(\bigr) \Bigl(\Bigr) \biggl(\biggr) \Biggl(\Biggr)\bigl<\bigr>\bigl|\bigr|$ 
  \end{CodeExample}

  \begin{CodeExample}{0.6}[Automatische Bestimmung der Größe:]
    \begin{lstlisting}
      \left(\right) \left(\right.
      \left\{ \,\middle|\, \right\}
    \end{lstlisting}
  \CodeResult
    $\left(\frac{1}{2}\right) \qquad \left(\frac{1}{2}\right.$ 
    $\left\{ x \,\middle|\, x < \frac{1}{2} \right\}$
  \end{CodeExample}
  
  \begin{CodeExample}{0.705}[hat kein optimales Spacing:]
    \begin{lstlisting}
      \sin(x) \sin\left(x\right) \sin\!\left(x\right)
    \end{lstlisting}
  \CodeResult
    $\sin(x) \sin\left(x\right) \sin\!\left(x\right)$
  \end{CodeExample}
\end{frame}

\begin{frame}[fragile]{
  Symbol-Sammlung
  \hfill\doc{http://mirrors.ctan.org/info/symbols/comprehensive/symbols-a4.pdf}{symbols-a4}
  \hfill\doc{http://mirrors.ctan.org/macros/latex/contrib/unicode-math/unimath-symbols.pdf}{unimath-symbols}
}
  Praktischer Link: \\
  \url{http://detexify.kirelabs.org/classify.html} \\
  (Symbol malen und \LaTeX-Code angezeigt bekommen)
\end{frame}

\begin{frame}[fragile]{Konventionen: Variablen, Zahlen, Einheiten, Indizes}
  \begin{itemize}
    \item Variablen werden kursiv gesetzt.
      Dies geschieht im Mathematikmodus automatisch.
    \item Die eulersche Zahl $\mathup{e}$, das imaginäre $\mathup{i}$, die Kreiszahl $\mathup{\pi}$ und das infinitesimale $\mathup{d}$ werden ebenfalls aufrecht gesetzt.
      Im Mathematikmodus erreicht man dies mit
      \begin{lstlisting}
        \mathup{e}, \mathup{d}, \mathup{\pi}, \mathup{i}.
      \end{lstlisting}
    \item Bestehen Indizes aus Text, wie min oder max, so wird dies ebenfalls aufrecht gesetzt.
      \begin{lstlisting}
        x_\text{min}
      \end{lstlisting}
    \item ein $\mathrm{d}x$ sollte durch ein kleines Leerzeichen (\verb+\,+) vom Integranden abgetrennt werden.

      \vspace{5pt}
      \begin{lstlisting}
        \int_0^1 \int_0^{\mathup{\pi}} \int_0^{2 \mathup{\pi}}
          r^2 \sin(\vartheta) \, \mathup{d}\varphi \, \mathup{d}\vartheta
          \, \mathup{d}r = \frac{4}{3} \mathup{\pi}
      \end{lstlisting}

      \begin{equation*}
        \int_0^1 \int_0^{\mathup{\pi}} \int_0^{2 \mathup{\pi}} r^2 \sin(\vartheta) \, \mathup{d}\phi \, \mathup{d}\vartheta \, \mathup{d}r = \frac{4}{3} \mathup{\pi}
      \end{equation*}
  \end{itemize}
\end{frame}

\begin{frame}[fragile]{
  Benötigte Pakete
  \hfill\doc{http://mirrors.ctan.org/macros/latex/required/amslatex/math/amsldoc.pdf}{amsmath}
  \doc{http://mirrors.ctan.org/macros/latex/contrib/mathtools/mathtools.pdf}{mathtools}
}
  \begin{lstlisting}
    \usepackage{amsmath}   % unverzichtbare Mathe-Befehle
    \usepackage{amssymb}   % viele Mathe-Symbole
    \usepackage{mathtools} % Erweiterungen für amsmath

    \usepackage{fontspec} % nach amssymb

    \usepackage[
      math-style=ISO,    % \
      bold-style=ISO,    % |
      sans-style=italic, % | ISO-Standard folgen
      nabla=upright,     % |
      partial=upright,   % /
    ]{unicode-math}      % "Does exactly what it says on the tin."

    % \setmathfont{Latin Modern Math}     % standard
    % \setmathfont{Tex Gyre Pagella Math} % alternativ
  \end{lstlisting}
\end{frame}

\begin{frame}[fragile]{\lstinline+\$...\$+-Umgebung}
  Aktiviert den Mathematikmodus im Fließtext.

  \begin{CodeExample}{0.56}
    \begin{lstlisting}
      Dies ist eine Variable: $x$.
      Liste von Variablen $x$, $y$, $z$.
      Kleine Formel: $a^2 + b^2 = c^2$.
      Vorsicht Höhe: $x^{2^{2^{2^2}}}$
          $x_{2_{2_{2_2}}}$
    \end{lstlisting}
  \CodeResult
    Dies ist eine Variable: $x$. \\
    Liste von Variablen $x$, $y$, $z$. \\
    kleine Formel: $a^2 + b^2 = c^2$. \\
    Vorsicht Höhe: $x^{2^{2^{2^2}}}$ $x_{2_{2_{2_2}}}$ \\
    Mehr Text. Mehr Text. Mehr
  \end{CodeExample}

  \vspace{1em}
  \begin{itemize}
    \item Leerzeichen werden im Mathe-Modus ignoriert.
    \item \TeX\ hat Algorithmen für das richtige Spacing.
    \item Satzzeichen gehören nicht in die \lstinline+$...$+-Umgebung!
  \end{itemize}
\end{frame}

\begin{frame}[fragile]{Griechisch und mehr}
  \begin{CodeExample}{0.70}
    \begin{lstlisting}
      \alpha \beta \gamma \delta \epsilon \zeta \eta \theta \iota \kappa \lambda \mu \nu \xi \omicron \pi \rho \sigma \tau \upsilon \phi \chi \psi \omega
      \varepsilon \vartheta \varkappa \varpi \varrho \varsigma \varphi
      \Alpha \Beta \Gamma \Delta \Epsilon \Zeta \Eta \Theta \Iota \Kappa \Lambda \Mu \Nu \Xi \Omicron \Pi \Rho \Sigma \Tau \Upsilon \Phi \Chi \Psi \Omega
      \hbar \imath \jmath \ell \wp
      \aleph \beth \gimel
      \partial \eth \nabla \square \increment \infty
    \end{lstlisting}
  \CodeResult
    \Umathordordspacing\textstyle=4mu
    $\alpha \beta \gamma \delta \epsilon \zeta \eta$ \\
    $\quad \theta \iota \kappa \lambda \mu \nu \xi$ \\
    $\quad \omicron \pi \rho \sigma \tau \upsilon \phi$ \\
    $\quad \chi \psi \omega$ \\
    $\varepsilon \vartheta \varkappa \varpi \varrho$ \\
    $\quad \varsigma \varphi$ \\
    $\Alpha \Beta \Gamma \Delta \Epsilon \Zeta \Eta$ \\
    $\quad \Theta \Iota \Kappa \Lambda \Mu \Nu \Xi$ \\
    $\quad \Omicron \Pi \Rho \Sigma \Tau \Upsilon \Phi$ \\
    $\quad \Chi \Psi \Omega$ \\
    $\hbar \imath \jmath \ell \wp$ \\
    $\aleph \beth \gimel$ \\
    $\partial \eth \nabla \square \increment \infty$
  \end{CodeExample}
\end{frame}

\begin{frame}{Operatoren und Relationen}
\end{frame}

\begin{frame}{Indizes}
\end{frame}

\begin{frame}{Akzente}
\end{frame}

\begin{frame}{Funktionen}
\end{frame}

\begin{frame}{Große Operatoren}
\end{frame}

\begin{frame}{Fonts}
\end{frame}

\begin{frame}{Spaces}
\end{frame}

\begin{frame}{Klammern}
\end{frame}

\begin{frame}[fragile]{
  Symbol-Sammlung
  \hfill\doc{http://mirrors.ctan.org/info/symbols/comprehensive/symbols-a4.pdf}{symbols-a4}
}
  Praktischer Link: \\
  \url{http://detexify.kirelabs.org/classify.html} \\
  (Symbol malen und \LaTeX-Code angezeigt bekommen)
\end{frame}

\begin{frame}[fragile]{Konventionen: Variablen, Zahlen, Einheiten, Indizes}
  \begin{itemize}
    \item Variablen werden kursiv gesetzt.
      Dies geschieht im Mathematikmodus automatisch.
    \item Die eulersche Zahl $\mathup{e}$, das imaginäre $\mathup{i}$, die Kreiszahl $\mathup{\pi}$ und das infinitesimale $\mathup{d}$ werden ebenfalls aufrecht gesetzt.
      Im Mathematikmodus erreicht man dies mit
      \begin{lstlisting}
        \mathup{e}, \mathup{d}, \mathup{\pi}, \mathup{i}.
      \end{lstlisting}
    \item Bestehen Indizes aus Text, wie min oder max, so wird dies ebenfalls aufrecht gesetzt.
      \begin{lstlisting}
        x_\text{min}
      \end{lstlisting}
    \item ein $\mathrm{d}x$ sollte durch ein kleines Leerzeichen (\verb+\,+) vom Integranden abgetrennt werden.

      \vspace{5pt}
      \begin{lstlisting}
        \int_0^1 \int_0^{\mathup{\pi}} \int_0^{2 \mathup{\pi}}
          r^2 \sin(\vartheta) \, \mathup{d}\varphi \, \mathup{d}\vartheta
          \, \mathup{d}r = \frac{4}{3} \mathup{\pi}
      \end{lstlisting}

      \begin{equation*}
        \int_0^1 \int_0^{\mathup{\pi}} \int_0^{2 \mathup{\pi}} r^2 \sin(\vartheta) \, \mathup{d}\phi \, \mathup{d}\vartheta \, \mathup{d}r = \frac{4}{3} \mathup{\pi}
      \end{equation*}
  \end{itemize}
\end{frame}

\begin{frame}[fragile]
    \frametitle{Standardpakete}
    Die hier aufgezählten Pakete sollten immer geladen werden, da sie wesentliche Funktionen bieten und wichtige Einstellungen vornehmen.
    \begin{columns}[T]
        \begin{column}{0.47\textwidth}
            \begin{block}{Paket}
                \begin{lstverbatim}
                \usepackage{polyglossia}
                \setmainlanguage{german}
                \usepackage{fontspec}



                \usepackage[unicode, pdfusetitle]{hyperref}
                \end{lstverbatim}
            \end{block}
        \end{column}
        \begin{column}{0.47\textwidth}
            \begin{block}{Funktion}
                Deutsche Spracheinstellungen für das Dokument. \\
                Schriftpaket. Wichtig, damit Umlaute korrekt dargestellt werden.
                Erlaubt einfaches ändern der verschiedenen Schriftarten.\\
                Für Hyperlinks im Dokument (z.B. Inhaltsverzeichnis $\rightarrow$ Kapitel).
            \end{block}
        \end{column}
    \end{columns}
\end{frame}

\begin{frame}[fragile]
    \frametitle{KOMA-Script-Klassen \hfill \doc{http://ftp.fernuni-hagen.de/ftp-dir/pub/mirrors/www.ctan.org/macros/latex/contrib/koma-script/doc/scrguide.pdf}{KOMA-Skript}}
    Stellt die wichtigen Klassen \texttt{scrartcl}, \texttt{scrreprt} und \texttt{scrbook} zur Verfügung. Sehr gute Vorlagen schon mit den Standardeinstellungen, schnell global mit Klassenoptionen anpassbar.
    \begin{block}{fürs Praktikum empfohlenene Einstellungen}
        \begin{lstverbatim}
        \documentclass[
            parskip=half,          % Absätze durch halbe Leerzeile
            captions=tableheading, % Spacing für Tabellenüberschriften
            headsepline,           % Linie zwischen Kopfzeile und Text
            titlepage=firstiscover % Titleseite ist Deckblatt
            ]{scrartcl}
        \end{lstverbatim}
    \end{block}
\end{frame}


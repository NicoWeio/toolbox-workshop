\subsection{Makefiles}

\begin{frame}[fragile]{\texttt{build}-Ordner}
  \LuaTeX\ und \texttt{biber} bieten Optionen an, um einen \texttt{build}-Ordner zu benutzen.
  \begin{block}{Aufrufe}
    \begin{lstlisting}[language=, keywordstyle={}]
      lualatex --output-directory=build file.tex
      biber build/file.tex
    \end{lstlisting}
  \end{block}

  Um Dateien aus dem \texttt{build}-Ordner zu finden (Plots, Tabellen)
  \begin{block}{Aufrufe}
    \begin{lstlisting}[language=, keywordstyle={}]
      TEXINPUTS=build:.: lualatex --output-directory=build file.tex
      BIBINPUTS=build:.  biber build/file.tex
    \end{lstlisting}
  \end{block}
\end{frame}

\begin{frame}[fragile]{\texttt{nonstopmode}}
  In Makefiles will man keine Interaktion.

  \begin{block}{Keine Interaktion}
    \begin{lstlisting}[language=, keywordstyle={}]
      lualatex --interaction=nonstopmode file.tex
    \end{lstlisting}
  \end{block}

  \begin{block}{Beim ersten Fehler abbrechen}
    \begin{lstlisting}[language=, keywordstyle={}]
      lualatex --interaction=nonstopmode --halt-on-error file.tex
    \end{lstlisting}
  \end{block}

  \begin{block}{Log schöner machen}
    \begin{lstlisting}[language=, keywordstyle={}]
     max_print_line=1048576 lualatex file.tex
    \end{lstlisting}
  \end{block}
\end{frame}

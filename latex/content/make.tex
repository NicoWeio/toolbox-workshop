\subsection{Makefiles}

\begin{frame}[fragile]{\texttt{build}-Ordner}
  \LuaTeX\ und \texttt{biber} bieten Optionen an, um einen \texttt{build}-Ordner zu benutzen.
  \begin{block}{Aufrufe}
    \begin{lstlisting}[language=, keywordstyle={}]
      lualatex --output-directory=build file.tex
      biber build/file.bcf
    \end{lstlisting}
  \end{block}

  Um Dateien aus dem \texttt{build}-Ordner zu finden (Plots, Tabellen):
  \begin{block}{Aufrufe}
    \begin{lstlisting}[language=, keywordstyle={}]
      TEXINPUTS=build: lualatex --output-directory=build file.tex
      BIBINPUTS=build: biber build/file.bcf
    \end{lstlisting}
  \end{block}
  \vspace*{-0.5em}
  \begin{itemize}
    \item \texttt{TEXINPUTS}, \texttt{BIBINPUTS}: Suchpfade für \TeX- und \texttt{.bib}-Dateien
    \item Elemente getrennt mit \texttt{:}, der erste Treffer wird genommen (wie \texttt{PATH})
      \begin{itemize}
        \item Auf Windows muss man \texttt{:} durch \texttt{;} ersetzen
      \end{itemize}
    \item \texttt{TEXINPUTS} auch für \lstinline+\includegraphics+
    \item \texttt{:} am Ende der Liste: Standardsuchpfade anhängen (wichtig!)
    \item \texttt{.} (der aktuelle Ordner) ist am Anfang der Standardliste, braucht man also nicht selbst angeben
    \item Endet ein Element mit \texttt{//}, werden auch alle Unterordner durchsucht
  \end{itemize}
\end{frame}

\begin{frame}[fragile]{\texttt{nonstopmode}}
  In Makefiles will man keine Interaktion.

  \begin{block}{Keine Interaktion}
    \begin{lstlisting}[language=, keywordstyle={}]
      lualatex --interaction=nonstopmode file.tex
    \end{lstlisting}
  \end{block}

  \begin{block}{Beim ersten Fehler abbrechen}
    \begin{lstlisting}[language=, keywordstyle={}]
      lualatex --interaction=nonstopmode --halt-on-error file.tex
    \end{lstlisting}
  \end{block}

  Neben \texttt{nonstopmode} gibt es auch \texttt{batchmode}, was die Ausgabe nur in der \texttt{.log}-Datei speichert, aber nicht ausgibt.

  \begin{block}{Log schöner machen}
    \begin{lstlisting}[language=, keywordstyle={}]
     max_print_line=1048576 lualatex file.tex
    \end{lstlisting}
  \end{block}
\end{frame}

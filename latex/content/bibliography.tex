\section{Literaturverzeichnis}

\begin{frame}[fragile]{Literaturverzeichnis}
  \begin{itemize}
    \item Wichtiger Teil vieler Dokumente, für wissenschaftliche Texte zwingend
    \item \BibLaTeX\ und \texttt{biber} bieten eine sehr angenehme Arbeitsweise
    \item Auch für sehr große Referenzdatenbanken geeignet
    \item Es gibt viele unterschiedliche Stile
    \item Standardstil fürs Praktikum geeignet
    \item Referenzen in \texttt{.bib}-Dateien
  \end{itemize}
  \begin{block}{Neue Klassenoption}
    \begin{lstlisting}
      \documentclass[…, bibliography=totoc, …]{scrartcl}
    \end{lstlisting}
  \end{block}
\end{frame}

\begin{frame}{Die \BibTeX-Familie}
  \centering
  \vspace{0.025\textheight}
  \includegraphics[height=0.95\textheight]{figures/bibtex-many.pdf}
  \begin{tikzpicture}[remember picture,overlay]
    \tikzset{shift={(current page.center)}}
    \only<2>{%
      \node (here) at (-4.3,-2.7) {%
        Sie sind hier
      };
      \draw [->, >=Stealth] (here.east) -- ++(3.3,0);
    }
  \end{tikzpicture}
\end{frame}

\begin{frame}{Warum \texttt{biber}?}
  \begin{itemize}
    \item Unterstützt Unicode-Input
    \item Wird weiterentwickelt, zusammen mit \BibLaTeX
    \item Sortiert richtig, nach regeln der jeweiligen Sprache
    \item Kann noch viele weitere Formate außer \texttt{.bib} lesen
    \item Unterstützt alle Funktionen von \BibLaTeX
  \end{itemize}
\end{frame}

\begin{frame}[fragile]{\texttt{.bib}-Dateien (I)}
  \lstinputlisting[firstline=1, lastline=5]{examples.bib}
  \vspace{1em}
  \fullcite{anleitung01}
\end{frame}

\begin{frame}[fragile]{\texttt{.bib}-Dateien (II)}
  \lstinputlisting[firstline=7, lastline=18]{examples.bib}
  \vspace{1em}
  \fullcite{numpy}
\end{frame}

\begin{frame}[fragile]{\texttt{.bib}-Dateien (III)}
  \lstinputlisting[firstline=20, lastline=30]{examples.bib}
  \vspace{1em}
  \fullcite{root}
\end{frame}

\begin{frame}[fragile]{\texttt{.bib}-Dateien (IV)}
  \lstinputlisting[firstline=32, lastline=39]{examples.bib}
  \vspace{1em}
  \fullcite{wingate}
\end{frame}

\begin{frame}[fragile]{\texttt{.bib}-Dateien (V)}
  \lstinputlisting[firstline=41, lastline=48]{examples.bib}
  \vspace{1em}
  \fullcite{hastie}
\end{frame}


\begin{frame}[fragile]{\texttt{.bib}-Dateien (VI)}
  \lstinputlisting[firstline=50, lastline=53]{examples.bib}
  \vspace{1em}
  \fullcite{curvefit}
\end{frame}

\begin{frame}[fragile]{%
  \BibLaTeX{}%
  \hfill%
  \doc{http://mirrors.ctan.org/macros/latex/contrib/biblatex/doc/biblatex.pdf}{biblatex}
}
  \begin{Packages}
    \begin{lstlisting}
      \usepackage{biblatex} % nach babel
      \addbibresource{lit.bib}
    \end{lstlisting}
  \end{Packages}
  \begin{CodeExample}{0.60}[Zitieren]
    \begin{lstlisting}
      \cite{numpy}
      \cite[20]{numpy}
      \cite[1--3]{numpy}
      \cite{hastie, root}
    \end{lstlisting}
  \CodeResult
    \cite{numpy} \\
    \cite[20]{numpy} \\
    \cite[1--3]{numpy} \\
    \cite{hastie, root}
  \end{CodeExample}
  \begin{block}{Verzeichnis ausgeben}
    \begin{lstlisting}
      \nocite{wingate}   % ins Verzeichnis, obwohl nicht explizit zitiert
      \nocite{*}         % alles aus .bib ins Verzeichnis
      \printbibliography
    \end{lstlisting}
  \end{block}
\end{frame}

\begin{frame}{Literaturverzeichnis}
  \centering
  \pause
  \Huge ???
\end{frame}

\begin{frame}[fragile]{
  \texttt{biber}
  \hfill
  \doc{http://mirrors.ctan.org/biblio/biber/documentation/biber.pdf}{biber}
}
  Die Idee ist:
  \begin{enumerate}
    \item \BibLaTeX\ erstellt eine Liste der \texttt{.bib}-Dateien und der benötigten Referenzen \\
      → \texttt{.bcf}-Datei
    \item \texttt{biber} liest Anweisungen, liest \texttt{.bib}, sucht und sortiert Referenzen \\
      → \texttt{.bbl}-Datei
    \item \BibLaTeX\ liest \texttt{.bbl}, gibt Verzeichnis aus
  \end{enumerate}

  \vspace{10pt}
  Also:
  \begin{block}{Aufrufe mit Literaturverzeichnis}
    \begin{lstlisting}
       lualatex file.tex
       biber file.bcf
       lualatex file.tex
    \end{lstlisting}
  \end{block}
\end{frame}

\begin{frame}{Literaturverzeichnis}
  \nocite{*}
  \printbibliography[heading=none]
\end{frame}

\begin{frame}[fragile]{Stile}
  \begin{itemize}
    \item Standardstil ist \enquote{numeric}
    \item Häufig genutzte Alternative: \enquote{alphabetic}
    \item Kombination aus Autorenname und Jahr: z.B. [Oli07]
    \item Viele weitere Stile → Doku
    \item Setzen mit \texttt{style=…} als Option für \texttt{biblatex}
  \end{itemize}
  \begin{block}{Code}
    \begin{lstlisting}
      \usepackage[style=alphabetic]{biblatex}
    \end{lstlisting}
  \end{block}
\end{frame}

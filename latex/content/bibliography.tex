\section{Literaturverzeichnis}

\begin{frame}{Literaturverzeichnis}
  \begin{itemize}
    \item wichtiger Teil vieler Dokumente, für wissenschaftliche Texte zwingend
    \item Bib\LaTeX\ und biber bieten eine sehr angenehme Arbeitsweise
    \item auch für sehr große Referenzdatenbanken geeignet
    \item es gibt viele unterschiedliche Stile
    \item Standardstil fürs Praktikum geeignet
    \item Referenzen in .bib-Dateien
  \end{itemize}
\end{frame}

\begin{frame}
  \centering
  \includegraphics[height=0.95\textheight]{figures/bibtex-many.pdf}
  \begin{tikzpicture}[remember picture,overlay]
    \tikzset{shift={(current page.center)}}
    \node at (-4.3,3) {
      \textsc{Bib}\TeX-Familie
    };
    \only<2>{
      \node (here) at (-4.3,-2.8) {
        Sie sind hier
      };
      \draw [->, >=Stealth] (here.east) -- ++(3.5,0);
    }
  \end{tikzpicture}
\end{frame}

\begin{frame}{Warum biber?}
  \begin{itemize}
    \item unterstützt Unicode-Input
    \item wird weiterentwickelt, zusammen mit Bib\LaTeX
    \item sortiert richtig, nach regeln der jeweiligen Sprache
    \item kann noch viele weitere Formate außer .bib lesen
    \item unterstützt alle Funktionen von Bib\LaTeX
  \end{itemize}
\end{frame}

\defverbatim{\bibexamplei}{%
\begin{verbatim}
@manual{anleitung01,
  author = "TU Dortmund",    % alternativ ist {…} statt "…" möglich
  title = "Versuchsanleitung zu Versuch Nr. 01
           Lebensdauer der Myonen",
  year = 2004
}
\end{verbatim}
}

\begin{frame}{.bib-Dateien (I)}
  \bibexamplei
  \fullcite{anleitung01}
\end{frame}

\defverbatim{\bibexampleii}{%
\begin{verbatim}
@article{numpy,
  author = "Travis E. Oliphant",
  title = "Python for Scientific Computing",
  publisher = "IEEE",
  year = "2007",
  journal = "Computing in Science \& Engineering",
  volume = "9",
  number = "3",
  pages = "10–20",
  url = "http://link.aip.org/link/?CSX/9/10/1",
  addendum = "Version 1.8.1"
}
\end{verbatim}
}

\begin{frame}{.bib-Dateien (II)}
  \bibexampleii
  \fullcite{numpy}
\end{frame}

\defverbatim{\bibexampleiii}{%
\begin{verbatim}
@inproceedings{root,
  author = "Brun, Rene and Rademakers, Fons",
  booktitle = "AIHENP'96 Workshop, Lausanne",
  url = "http://root.cern.ch/",
  journal = "Nucl. Inst. \& Meth. in Phys. Res. A",
  pages = "81–86",
  title = "ROOT — An Object Oriented Data Analysis Framework",
  volume = 389,
  year = 1996,
  addendum = "Version 5.34.18"
}
\end{verbatim}
}

\begin{frame}{.bib-Dateien (III)}
  \bibexampleiii
  \fullcite{root}
\end{frame}

\defverbatim{\bibexampleiv}{%
\begin{verbatim}
@online{splot,
  author = "Muriel Pivk and Francois R. Le Diberder",
  title = "sPlot: a statistical tool to unfold data distributions",
  date = "2005-09-02",
  archivePrefix = "arXiv",
  eprint = "physics/0402083v3"
}
\end{verbatim}
}

\begin{frame}{.bib-Dateien (IV)}
  \bibexampleiv
  \fullcite{splot}
\end{frame}

\defverbatim{\bibexamplev}{%
\begin{verbatim}
@online{wingate,
  author = "Zhaofeng Liu and Stefan Meinel and Alistair Hart and
            Ron R. Horgan and Eike H. Müller and Matthew Wingate",
  title = "A lattice calculation of \HepProcess{\PB \to \PK^{(*)}}
           form factors",
  date = "2011-01-14",
  archivePrefix = "arXiv",
  eprint = "1101.2726v1",
  primaryClass = "hep-ph"
}
\end{verbatim}
}

\begin{frame}{.bib-Dateien (V)}
  \bibexamplev
  \fullcite{wingate}
\end{frame}

\begin{frame}[fragile]{Bib\LaTeX}
  \begin{block}{Paket einbinden}
    \begin{lstverbatim}
    \usepackage[backend=biber]{biblatex}
    \addbibresource{lit.bib}
    \end{lstverbatim}
  \end{block}
  \begin{block}{Zitieren}
      \begin{minipage}{0.6\linewidth}
        \begin{lstverbatim}
        \cite{numpy}
        \cite{splot,root}
        \end{lstverbatim}
      \end{minipage}
      \begin{minipage}{0.35\linewidth}
        \cite{numpy}\\
        \cite{splot,root}
      \end{minipage}
  \end{block}
  \begin{block}{Verzeichnis ausgeben}
    \begin{lstverbatim}
    \nocite{anleitung01} % ins Verzeichnis, obwohl nicht explizit zitiert
    \nocite{*}           % alles aus .bib ins Verzeichnis
    \printbibliography
    \end{lstverbatim}
  \end{block}
\end{frame}

\begin{frame}{Literaturverzeichnis}
  \centering
  \pause
  \Huge ???
\end{frame}

\begin{frame}[fragile]{biber}
  Die Idee ist:
  \begin{enumerate}
    \item Bib\LaTeX\ erstellt eine Liste der .bib-Dateien und der benötigten Referenzen\\
          → .bcf-Datei
    \item biber liest Anweisungen, liest .bib, sucht und sortiert Referenzen\\
          → .bbl-Datei
    \item Bib\LaTeX\ liest .bbl, gibt Verzeichnis aus
  \end{enumerate}

  \vspace{10pt}
  Also:
  \begin{block}{Aufrufe mit Literaturverzeichnis}
    \begin{lstverbatim}
    $ lualatex file.tex
    $ biber file.bcf
    $ lualatex file.tex
    \end{lstverbatim}
  \end{block}
\end{frame}

\begin{frame}{Literaturverzeichnis}
  \nocite{*}
  \printbibliography
\end{frame}

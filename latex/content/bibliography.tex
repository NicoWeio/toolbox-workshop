\section{Literaturverzeichnis}

\begin{frame}[fragile]{Literaturverzeichnis}
  \begin{itemize}
    \item Wichtiger Teil vieler Dokumente, für wissenschaftliche Texte zwingend
    \item Bib\LaTeX\ und \texttt{biber} bieten eine sehr angenehme Arbeitsweise
    \item Auch für sehr große Referenzdatenbanken geeignet
    \item Es gibt viele unterschiedliche Stile
    \item Standardstil fürs Praktikum geeignet
    \item Referenzen in \texttt{.bib}-Dateien
  \end{itemize}
  \begin{block}{Neue Klassenoption}
    \begin{lstlisting}
      \documentclass[…, bibliography=totoc, …]{scrartcl}
    \end{lstlisting}
  \end{block}
\end{frame}

\begin{frame}
  \centering
  \includegraphics[height=0.95\textheight]{figures/bibtex-many.pdf}
  \begin{tikzpicture}[remember picture,overlay]
    \tikzset{shift={(current page.center)}}
    \node at (-4.3,3) {
      \textrm{\textsc{Bib}\TeX}-Familie
    };
    \only<2>{
      \node (here) at (-4.3,-2.8) {
        Sie sind hier
      };
      \draw [->, >=Stealth] (here.east) -- ++(3.5,0);
    }
  \end{tikzpicture}
\end{frame}

\begin{frame}{Warum \texttt{biber}?}
  \begin{itemize}
    \item Unterstützt Unicode-Input
    \item Wird weiterentwickelt, zusammen mit Bib\LaTeX
    \item Sortiert richtig, nach regeln der jeweiligen Sprache
    \item Kann noch viele weitere Formate außer \texttt{.bib} lesen
    \item Unterstützt alle Funktionen von Bib\LaTeX
  \end{itemize}
\end{frame}

\begin{frame}[fragile]{\texttt{.bib}-Dateien (I)}
  \begin{lstlisting}[language=, keywordstyle={}]
    @manual{anleitung01,
      author = "TU Dortmund", % alternativ ist {...} statt "..." möglich
      title = "Versuchsanleitung zu Versuch Nr. 01
               Lebensdauer der Myonen",
      year = 2004
    }
  \end{lstlisting}
  \vspace{1em}
  \fullcite{anleitung01}
\end{frame}

\begin{frame}[fragile]{\texttt{.bib}-Dateien (II)}
  \begin{lstlisting}[language=, keywordstyle={}]
    @article{numpy,
      author = "Travis E. Oliphant",
      title = "Python for Scientific Computing",
      publisher = "IEEE",
      year = "2007",
      journal = "Computing in Science \& Engineering",
      volume = "9",
      number = "3",
      pages = "10–20",
      url = "http://link.aip.org/link/?CSX/9/10/1",
      addendum = "Version 1.8.1"
    }
  \end{lstlisting}
  \vspace{1em}
  \fullcite{numpy}
\end{frame}

\begin{frame}[fragile]{\texttt{.bib}-Dateien (III)}
  \begin{lstlisting}[language=, keywordstyle={}]
    @inproceedings{root,
      author = "Brun, Rene and Rademakers, Fons",
      booktitle = "AIHENP'96 Workshop, Lausanne",
      url = "http://root.cern.ch/",
      journal = "Nucl. Inst. \& Meth. in Phys. Res. A",
      pages = "81–86",
      title = "ROOT — An Object Oriented Data Analysis Framework",
      volume = 389,
      year = 1996,
      addendum = "Version 5.34.18"
    }
  \end{lstlisting}
  \vspace{1em}
  \fullcite{root}
\end{frame}

\begin{frame}[fragile]{\texttt{.bib}-Dateien (IV)}
  \begin{lstlisting}[language=, keywordstyle={}]
    @online{splot,
      author = "Muriel Pivk and Francois R. Le Diberder",
      title = "sPlot: a statistical tool to unfold data distributions",
      date = "2005-09-02",
      archivePrefix = "arXiv",
      eprint = "physics/0402083v3"
    }
  \end{lstlisting}
  \vspace{1em}
  \fullcite{splot}
\end{frame}

\begin{frame}[fragile]{\texttt{.bib}-Dateien (V)}
  \begin{lstlisting}[language=, keywordstyle={}]
    @online{wingate,
      author = "Zhaofeng Liu and Stefan Meinel and Alistair Hart and
                Ron R. Horgan and Eike H. Müller and Matthew Wingate",
      title = "A lattice calculation of {$\symup{B} \to
               \symup{K}^{(*)}$} form factors",
      date = "2011-01-14",
      archivePrefix = "arXiv",
      eprint = "1101.2726v1",
      primaryClass = "hep-ph"
    }
  \end{lstlisting}
  \vspace{1em}
  \fullcite{wingate}
\end{frame}

\begin{frame}[fragile]{Bib\LaTeX\hfill\doc{http://mirrors.ctan.org/macros/latex/contrib/biblatex/doc/biblatex.pdf}{biblatex}}
  \begin{Packages}
    \begin{lstlisting}
      \usepackage{biblatex} % nach polyglossia
      \addbibresource{lit.bib}
    \end{lstlisting}
  \end{Packages}
  \begin{CodeExample}{0.60}[Zitieren]
    \begin{lstlisting}
      \cite{numpy}
      \cite[20]{numpy}
      \cite[1--3]{numpy}
      \cite{splot, root}
    \end{lstlisting}
  \CodeResult
    \cite{numpy} \\
    \cite[20]{numpy} \\
    \cite[1--3]{numpy} \\
    \cite{splot, root}
  \end{CodeExample}
  \begin{block}{Verzeichnis ausgeben}
    \begin{lstlisting}
      \nocite{wingate}   % ins Verzeichnis, obwohl nicht explizit zitiert
      \nocite{*}         % alles aus .bib ins Verzeichnis
      \printbibliography
    \end{lstlisting}
  \end{block}
\end{frame}

\begin{frame}{Literaturverzeichnis}
  \centering
  \pause
  \Huge ???
\end{frame}

\begin{frame}[fragile]{
  \texttt{biber} \hfill
  \doc{http://mirrors.ctan.org/biblio/biber/documentation/biber.pdf}{biber}
}
  Die Idee ist:
  \begin{enumerate}
    \item Bib\LaTeX\ erstellt eine Liste der \texttt{.bib}-Dateien und der benötigten Referenzen \\
      → \texttt{.bcf}-Datei
    \item \texttt{biber} liest Anweisungen, liest \texttt{.bib}, sucht und sortiert Referenzen \\
      → \texttt{.bbl}-Datei
    \item Bib\LaTeX\ liest \texttt{.bbl}, gibt Verzeichnis aus
  \end{enumerate}

  \vspace{10pt}
  Also:
  \begin{block}{Aufrufe mit Literaturverzeichnis}
    \begin{lstlisting}
       lualatex file.tex
       biber file.bcf
       lualatex file.tex
    \end{lstlisting}
  \end{block}
\end{frame}

\begin{frame}{Literaturverzeichnis}
  \nocite{*}
  \printbibliography[heading=none]
\end{frame}

\begin{frame}[fragile]{Stile}
  \begin{itemize}
    \item Standardstil ist \enquote{numeric}
    \item Häufig genutzte Alternative: \enquote{alphabetic}
    \item Kombination aus Autorenname und Jahr: z.B. [Oli07]
    \item Viele weitere Stile → Doku
    \item Setzen mit \texttt{style=…} als Option für \texttt{biblatex}
  \end{itemize}
  \begin{block}{Code}
    \begin{lstlisting}
      \usepackage[style=alphabetic]{biblatex}
    \end{lstlisting}
  \end{block}
\end{frame}

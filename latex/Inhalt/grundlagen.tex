\section{Grundlagen}
\begin{frame}
    \frametitle{Was ist \LaTeX?}
    \begin{itemize}
        \item \emph{Programmiersprache} zum Setzen von Text
        \item Kein WYSIWYG, es werden Befehle und Inhalt in normale Text-Dateien geschrieben.
        \item Kompiler überträgt \LaTeX-Code in ein Ausgabedokument (meist PDF)
        \item OpenSource mit zahlreichen Erweiterungsmöglichkeit (Pakete)
    \end{itemize}
\end{frame}
\begin{frame}
    \frametitle{Warum \LaTeX \ benutzen?}
    \begin{itemize}
        \item hervorragender Text- und Formelsatz
        \item automatisierte Erstellung von Inhalts- und Literaturverzeichnis
        \item Tex-Dateien sind reine Text-Dateien \\
              $\Rightarrow$ kleine Datein, gut für Versionskontrolle geeignet
        \item sehr gute Vorlagen für wissenschaftliche Arbeiten 
        \item aber auch: Notensatz, Präsentationen 
        \item ausgezeichnete Dokumentionen
        \item erweiterbar durch zahlreiche und mächtige Pakete
        \item aus allen geläufigen Betriebssystem verfügbar
        \item Ausgabe nach PDF
    \end{itemize}
\end{frame}

\begin{frame}[fragile]
    \frametitle{Das Dokument}
    Diese drei Zeilen braucht jedes \LaTeX-Dokument:
    \begin{columns}[T]
        \begin{column}{0.4\textwidth}
            \begin{block}{\LaTeX-Code}
                \begin{lstverbatim}
                \documentclass{scrartcl}
                    % Präambel
                    .
                    .
                \begin{document}
                    % Inhalt des Dokuments
                    .
                    .
                \end{document}
                \end{lstverbatim}
            \end{block}
        \end{column}
        \begin{column}{0.55\textwidth}
            \begin{block}{documentclass}
                Hier wird die Dokumentenklasse definiert. \\
                Es folgt die \alert{Präambel} des Dokuments. 
                Hier werden globale Optionen gesetzt und zusätzliche Pakete eingebunden.\\
            \end{block}

            \vspace{0.55cm}

            \begin{block}{document-Umgebung}
                Hier wird das eigentliche Dokument erstellt.
            \end{block}
        \end{column}
    \end{columns}
\end{frame}

\begin{frame}[fragile]
    \frametitle{Syntax: Befehle}
    \LaTeX-Befehle beginnen stets mit einem Back-Slash.

    Obligatorische Argumente stehen in \{ \}, optionale Argumente stehen in [ ].
    \begin{columns}[T]
        \begin{column}{0.47\textwidth}
            \begin{block}{\LaTeX-Code}
                \begin{lstverbatim}
                \befehl[optional]{obligatorisch}
                \documentclass[paper=a4]
                              {scrartcl}
                \frac{1}{2}


                % Kommentar
                \end{lstverbatim}
                \vspace{0.4cm}
            \end{block}
        \end{column}
        \begin{column}{0.47\textwidth}
            \begin{block}{Erklärung}
                Beispiel \\
                Setzt die Dokumentenklasse auf \emph{scrartcl} und das Papierformat auf DIN\,A4. \\
                Es gibt auch Befehle mit zwei oder mehr Pflichtargumenten, z.B. der Bruch. \\
                Text oder Befehle nach einem \%-Zeichen werden nicht berücksichtigt.
            \end{block}
        \end{column}
    \end{columns}
\end{frame}

\begin{frame}[fragile]
    \frametitle{Das Ausgabedokument erstellen}
    Es gibt mehrere verschiedene \LaTeX-Kompiler, die verschiedene Ausgabeformate erzeugen können.
    Der modernste Kompiler, der PDF-Dateien erstellt, ist \alert{lualatex}.

    \begin{block}{\LaTeX-Dakument kompilieren}
        Konsole öffnen:
        \begin{lstverbatim}
        $ lualatex MeinDokument.tex
        \end{lstverbatim}
    \end{block}
    \begin{alertblock}{Vorsicht!}
        \begin{itemize}
        \item Es muss fast immer mindestens zweimal kompiliert werden.
        \item Es werden diverse Hilfs- und Logdateien erzeugt.
        \end{itemize}
    \end{alertblock}
\end{frame}

\PassOptionsToPackage{unicode}{hyperref}
\PassOptionsToPackage{aux}{rerunfilecheck}
\documentclass[professionalfonts]{beamer}

\usepackage{amsmath}
\usepackage{amssymb}
\usepackage{mathtools}

\usepackage{fontspec}
\usepackage{polyglossia}
\setmainlanguage{german}
\usepackage[
  math-style=ISO,
  bold-style=ISO,
  sans-style=italic,
  nabla=upright,
  partial=upright,
]{unicode-math}

\usepackage[
  math-micro=µ,
  locale=DE,
]{siunitx}

\usetheme{Frankfurt}
\usecolortheme{seagull}
\setbeamertemplate{navigation symbols}{}

\author{PeP et al.\ e.\,V.}
\institute{TU Dortmund}
\date{Toolbox-Workshop 2015}
\title{Beamer-Musterlösung}

\begin{document}

\begin{frame}
  \maketitle
\end{frame}
\begin{frame}{Übersicht}
  \maketitle
\end{frame}

\section{Toller Plot}
\begin{frame}{Plot}
  \begin{columns}[c, onlytextwidth]
    \begin{column}{0.47\textwidth}
      \includegraphics[width=\textwidth]{example-image-a}
    \end{column}
    \begin{column}{0.47\textwidth}
      \begin{itemize}
        \item Dies ist ein Beispielbild
        \item Es kommt mit dem Paket \texttt{mwe}
        \item Gut zum testen
      \end{itemize}
    \end{column}
  \end{columns} 
\end{frame}
\section{Maxwell-Gleichungen}
\begin{frame}{Maxwell-Gleichungen}
  \begin{columns}[c, onlytextwidth]
    \begin{column}{0.45\textwidth}
      \begin{block}{Maxwell I}
        \begin{equation}
          \nabla \cdot \symbf{E} = \frac{ρ}{ε_0}
        \end{equation}
      \end{block}
      \begin{block}{Maxwell III}
        \begin{equation}
          \nabla \times \symbf{E} = - \partial_t \symbf{B}
        \end{equation}
      \end{block}
    \end{column}
    \begin{column}{0.5\textwidth}
      \begin{block}{Maxwell II}
        \begin{equation}
          \nabla \cdot \symbf{B} = 0
        \end{equation}
      \end{block}
      \begin{block}{Maxwell IV}
        \begin{equation}
          \nabla \times \symbf{B} = μ_0 \symbf{j} - μ_0 ε_0 \partial_t \symbf{E}
        \end{equation}
      \end{block}
    \end{column}
  \end{columns} 
\end{frame}
\end{document}

\documentclass{scrartcl}

\usepackage[aux]{rerunfilecheck}


\usepackage{polyglossia}
\setmainlanguage{german}

\usepackage{fontspec}

%%%%%%%%%%%%%%%%%%%%%%%%%%%%%%%%%%%%%%%%%%%%%%%%%%%%%%%
%%%%%%%%%%%%%%%%%%%%    Mathe-Pakete    %%%%%%%%%%%%%%%
%%%%%%%%%%%%%%%%%%%%%%%%%%%%%%%%%%%%%%%%%%%%%%%%%%%%%%%

\usepackage{amsmath}
\usepackage{amssymb}
\usepackage{mathtools}

\usepackage[
  math-style=ISO,
  bold-style=ISO,
  sans-style=italic,
  nabla=upright,
  partial=upright,
  ]{unicode-math}

\usepackage[unicode]{hyperref}
\usepackage{bookmark}

\begin{document}

\section{Biot-Savart}
Das Magnetfeld $\vec{B}$ am Ort $\vec{r}$ eines stromdurchflossenen Leiters ergbibt sich zu
\begin{equation}
  \vec{B}(\vec{r}) = \frac{\mu_0}{4\mathup{\pi}}
  \int_V  \vec{\jmath}(r') \times \frac{\vec{r}-\vec{r}'}{|\vec{r}-\vec{r}'|^3} \, \mathup{d}V' \quad .
\end{equation}
Hierbei bezeichnet $\vec{\jmath}$ die Stromdichte am Ort $r'$, $\mu_0$ ist die Magnetische Feldkonstante.

\section{Fehlerfortpflanunzung}

\begin{equation}
\sigma_f = \sqrt{\sum\limits_{i=1}^N \left( \frac{\partial f}{\partial x_i}
                 \sigma_i \right)^2}
\end{equation}



\section{Die vier Maxwell-Gleichungen}
\begin{align}
  \nabla \cdot \vec{E} &= \frac{\rho}{\epsilon_0} &
  \nabla \cdot \vec{B} &= 0 \\
  \nabla \times \vec{E} &= - \partial_t \vec{B} &
  \nabla \times \vec{B} &= \mu_0 \vec{\jmath} + \mu_0 \epsilon_0 \partial_t \vec{E} 
\end{align}
\end{document}



\documentclass[captions=tableheading]{scrartcl}

\usepackage{fixltx2e}
\usepackage[aux]{rerunfilecheck}

\usepackage{polyglossia}
\setmainlanguage{german}
\usepackage[backend=biber, style=alphabetic]{biblatex}
\addbibresource {loesung.bib}

\usepackage{fontspec}

\usepackage{float}
\usepackage[section, below]{placeins}

\usepackage[unicode]{hyperref}
\usepackage{bookmark}

\begin{document}
  \section{Theorie}
    Um den Phasenübergang zu beschreiben schlägt deGennes den Ordnungsparameter $a$ vor \cite[2-8]{deGennes}.
  \section{Aufbau und Durchführung}
    Ein übersichtliches Schema für die experimentelle Realisierung kann den Quellen \cite{martin, magnet} entnommen werden.
  \section{Diskussion}
    Eine Übersicht über weitere Verfahren befindet sich in \cite{kent}.
  \printbibliography
\end{document}

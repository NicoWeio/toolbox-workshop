\begin{frame}{Automatisierte, reproduzierbare Prozesse}

  {\huge Problem:}
  \vspace{1em}
  \begin{itemize}
    \item kurz vor Abgabe noch neue Korrekturen einpflegen
      \begin{itemize}
        \item Tippfehler korrigieren, Plots bearbeiten
      \end{itemize}
    \item \TeX{} ausführen, ausdrucken
    \item vergessen, Plots neu zu erstellen
  \end{itemize}
\end{frame}

\begin{frame}{Automatisierte, reproduzierbare Prozesse}

  {\huge Lösung: Make!}
  \vspace{1em}
  \begin{itemize}
    \item sieht, welche Dateien geändert wurden
    \item berechnet nötige Operationen
    \item führt Python-Skript aus, führt \TeX{} aus
  \end{itemize}
\end{frame}

\begin{frame}{Makefile}
  \begin{itemize}
    \item Datei heißt \texttt{Makefile}, keine Endung!
      \begin{itemize}
        \item bei Windows Dateiendungen einschalten! (\url{http://support.microsoft.com/kb/865219/de})
      \end{itemize}
    \item besteht aus Rules:
  \end{itemize}
  \begin{block}{Rule}
    \texttt{target: prerequisites\\
    \hspace{1cm} recipe}
  \end{block}
  \begin{description}
    \item[\texttt{\hphantom{prerequisites}\llap{target}}] Datei(en), die von dieser Rule erzeugt wird
    \item[\texttt{prerequisites}]                         Dateien, von denen diese Rule abhängt
    \item[\texttt{\hphantom{prerequisites}\llap{recipe}}] Befehle: \texttt{prerequisites} → \texttt{target} (mit Tab eingerückt)
  \end{description}
\end{frame}

\begin{frame}[fragile]{Beispiel}
  \begin{verbatim}
    all: report.pdf

    plot.pdf: plot.py data.txt
        python plot.py

    report.pdf: report.tex
        lualatex report.tex

    report.pdf: plot.pdf
  \end{verbatim}

  \begin{itemize}
    \item wenn nur \texttt{make} gestartet wird, wird der erste Target erstellt, hier \texttt{all}; um ein anderes Target zu erstellen, startet man \texttt{make \textit{target}}
      \begin{itemize}
        \item nützlich, wenn man Plot bearbeitet: \texttt{make plot.pdf}
      \end{itemize}
    \item \texttt{all} braucht \texttt{report.pdf}, also wird \texttt{latex report.tex} ausgeführt
    \item \texttt{report.pdf} braucht noch \texttt{plot.pdf}, also wird Python ausgeführt
  \end{itemize}
\end{frame}

\begin{frame}{Funktionsweise}
  \begin{center}
    \begin{tikzpicture}
      \graph [grow down, branch right sep] {
        "all" -> "report.pdf" -> {
          "report.tex",
          "plot.pdf" -> {
            "plot.py", "data.txt"
          },
          "plot2.pdf" -> {
            "plot2.py", "data.txt", "data2.txt"
          },
        },
      };
    \end{tikzpicture}
  \end{center}

  \begin{itemize}
    \item Abhängigkeiten bilden einen DAG (directed acyclic graph)
    \item Dateien werden neu erstellt, falls sie nicht existieren oder älter als ihre Prerequisites sind
    \item Prerequisites werden zuerst erstellt
    \item top-down Vorgehen
  \end{itemize}
\end{frame}

\begin{frame}[fragile]{Advanced}
  \begin{itemize}
    \item \texttt{make clean}
      \begin{verbatim}
    clean:
        # alles löschen, was vom Makefile erzeugt wird
        rm plot.pdf report.pdf
      \end{verbatim}
    \item \texttt{build}-Ordner: Ordner sauber halten
      \begin{verbatim}
    build/plot.pdf: plot.py data.txt | build
        python plot.py # savefig('build/plot.pdf')

    build/report.pdf: report.tex build/plot.pdf | build
        lualatex --output-directory=build report.tex

    build:
        mkdir -p build
    clean:
        rm -rf build
      \end{verbatim}
    \item \texttt{| build} ist ein order-only Prerequisite: Alter wird ignoriert
  \end{itemize}
\end{frame}

\begin{frame}[fragile]{Expert}
  Mehrere unabhängige Auswertungen: könnte man sie parallel ausführen? \\
  Ja! \texttt{make -j4} (4 Prozesse gleichzeitig)

  \begin{verbatim}
    plot1.pdf plot2.pdf: plot.py data.txt
        python plot.py
  \end{verbatim}

  Problem: \texttt{make} führt \texttt{plot.py} gleichzeitig zweimal aus

  Lösung: manuell synchronisieren

  \begin{verbatim}
    plot1.pdf plot2.pdf: plot.sync

    plot.sync: plot.py data.txt
        python plot.py
        touch plot.sync
  \end{verbatim}
\end{frame}

\begin{frame}{Motivation}
  \begin{itemize}
    \item Automatisierung verhindert Fehler
    \item Dient als Dokumentation
    \item Reproduzierbarkeit unverzichtbar in der Wissenschaft
    \item Idealfall: Eingabe von \texttt{make} erstellt komplettes Protokoll/Paper aus Daten
    \item Spart Zeit, da nur nötige Operationen ausgeführt werden
  \end{itemize}
\end{frame}

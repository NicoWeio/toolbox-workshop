\begin{frame}{Automatisierte, reproduzierbare Prozesse}

  {\huge Problem:}
  \vspace{1em}
  \begin{itemize}
    \item kurz vor Abgabe noch neue Korrekturen einpflegen
      \begin{itemize}
        \item Tippfehler korrigieren, Plots bearbeiten
      \end{itemize}
    \item \TeX{} ausführen, ausdrucken
    \item vergessen, Plots neu zu erstellen
  \end{itemize}
\end{frame}

\begin{frame}{Automatisierte, reproduzierbare Prozesse}

  {\huge Lösung: Make!}
  \vspace{1em}
  \begin{itemize}
    \item sieht, welche Dateien geändert wurden
    \item berechnet nötige Operationen
    \item führt Python-Skript aus, führt \TeX{} aus
  \end{itemize}
\end{frame}

\begin{frame}{Makefile}
  \begin{itemize}
    \item Datei heißt \texttt{Makefile}, keine Endung!
      \begin{itemize}
        \item bei Windows Dateiendungen einschalten! (\url{http://support.microsoft.com/kb/865219/de})
      \end{itemize}
    \item besteht aus Rules:
  \end{itemize}
  \begin{block}{Rule}
    \texttt{target: prerequisites\\
    \hspace{1cm} recipe}
  \end{block}
  \begin{description}
    \item[\texttt{target}] Datei(en), die von dieser Rule erzeugt wird
    \item[\texttt{prerequisites}] Dateien, von denen diese Rule abhängt
    \item[\texttt{recipe}] Befehle: \texttt{prerequisites} → \texttt{target} (mit Tab eingerückt)
  \end{description}
\end{frame}

\begin{frame}[fragile]{Beispiel}
  \begin{verbatim}
    all: report.pdf

    plot.pdf: plot.py daten.txt
        python plot.py

    report.pdf: report.tex plot.pdf
        lualatex report.tex
  \end{verbatim}

  \begin{itemize}
    \item wenn nur \texttt{make} gestartet wird, wird der erste Target erstellt, hier \texttt{all}; um ein anderes Target zu erstellen, startet man \texttt{make \textit{target}}
    \item \texttt{all} braucht \texttt{report.pdf}, also wird \texttt{latex report.tex} ausgeführt
    \item \texttt{report.pdf} braucht noch \texttt{plot.pdf}, also wird Python ausgeführt
  \end{itemize}
\end{frame}

\begin{frame}{Motivation}
  \begin{itemize}
    \item Automatisierung verhindert Fehler
    \item Dient als Dokumentation
    \item Reproduzierbarkeit unverzichtbar in der Wissenschaft
    \item Idealfall: Eingabe von \texttt{make} erstellt komplettes Protokoll/Paper aus Daten
    \item Spart Zeit, da nur nötige Operationen ausgeführt werden
  \end{itemize}
\end{frame}

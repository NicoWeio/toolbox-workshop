\usepackage{fixltx2e}			        % LaTeX2e korrigieren.
\usepackage[aux]{rerunfilecheck}	% Warnung, falls nochmal kompiliert werden muss.

\usepackage{polyglossia}		% Deutsche Spracheinstellungen.
\setmainlanguage{german}

\usepackage{amsmath}			% Unverzichtbare Mathe-Befehle.
\usepackage{amssymb}			% Viele Mathe-Symbole.
\usepackage{mathtools}			% Erweiterungen für amsmath.

\usepackage{fontspec}			% Für Fonteinstellungen.

\usepackage[
	math-style=ISO,			% \
	bold-style=ISO,			% |
	sans-style=italic,		% | ISO-Standard folgen
	nabla=upright,			% |
	partial=upright,		% /
]{unicode-math}				% "Does exactly what it says on the tin."

\usepackage[autostyle]{csquotes}	% Sorgt für die richtigen Anführungszeichen.

\usepackage[
	locale=DE,			% LaTeX den Länderstandard mitteilen.
	separate-uncertainty=true,	% Immer Fehler mit \pm.
	per-mode=symbol-or-fraction,	% m/s im Text, sonst Brüche.
]{siunitx}

\usepackage[version=3]{mhchem}		% Für chemische Formeln

\usepackage[section, below]{placeins}	% Floats innerhalb einer Section halten.
\usepackage[
  labelfont=bf,        % Tabelle x: Abbildung y: ist jetzt fett
  font=small,          % Schrift etwas kleiner als Dokument
  width=0.9\textwidth,  % Maximale Breite einer Caption schmaler
]{caption}			% Captions schöner machen.
\usepackage{subcaption}			% subfigure, subtable, subref

\usepackage{graphicx}			% Grafiken können eingebunden werden.
\usepackage{grffile}			% Größere Variation von Dateinamen möglich.

\usepackage{float}			        % Standardplatzierung für Floats einstellen
\floatplacement{figure}{htbp}		% Grafiken schön setzen.
\floatplacement{table}{htbp}		% Tabellen schön setzen.

\usepackage{booktabs}			% für schöne Tabellen

\usepackage{biblatex}			% Für ordentliche Quellenverzeichnisse.
\addbibresource{res/lit.bib}		% Datei zur Literaturverwaltung.

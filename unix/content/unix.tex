\tabulinesep=4pt

{
  \setbeamercolor{background canvas}{bg=black}
  \color{white}

  \begin{frame}[fragile,t,plain]
    \verb+[ismo@it ~]$ _+
\end{frame}
}

\begin{frame}
  \centering
  \Huge
  Was ist das? \\[1em]

  Muss das sein? \\[1em]

  Ist das nicht völlig veraltet? \\[1em]

  Das sieht nicht so schick aus…
\end{frame}

\begin{frame}{Motivation}
  \begin{itemize}
    \item Die meisten Geräte basieren auf Unix
      \begin{itemize}
        \item Server, Cluster, Supercomputer
        \item Smartphones
        \item Router, Drucker, …
      \end{itemize}
    \item Wissenschaftliche Programme werden für Unix geschrieben
      \begin{itemize}
        \item Bedienung über Kommandozeile
      \end{itemize}
    \item Kommandozeile ist überlegenes Bedienkonzept
      \begin{itemize}
        \item Die meiste Zeit beim Arbeiten verbringen wir in der CLI
      \end{itemize}
    \item GUI versteckt die Details
      \begin{itemize}
        \item Details werden wichtig
        \item GUI kann versagen
      \end{itemize}
    \item GUIs sind nicht böse oder schlecht, man muss nur wissen, was dahinter steckt
  \end{itemize}
\end{frame}

\begin{frame}{Dateisystem}
  \begin{itemize}
    \item bildet \emph{einen} Baum
      \begin{itemize}
        \item beginnt bei \texttt{/} (root)
        \item \texttt{/} trennt Teile eines Pfads
        \item auf Groß-/Kleinschreibung achten!
      \end{itemize}
    \item es gibt ein aktuelles Verzeichnis
    \item relative vs. absolute Pfade
    \item spezielle Verzeichnisse:
      \begin{itemize}
        \item \texttt{.} das aktuelle Verzeichnis
        \item \texttt{..} das Oberverzeichnis
        \item \texttt{\textasciitilde} das Homeverzeichnis
      \end{itemize}
  \end{itemize}
\end{frame}

\begin{frame}{\texttt{man}, \texttt{pwd}, \texttt{cd}}
  \begin{tabu}{>{\ttfamily}l X}
    man \textit{topic}    & \enquote{manual}: zeigt die Hilfe für ein Programm \\
    pwd                   & \enquote{print working directory}: zeigt das aktuelle Verzeichnis \\
    cd \textit{directory} & \enquote{change directory}: wechselt in das angegebene Verzeichnis
  \end{tabu}
\end{frame}

\begin{frame}{\texttt{ls}}
  \begin{tabu}{>{\ttfamily}l X}
    ls [\textit{directory}] & \enquote{list}: zeigt den Inhalt eines Verzeichnisses an \\
    ls -l                   & \enquote{long}: zeigt mehr Informationen über Dateien und Verzeichnisse \\
    ls -a                   & \enquote{all}: zeigt auch versteckte Dateien (fangen mit \texttt{.} an)
  \end{tabu}
\end{frame}

\begin{frame}{\texttt{mkdir}, \texttt{touch}}
  \begin{tabu}{>{\ttfamily}l X}
    mkdir \textit{directory}    & \enquote{make directory}: erstellt ein neues Verzeichnis \\
    mkdir -p \textit{directory} & \enquote{parent}: erstellt auch alle notwendigen Oberverzeichnisse \\
    touch \textit{file}         & erstellt eine leere Datei
  \end{tabu}
\end{frame}

\begin{frame}{\texttt{cp}, \texttt{mv}, \texttt{rm}, \texttt{rmdir}}
  \begin{tabu}{>{\ttfamily}l X}
    cp \textit{source} \textit{destination}    & \enquote{copy}: kopiert eine Datei \\
    cp -r \textit{source} \textit{destination} & \enquote{recursive}: kopiert ein Verzeichnis rekursiv \\
    mv \textit{source} \textit{desination}     & \enquote{move}: verschiebt eine Datei (Umbenennung) \\
    rm \textit{file}                           & \enquote{remove}: löscht eine Datei (Es gibt keinen Papierkorb!) \\
    rm -r \textit{directory}                   & \enquote{recursive}: löscht ein Verzeichnis rekursiv \\
    rmdir \textit{directory}                   & \enquote{remove directory}: löscht ein \emph{leeres} Verzeichnis
  \end{tabu}
\end{frame}

\begin{frame}{\texttt{cat}, \texttt{less}, \texttt{grep}, \texttt{echo}}
  \begin{tabu}{>{\ttfamily}l X}
    cat \textit{file}                           & \enquote{concatenate}: gibt Inhalt einer (oder mehr) Datei(en) aus \\
    less \textit{file}                          & (besser als \texttt{more}): wie \texttt{cat}, aber navigabel \\
    grep \textit{pattern} \textit{file}         & \texttt{g/re/p}: sucht in einer Datei nach einem Muster \\
    grep -i \textit{pattern} \textit{file}      & \enquote{case insensitive} \\
    grep -r \textit{pattern} \textit{directory} & \enquote{recursive}: suche rekursiv in allen Dateien \\
    echo \textit{message}                       & gibt einen Text aus
  \end{tabu}
\end{frame}

\begin{frame}{Ein- und Ausgabe}
  \begin{tabu}{>{\ttfamily}l X}
    \textit{command} > \textit{file}      & überschreibt Datei mit Ausgabe \\
    \textit{command} >> \textit{file}     & fügt Ausgabe einer Datei hinzu \\
    \textit{command} < \textit{file}      & Datei als Eingabe \\
    \textit{command1} | \textit{command2} & Ausgabe als Eingabe (Pipe)
  \end{tabu}
\end{frame}

\begin{frame}{Tastaturkürzel}
  \begin{tabu}{>{\ttfamily}l X}
    Ctrl-C & beendet das laufende Programm \\
    Ctrl-D & EOF (end of file) eingeben, kann Programme beenden \\
    Ctrl-L & leert den Bildschirm
  \end{tabu}
\end{frame}

\begin{frame}{Globbing}
  \begin{tabu}{>{\ttfamily}l X}
    *                         & wird ersetzt durch alle passenden Dateien \\
    \{\textit{a},\textit{b}\} & bildet alle Kombinationen
  \end{tabu}

  \vspace{2cm}
  Beispiele:\\[10pt]
  \begin{tabu}{l c l}
    *.log           & → & \texttt{foo.log bar.log} \\
    foo.\{tex,pdf\} & → & \texttt{foo.tex foo.pdf}
  \end{tabu}
\end{frame}

\ExplSyntaxOn

\RenewDocumentEnvironment {block} {m o} {
  \IfValueTF{#2}{
    \begin{tcolorbox}[
      adjusted~title=#1,
      #2,
    ]
  }{
    \begin{tcolorbox}[
      adjusted~title=#1,
    ]
  }
}{
  \end{tcolorbox}
}

\RenewDocumentEnvironment {alertblock} {m o} {
  \IfValueTF{#2}{
    \begin{tcolorbox}[
      adjusted~title=#1,
      colframe=vertexDarkRed,
      #2,
    ]
  }{
    \begin{tcolorbox}[
      adjusted~title=#1,
      colframe=vertexDarkRed,
    ]
  }
}{
  \end{tcolorbox}
}

\RenewDocumentEnvironment {exampleblock} {m o} {
  \IfValueTF{#2}{
    \begin{tcolorbox}[
      adjusted~title=#1,
      colframe=blue!80!black,
      #2,
    ]
  }{
    \begin{tcolorbox}[
      adjusted~title=#1,
      colframe=blue!80!black,
    ]
  }
}{
  \end{tcolorbox}
}

\newcounter{heightgroup}

\NewDocumentEnvironment {CodeExplanation} {o m O{Code} O{Erklärung} D<>{}} {
  \NewDocumentCommand \Explanation {D<>{}} {
      \end{tcolorbox}
    \end{column}
    \begin{column}{
      \IfValueTF {#1} {
        #1
      }{
        \fp_eval:n {1 - 0.02 - #2}
      }
      \textwidth
    }
    \begin{tcolorbox}[
      equal~height~group=\theheightgroup,
      adjusted~title=#4,
      valign=top,
      ##1,
    ]
  }
  \begin{columns}[onlytextwidth, t]
    \begin{column}{#2 \textwidth}
      \begin{tcolorbox}[
        adjusted~title=#3,
        equal~height~group=\theheightgroup,
        valign=top,
        #5,
      ]
}{
      \end{tcolorbox}
    \end{column}
  \end{columns}
  \stepcounter{heightgroup}
}

\NewDocumentEnvironment {CodeExample} {o m O{Code} O{Ergebnis} D<>{}} {
  \NewDocumentCommand \CodeResult {D<>{}} {
      \end{tcolorbox}
    \end{column}
    \begin{column}{
      \IfValueTF {#1} {
        #1
      }{
        \fp_eval:n {1 - 0.02 - #2}
      }
      \textwidth
    }
      \begin{tcolorbox}[
        adjusted~title=#4,
        colframe=gray!40,
        coltitle=black,
        equal~height~group=\theheightgroup,
        valign=top,
        ##1,
      ]
        \begin{EmulateArticle}
  }
  \begin{columns}[onlytextwidth, t]
    \begin{column}{#2 \textwidth}
      \begin{tcolorbox}[
        adjusted~title=#3,
        equal~height~group=\theheightgroup,
        valign=top,
        #5,
      ]
}{
        \end{EmulateArticle}
      \end{tcolorbox}
    \end{column}
  \end{columns}
  \stepcounter{heightgroup}
}

\NewDocumentEnvironment {Packages} {} {
  \begin{block}{Benötigte~Pakete}
}{
  \end{block}
}

\makeatletter
\NewDocumentEnvironment {EmulateArticle} {} {
  \rmfamily
  \let\textrm=\latinmodernrm
  \let\textsf=\latinmodernsf
  \let\texttt=\latinmoderntt
  \renewcommand\thempfootnote{\arabic{mpfootnote}}

  \color{black}
  \setbeamercolor{item}{fg=black}
  \setbeamercolor{itemize/enumerate~body}{fg=black}
  \setbeamercolor{itemize/enumerate~subbody}{fg=black}
  \setbeamercolor{itemize/enumerate~subsubbody}{fg=black}
  \setbeamercolor{description~item}{fg=black}
  \setbeamercolor{enumerate~item}{fg=black}
  \setbeamercolor{itemize~item}{fg=black}
  \setbeamercolor{normal~text}{fg=black}
  \setbeamercolor{block~body}{fg=black}
  \setbeamerfont{item}{family=\rmfamily, size=\normalsize}
  \setbeamerfont{itemize/enumerate~body}{family=\rmfamily, size=\normalsize}
  \setbeamerfont{itemize/enumerate~subbody}{family=\rmfamily, size=\normalsize}
  \setbeamerfont{itemize/enumerate~subsubbody}{family=\rmfamily, size=\normalsize}
  \setbeamerfont{description~item}{series=\bfseries}
  \setbeamertemplate{itemize~item}{\bullet}
  \setbeamertemplate{itemize~subitem}{--}
  \setbeamertemplate{itemize~subsubitem}{\textasteriskcentered}
  \setbeamertemplate{enumerate~item}{\theenumi.}
  \setbeamertemplate{enumerate~subitem}{\alph{enumii})}
  \setbeamertemplate{enumerate~subsubitem}{\roman{enumiii}.}

  \setbeamerfont{footnote}{family=\rmfamily}
  \setbeamerfont{footnote~mark}{family=\rmfamily}

  \setbeamertemplate{caption}[numbered]
  \setbeamercolor{caption}{fg=black}
  \setbeamerfont{caption}{family=\rmfamily}
  \setbeamercolor{caption~name}{fg=black}
  \setbeamerfont{caption~name}{family=\rmfamily, series=\bfseries}

  % from amsmath.dtx
  % \def\maketag@@@#1{\hbox{\m@th\normalfont#1}}
  \def\maketag@@@##1{\hbox{\m@th\normalfont\rmfamily##1}}
  \hypersetup{linkcolor=black}
}{
}
\makeatother

\NewDocumentCommand \lstMakeSettingsInline {}
{
  language={[gnu]make},
  basicstyle=\ttfamily\color{black},
  otherkeywords={:, |, \\},
}

\NewDocumentCommand \lstMakeSettings {}
{
  \expandafter\lstset\expandafter{\lstMakeSettingsInline}
}

\NewDocumentCommand \lstmakeinline {O{}}
{
  \expandafter\lstinline\expandafter[\lstMakeSettingsInline,#1]
}

\box_new:N \l_listings_boxed_box
\lstnewenvironment{lstmake}[1][]{
  \lstMakeSettings
  \lstset{#1}
  \hbox_set:Nw \l_listings_boxed_box
}{
  \hbox_set_end:
  \leavevmode\box_use:N \l_listings_boxed_box
}

\NewDocumentCommand \lstPythonSettingsInline {}
{
  language=python,
  basicstyle=\ttfamily\color{black},
  otherkeywords={True, False, None, as},
  showstringspaces=false,
  stringstyle=\color{vertexDarkRed},
}

\NewDocumentCommand \lstPythonSettings {}
{
  \expandafter\lstset\expandafter{\lstPythonSettingsInline}
}

\NewDocumentCommand \lstinputpython {O{} m}
{
  \lstPythonSettings
  \hbox_set:Nw \l_listings_boxed_box
  \lstinputlisting[#1]{#2}
  \hbox_set_end:
  \leavevmode\box_use:N \l_listings_boxed_box
}

\NewDocumentCommand \lstpythoninline {O{}}
{
  \expandafter\lstinline\expandafter[\lstPythonSettingsInline,#1]
}

\lstnewenvironment{lstpython}[1][]{
  \lstPythonSettings
  \lstset{#1}
  \hbox_set:Nw \l_listings_boxed_box
}{
  \hbox_set_end:
  \leavevmode\box_use:N \l_listings_boxed_box
}

\NewDocumentCommand \lstMatplotlibrcSettings {}
{
  \lstset{
    basicstyle={\ttfamily\color{black}},
    morecomment={[l]\#},
    otherkeywords={:, \{, \}},
  }
}

\NewDocumentEnvironment {CenterStrip} {O{\textwidth} m}
{
  \begin{minipage}[c][#2\baselineskip][c]{#1}
}{
  \end{minipage}
}

\NewDocumentCommand \OverfullCenter {+m}
{
	\noindent
	\makebox[\linewidth]{#1}
}

\tikzstyle{buttonstyle} = [
  align=center,
  rectangle,
  fill=black!10,
  draw=black!80,
  drop~shadow,
  font={\bfseries},
]

\NewDocumentCommand \button {m}
{
  \tikz[
    baseline={([yshift=-.8ex]current~bounding~box.center)},
  ]{
    \node[buttonstyle] {\normalsize #1};
  }
}

\NewDocumentCommand \doc {m m}
{
  \button{\href{#1}{Doku:~\texttt{#2}}}
}

% used to show headline at beginning of each logical section
\NewDocumentCommand \headlineframe {m}
{
  \begin{frame}
    \begin{center}
      \Huge\color{vertexDarkRed}#1
    \end{center}
  \end{frame}
}

\NewDocumentCommand \removedisplayskip {}
{
  \setlength\abovedisplayskip{0ex}
  \setlength\belowdisplayskip{0ex}
  \setlength\abovedisplayshortskip{0ex}
  \setlength\belowdisplayshortskip{0ex}
}

\makeatletter
\seteverylogo
{
  \setlogodrop{0.7ex}
  \setLaTeXa{{\scshape a}}
  \fp_compare:nNnTF {\f@size} > {10}
  {
    \setlogokern{La}{-0.33em}
  }
  {
    \fp_compare:nNnTF {\f@size} < {10}
    {
      \setlogokern{La}{-0.4em}
    }
    {
      \setlogokern{La}{-0.36em}
    }
  }
}
\makeatother

\let\TeXold=\TeX
\def\TeX{\latinmodernrm{\TeXold}}
\def\eTeX{$ε$-\TeX}
\let\LaTeXold=\LaTeX
\def\LaTeX{\latinmodernrm{\LaTeXold}}
\let\XeTeXold=\XeTeX
\def\XeTeX{\latinmodernrm{\XeTeXold}}
\let\XeLaTeXold=\XeLaTeX
\def\XeLaTeX{\latinmodernrm{\XeLaTeXold}}
\let\LuaTeXold=\LuaTeX
\def\LuaTeX{\latinmodernrm{\LuaTeXold}}
\let\LuaLaTeXold=\LuaLaTeX
\def\LuaLaTeX{\latinmodernrm{\LuaLaTeXold}}
\def\pdfTeX{\latinmodernrm{pdf\TeX}}
\def\BibTeX{\latinmodernrm{\textsc{Bib}\TeX}}
\def\BibLaTeX{\latinmodernrm{Bib\LaTeX}}

% macro examples
\NewDocumentCommand \dif {m}
{
  \mathinner{\symup{d} #1}
}
\NewDocumentCommand \Dif {m}
{
  \mathinner{\symup{D} #1}
}
\NewDocumentCommand \del {m}
{
  \mathinner{\symup{\delta} #1}
}
\NewDocumentCommand \Del {m}
{
  \mathinner{\symup{\Delta} #1}
}

\let\vaccent=\v
\RenewDocumentCommand \v {}
{
  \TextOrMath{
    \vaccent
  }{
    \symbf
  }
}

\let\ltext=\l
\RenewDocumentCommand \l {}
{
  \TextOrMath { \ltext }{ \mleft }
  % \mathopen{}\mathclose\bgroup\left
}
\let\raccent=\r
\RenewDocumentCommand \r {}
{
  \TextOrMath { \raccent }{ \mright }
  % \aftergroup\egroup\right
}

\AtBeginDocument{
  \let\symRe=\Re
  \RenewDocumentCommand \Re {}
  {
    \operatorname{Re}
  }
  \let\symIm=\Im
  \RenewDocumentCommand \Im {}
  {
    \operatorname{Im}
  }
}

\DeclarePairedDelimiter{\abs}{\lvert}{\rvert}
\DeclarePairedDelimiter{\norm}{\lVert}{\rVert}
\DeclarePairedDelimiter{\bra}{\langle}{\rvert}
\DeclarePairedDelimiter{\ket}{\lvert}{\rangle}
\DeclarePairedDelimiterX{\braket}[2]{\langle}{\rangle}{#1\delimsize|#2}

\ExplSyntaxOff
